\section{Linguística}

Um sistema de símbolos físicos possui os meios necessários e suficientes para realizar ações inteligentes de maneira geral \cite[116]{newell1976ComputerSA}, o nosso dialeto é uma exemplificação disso. A primeira área cognitiva do ser humano é a capacidade se obter descrições de forma simplificada através da linguagem \cite[131]{putnam1975mind}

A gramática nada mais seria do que um conjunto finito ou infinito de sentenças e um numero finito de fonemas ou letras em seu alfabeto. Pensando como uma máquina, esse conjunto de regras seguiria um estado racional e a troca ou inserção de fonemas durante a troca de estados poderia mudar o sentido da frase em questão \cite[13-16]{chomsky2002syntactic}.

\subsection{Analise de Discurso}
// TODO

\subsection{Processamento de Linguagem Natural}
Partindo desse principio o que impediria o computador de entender o que lhe foi dito? Mesmo parecendo óbvio, entender palavras não significa entender o contexto, é necessário se familiarizar com o ambiente e o momento afim de idealizar o que está sendo transmitido. Essa conexão entre elementos é tratada no estudo do \textit{connectivism}\footnote{Integração dos princípios de rede, caos, complexidade e teorias de auto-organização. Seu objetivo é entender decisões baseado nas mudanças de componentes fundamentais \cite[5]{siemens2014connectivism}.}. De acordo com a linha de pensamento, estabelecida pelo estudo, os neurônios seriam os agentes cognitivos responsáveis por planejar, construir e representar essas informações que nosso cérebro recebe \cite[22]{brandura1996}. Mesmo com toda essa relevância o assunto não tinha tanta visibilidade até o nascimento  do campo hibrido chamado de Processamento de Linguagem Natural. \cite[16]{russell2003artificial}
