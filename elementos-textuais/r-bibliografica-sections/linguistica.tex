\section{Linguística}
A gramática é composta por: um conjunto finito de letras que formam o chamado alfabeto e um conjunto de regras e normas. Utilizando das regras e do finito número de palavras formadas a partir das letras, é possível se expressar através de uma sentença. O estado representado por essa sentença pode variar de acordo com a regra aplicada. É impossível cobrir todos os estados com uma única regra pelo motivo de existirem números infinitos de sentenças a serem formadas \cite[13-25]{chomsky2002syntactic}. Essa capacidade de obter descrições de forma simplificada através da linguagem é a primeira área cognitiva do ser humano \cite[131]{putnam1975mind}.

No dia-a-dia, existem multiplos fatores que ajudam a entender o sentido de uma frase, porem, em uma máquina os mesmos fatores muitas vezes não se aplicam. Nessa sessão o enfoque é em introduzir alguns estudos fomentados pela linguística. Na inteligência artificial, o ato de juntar símbolos (padrões físicos) em expressões (estruturas) utilizando um conjunto de regras (processos), é considerado um sistema de símbolos físicos. Acredita-se que um sistema desse nesse formato possui os meios necessários e suficientes para realizar ações inteligentes de forma geral \cite[116]{newell1976ComputerSA}. Entretanto, o que foi escrito é diferente do que é compreendido, vide duplo sentidos, logo o contexto é necessário.

% sections
\subsection{Analise de Discurso}
No decorrer de um texto (que é algo concreto), pode-se caracterizar diversos níveis de geração de sentido.
A primeira formulação de sentido vem do discernimento de termos dentro de um contexto, esse nível é chamado de fundamental. Após distinguir esse primeiro sentido ele é aplicado pelo autor através de um sujeito fazendo com que a prosa tome uma direção, esse nível é chamado narrativo. Por fim, existe o nível discursivo, relacionado as escolhas de tempo, espaço, pessoa e figura durante a narrativa dos fundamentos, dando a essa narrativa um ponto de vista. Logo, o termo discurso é dado como um suporte abstrato por trás do texto, afim da concretização da sua ideia central \cite[13-17]{gregolin1995ad}.

A analise de discurso é, de forma sucinta, uma analise do que foi dito, de como foi dito e qual o sentido do que foi dito. As primeiras manifestações do assunto foram no século XX com autores russos que, além de isolar e definir elementos de uma linguagem poética queriam definir determinantes por trás do perfil artístico do escritor. O tempo fez com que a analise de discurso se desenvolvese e ramificasse em várias vertentes, uma delas a francesa, que apoia a possibilidade de automatizar essa analise por meio da informática. A área continua sendo um campo complexo e de contínuo estudo por trás das definições e metodologias para abordar e sustentar as novas unidades de analise. \cite[22]{souza2006ad}.

Os discursos se diferem de pessoa para pessoa devido ao nível discursivo, a necessidade de expressar um determinado sentido leva o autor a se colocar em um ponto de vista durante sua narrativa. Do contexto da pesquisa, entender o discurso do usuário para mapear o motivo do seu estado mental é um fator de total relevância para entender o estado dele. A pesquisa realizada por Modesto Leite \cite[134]{modesto2005adepre}, mostra em seus resultados que os discursos apresentados pelos pacientes fundamentavam o motivo psicológico do por que os mesmo teriam o transtorno. 

Partindo dos principio apresentados sobre um discurso, por mais que as palavras sejam localizadas, o ponto chave da discussão está em como um computador seria capaz de inferir o sentido da frase. Existem áreas, seguindo os campos multidiciplinares que envolvem linguitica e computação, responsáveis por garantir que o processamento dos textos gerará os resultados esperados.

\subsection{Processamento de Linguagem Natural}
Se entender palavras não significa entender o contexto, logo, se familiarizar com o ambiente e o momento afim de idealizar o que está sendo transmitido é algo necessário. Essa conexão entre elementos é tratada no estudo do \textit{connectivism}\footnote{Integração dos princípios de rede, caos, complexidade e teorias de auto-organização. Seu objetivo é entender decisões baseado nas mudanças de componentes fundamentais \cite{siemens2014connectivism}.}. De acordo com a linha de pensamento, estabelecida pelo estudo, os neurônios seriam os agentes cognitivos responsáveis por planejar, construir e representar essas informações que o cérebro humano recebe. Criar soluções para problemas pontuais que envolvam a língua que é utilizado no dia-a-dia se uma pessoa, essa é a definição por trás do Processamento de Linguagem Natural (PLN). Fornecer dados linguísticos que a maquina não é capaz de inferir, ou que seja necessário uma ajuda para seu melhor desempenho, é o ponto principal dessa área \cite{brandura1996, maria2015npl}.

Já que o PLN será inicialmente utilizado para análise de palavras-chaves e padrões, é necessário vislumbrar que será necessário um conjunto de regras a fim de melhorar uma determinada métrica durante o processo de aprendizagem. Para que isso seja possível, um conhecimento dentro da área de psicologia se torna altamente relevante.

