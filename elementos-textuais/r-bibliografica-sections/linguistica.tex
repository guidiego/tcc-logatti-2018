\section{Linguística}

Um sistema de símbolos físicos possui os meios necessários e suficientes para realizar ações inteligentes de maneira geral \cite[116]{newell1976ComputerSA}, o nosso dialeto é uma exemplificação disso. A primeira área cognitiva do ser humano é a capacidade se obter descrições de forma simplificada através da linguagem \cite[131]{putnam1975mind}

A gramática nada mais seria do que um conjunto finito ou infinito de sentenças e um numero finito de fonemas ou letras em seu alfabeto. Pensando como uma máquina, esse conjunto de regras seguiria um estado racional e a troca ou inserção de fonemas durante a troca de estados poderia mudar o sentido da frase em questão \cite[13-16]{chomsky2002syntactic}.

\subsection{Analise de Discurso}
Antes de entrar no tema dessa subsessão, é necessário entender o que é um discurso. Dentro de um texto (que é algo concreto), pode-se caracterizar diversos niveis de geração de sentido, o \textbf{fundamental}, responsavel pela primeira formulação desse sentido a partir do discernimento de termos dentro de um contexto. O \textbf{narrativo}, é o modo com que o autor utiliza dos valores fundamentais a partir de um sujeito tomando a diretiva da narrativa. E por fim, o \textbf{discursivo}, que é o modo mais superficial da geração de sentido, e tem como pauta a série de escolhas do autor ao decorrer da narrativa dos fundamentos a partir de um determinado ponto de vista. Logo, o termo \textbf{discurso} é dado como um suporte abstrato por trás do texto, afim da concretização da sua idéia central \cite[13-17]{gregolin1995ad}.

A Analise de Discurso, nada mais é do que realizarmos uma analise de "do que?" e de "como?" um texto diz algo, essa chamada de analise interna, e o "por que o texto diz", chamada de analise externa.  O estudo dessa area teve as primeiras manifestações no século XX com autores Russos que alem de isolar e definir elementos de uma linguagem poética queriam definir determinantes por trás do perfil ártistico do escritor. Depois de passar por várias manifestações ao longo do tempo e se quebrar em varias vertentes como a francesa, que apoia a possibilidade de automatizar essa analise por meio da informatica, a AD continua sendo um campo complexo e de continuo estudo por trás das definições e metodologias para abordar e sustentar as novas unidades de analise. \cite[22]{souza2006ad}.

Os dicursos varião de ser para ser devido ao nivel discursivo de entendimento, a necessidade de expressar um determinado sentido leva o autor a se colocar em um ponto de vista durante sua narrativa, do contexto da pesquisa, mapear o motivo da depressão é o fator chave para analisarmos que tipo de discurso nossa amostra teria. A pesquisa realizada por Modesto Leite \cite[134]{modesto2005adepre}, afirma nos seus resultados que os discursos dos pacientes fundamentavam o motivo psicológico do por que os mesmo teriam o transtorno. Sem uma analise coerente do discursos de nossa amostra ainda se faz possivel a geração de dados relevantes, porem, as chances de não avançarmos em acertividade se torna grande.  

\subsection{Processamento de Linguagem Natural}
Partindo desse principio o que impediria o computador de entender o que lhe foi dito? Mesmo parecendo óbvio, entender palavras não significa entender o contexto, é necessário se familiarizar com o ambiente e o momento afim de idealizar o que está sendo transmitido. Essa conexão entre elementos é tratada no estudo do \textit{connectivism}\footnote{Integração dos princípios de rede, caos, complexidade e teorias de auto-organização. Seu objetivo é entender decisões baseado nas mudanças de componentes fundamentais \cite[5]{siemens2014connectivism}.}. De acordo com a linha de pensamento, estabelecida pelo estudo, os neurônios seriam os agentes cognitivos responsáveis por planejar, construir e representar essas informações que nosso cérebro recebe \cite[22]{brandura1996}. Mesmo com toda essa relevância o assunto não tinha tanta visibilidade até o nascimento  do campo hibrido chamado de Processamento de Linguagem Natural. \cite[16]{russell2003artificial}
