\section{Aprendizado de Máquina}
Se máquinas pensam ou não, o enfoque da IA é executar diretamente um processo que um ser humano aprendeu ao longo dos anos, e não, algo que implicitamente nasceu com ele. Logo, o fluxo mais sistemático para conseguir que uma máquina abstrai formalmente um conhecimento seria mapear um estado inicial, aplicar dados para sua aprendizagem e submete-la a outras experiências afim de reafirmar esses conhecimentos. Essa abordagem foi proposta por Turing em 1950, entretanto, o campo de pesquisa conhecido como \textit{machine learning} ou aprendizado de máquina surgiu, apenas em 1959 quando o termo nasceu ao ser aplicado em um estudo de redes neurais. Portanto, pode-se brevemente descrever o aprendizado de máquina como métodos computacionais designados a utilizarem de algoritmos precisos de predição juntamente a uma base de conhecimento para aumentar a sua assertividade\cite[1]{turing1950, samuel1959some, mohri2012foundations}.

Nessa sessão é dado um enfoque maior em \textit{machine learning}. Primeiramente serão introduzidos alguns conceitos que são utilizados a partir dessa sessão. Com os conceitos passados, sera apresentado, de maneira lacônica, a aplicação dos conceitos perante algoritmos de aprendizado de máquina. Por fim, é sugerido algumas ferramentas utilizadas para implementação do \textit{machine learning} e um breve enfoque em como é guiado esse campo dentro desta pesquisa.

\subsection{Conceitos do Aprendizado de Máquina}
Previamente foi dito que, seriam necessários dados de treino, esse dado bruto, como já dito também, contem várias propriedades que podem ser ou não processadas afim de gerar informações, esse dado é conhecido com dados de treinamento e um dado que contém as informações processadas necessárias, inserida por algoritimos ou por fatores externos, é chamado de dado rotulado ou anotado.

As propriedades podem ser utilizadas para classificar um determinado dado, isso as concede o nome de atributo. Também conhecido com \textit{feature} é considerada como um dos fatores mais relevantes para o aprendizado da maquina. O termo conhecido como \textit{feature engineering} ou, em tradução literal, engenharia de atributos, diferente do processo de aprendizagem em si, são baseados diretamente a suas entidades e em como escolher e pré-processar seus atributos chaves afim de otimizar o aprendizado \cite{domingos2012few}.

Esses atributos, dentre outras propriedades, estão em uma massa de dados. Por sua vez serão utilizadas como dado de entrada para um agente. Esse agente tambem pode ser representado, por fins explicativos, como uma simples função \textit{f(x) = y}, o objetivo é treinar o agente para que ele seja capaz de generalizar. Ou seja, capaz de predizer o melhor valor para \textit{y} a partir da entrada \textit{x}. Quando o resultado final depende totalmente do dado de entrada, ou seja, não existem muitas variações que levem a saída a grandes divergências, dizemos que o problema generaliza bem.

Essa função gerada que sucintamente traduz os padrões encontrados entre os dados, gera o que conhecemos como modelo. Muitos deles são utilizados para classificar o dado dentro de um grupo de possibilidades, esse tipo de modelo é descrito como classificador.

Os problemas a serem processados pelos classificadores ou outros modelos possuem um conjunto de  hipóteses. Quando é concebida uma preferencia a uma dessas hipóteses com a premissa de induzir o modelo a tomar uma decisão chamamos esse ato de tendência.

Uma quantidade extensa de hipóteses e dados de entrada não significam uma maquina mais assertiva, quando um agente é treinado para executar mais do que o necessário, um fenômeno peculiar chamado de \textit{overfitting} ou sobre-ajustes acontece, esse evento faz com que a maquina seja capaz de tomar decisões assertivas apenas para um conjunto de dados a qual foi submetida e não consegue predizer novas entradas.

As descrições dadas por Russel \cite[693]{russell2003artificial} ainda se aplicam a conceitos mais complexos que não cabem a essa pesquisa explicitamente. O jeito que é utilizado os dados ou os atributos previamente ditos, podem resultar em diversos tipos de abordagens diferentes.

\subsection{Algoritmos de Aprendizado de Máquina}
Os atributos presente nos dados podem ser utilizados com a premissa de predizer uma determinada hipótese. O ato de coletar um conjunto de dados para treino baseados em entra-saida e, posteriormente, utiliza-lo para predizer uma ou mais hipóteses é nomeado \textbf{aprendizagem supervisionada} e pode-se entender melhor o contexto apresentado na <figura 3>. O mais comum cenário dentro da aprendizagem supervisionada consiste em utilizar de um atributo discreto\footnote{todo} ou qualitativo\footnote{todo}, a partir de um indutor, para gerar o melhor classificador para aquele problema, esse método é chamado \textbf{classificação}. Ainda existe um outro método muito comum tambem, que se baseia em atributos continuos\footnote{todo}, com isso é possivel utilizar de predição numérica afim de identificar um modelo dentro do plano cartesiano e prever futuros valores para o mesmo atributo em outras situações, esse métdo se chama \textbf{regressão}. \cite{hastie2009unsupervised, russell2003artificial}

Em contra partida, existem casos onde os atributos não estaram anotados e a IA terá que literalmente inferir a probabilidade numérica em toda a base. Esse estilo de aprendizagem se chama \textbf{aprendizagem não supervisionada}, e o enfoque é utilizar da segregação dos atributos e suas dimensões para inferir resultados. O método mais comum dentro desse estilo é a \textbf{clusterização}, em resumo, ela se baseia em particionar os dados em grupos dentro de um plano cartesiano a partir do dado de inferencia. Outra pratica utilizada nesse tipo de aprendizagem é \textbf{redução dimensional}, ela se baseia em resumir o estado atual de um item para um estado de menor complexidade de dimensões com base nas propriedades chaves \cite{hastie2009unsupervised, mohri2012foundations}.

Ainda co-relacionado com as últimas explicações, existe um tipo onde é inserido no agente dados anotados e não anotados para pedrição de todos os items da base. A \textbf{aprendizagem semi-supervisionada}, sucintamente explicando, é a mescla das duas outras aprendizagens, visando o resultado, utiliza da parte anotada (supervisionada) e da lógica de grupos (não-supervisionada), para normalizar e optimizar o resultado final \cite[7]{mohri2012foundations}.

Por último, existem casos onde não teremos dados suficientes para executar outros tipos de aprendizagem, nesse momento a aprendizagem baseada no tradicional tentativa e error. Nomeada \textbf{aprendizado por reforço}, se observar a figura:3, pode-se notar que o funcionamento desse tipo de aprendizado se consiste em um \textbf{ambiente} (A), responsavel por emitir um estado para o \textbf{component} (C), feito isso o é aplicado uma \textbf{entrada de dados} (e) ao nosso \textbf{agente} (G) que será responsavel por tomar a decisão de que \textbf{ação} (a) tomara para a entrada recebida. Essa ação modifica o estado do ambiente e transmite um sinal de reforço visando através do componente (r) para que a aplicação tome a melhor escolha ao longo prazo \cite{kaelbling1996reinforcement, russell2003artificial}.


Não será definido exemplos práticos dessos algoritimos dados, durante nosso resultado e conclusão abordaremos a utilização de algumas das abordagens surgeridas dentro de nossa pesquisa e o motivo da escolha.

Indiferente do tipo ou situação em que o algoritimo se enquadra, a mutiplicidade de escolhas propostas pelo machine learning gerou várias propostas de aprendizado, algumas até tentando utilizar da estrutura proposta pela neurociencia para replicar nossoas redes neurais.





\subsection{Baeysan Network, Neural Network e Deep Learning}
TO DO...