\subsection{Dimensões Afetivas Negativas}
Se caracteriza \textbf{Dimensão Afetiva}, tambem conhecida como afetividade, um conjunto de sentimentos que nos afetão positiva ou negativamente \cite{pinto2009afetos}. Logo, quando abordado as \textbf{negativas}, pode-se pensar em tristeza, inveja e desesperança por exemplo, porem, o contexto da pesquisa sera limitado as dimensões afetivas negativas: depressão, ansiedade e stress.

A \textbf{depressão} é uma forma atenuada de \textbf{melancolia} \cite{roudinesco2000psicanalise}, que por sua vez, seria uma condição maléfica de enfraquecimento da sáude mental de um ser. Classificada como \textbf{transtorno de humor}, a depressão, diferente de outras variações mais regulares de humor, pode causar grandes danos a vida cotidiana uma vez que, por definição, altera a percepção de si mesmo maximizando o peso dos seus problemas diante de sua própria pespectiva \cite{esteves2006depressao}. A melancolia e a depressão compartilham de sintomas em comum como: Desanimo, perca de interesse, inibição, bloqueio de sentimentos e outros mais \cite[28]{freud2014livro}.

Segundamente existe a \textbf{Ansiedade}, muitas das disordem relacionadas a ansiedade são categorizadas pelo medo e evasão do usuario a um assunto especifico \cite[393]{dsmiv}. Esse sentimento é primitivo e tem fortes semelhanças as reações animais de defesa ao se colocarem em um abiente hostil. A inibição, assim como na depressão, é um dos sintomas dos transtornos relacionados a ansiedade, porem, o comportamento do ser assume um formato mais similar a um excesso de vigilancia e preocupação \cite{margis2003relaccao}.
