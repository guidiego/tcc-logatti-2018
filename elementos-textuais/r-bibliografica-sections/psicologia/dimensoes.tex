\subsection{Dimensões Afetivas Negativas}
Se caracteriza dimensão afetiva, também conhecida como afetividade, um conjunto de sentimentos que nos afetam positiva ou negativamente \cite{pinto2009afetos}. Logo, quando abordado as negativas, pode-se pensar em tristeza, inveja e desesperança por exemplo, porém, o contexto da pesquisa será limitado as dimensões afetivas negativas: depressão, ansiedade e stress.

A depressão é uma forma atenuada de melancolia \cite{roudinesco2000psicanalise}, que por sua vez, seria uma condição maléfica de enfraquecimento da saúde mental de um ser. Classificada como transtorno de humor, a depressão, diferente de outras variações mais regulares de humor, pode causar grandes danos a vida cotidiana uma vez que, por definição, altera a percepção de si mesmo maximizando o peso dos seus problemas diante de sua própria perspectiva \cite{esteves2006depressao}. A melancolia e a depressão compartilham de sintomas em comum como: desânimo, perca de interesse, inibição, bloqueio de sentimentos e outros mais \cite[28]{freud2014livro}. No Brasil há uma média de 11 milhões de pessoas afetadas por esse transtorno \cite{paho2017-letstalk}, sem contar os demais transtornos correlacionados, a partir desses dados já se nota a importância do assunto e as propostas feitas para auxiliar no seu combate.

Existe também a ansiedade. Muitas das desordem relacionadas a ansiedade são categorizadas pelo medo e evasão do usuário a um assunto especifico \cite[393]{dsmiv}. Esse sentimento é primitivo e tem fortes semelhanças as reações animais de defesa ao se colocarem em um ambiente hostil. A inibição, assim como na depressão, é um dos sintomas dos transtornos relacionados a ansiedade, porém, o comportamento do ser usuário assume um formato mais similar a um excesso de vigilância e preocupação \cite{margis2003relaccao}.

Durante os estudos das dimensões afetivas negativas, algumas dificuldades em analisar casos de ansiedade e depressão foram encontradas. Com um menor impacto ao estado clinico, porém, podendo apresentar estados inicias dos outros dois transtornos, o agrupamento de sintomas composto por tensão, irritabilidade e dificuldades para relaxar foi denominado Stress. \cite{lovibond1995structure, ribeiro2004contribuiccao, margis2003relaccao}

Já estabelecidas as diferenças e relações entre a ansiedade, depressão e o stress, tornar-se questionável como mensurar seu impacto em um ser humano. Essas dimensões foram selecionadas devido a existência de métodos já testados para analisar e avaliar pacientes que as portem.


