\subsection{Dimensões Afetivas Negativas}
Se caracteriza \textbf{Dimensão Afetiva}, tambem conhecida como afetividade, um conjunto de sentimentos que nos afetão positiva ou negativamente \cite{pinto2009afetos}. Logo, quando abordado as \textbf{negativas}, pode-se pensar em tristeza, inveja e desesperança por exemplo, porem, o contexto da pesquisa sera limitado as dimensões afetivas negativas: depressão, ansiedade e stress.

A \textbf{depressão} é uma forma atenuada de melancolia \cite{roudinesco2000}. Desanimo, perca de interesse, inibição e bloqueio de sentimentos são alguns sintomas exibidos por pessoas melancólicas \cite[276]{freud}. A \textbf{melancolia} seria uma condição maléfica de enfraquecimento da sáude mental de um ser. Partindo do principio de Fairbain onde o ser humano busca por gratificação, a não gratificação poderia ser o motivo de um estado melancólico. Classificada como \textbf{transtorno de humor}, a depressão, diferente de outras variações mais regulares de humor, pode causar grandes danos a vida cotidiana uma vez que, por definição, altera a percepção de si mesmo maximizando o peso dos seus problemas diante de sua própria pespectiva. Por tais motivos, a melancolia e a depressão compartilham de sintomas similares, entretando, a dinamica de suas origens, relações e concepções podem criar diversas perspectivas o que leva ao ponto de como se pode medir algo tão abstrado. \cite{}

Segundamente existe a \textbf{Ansiedade}

E por final o \textbf{Stress}
