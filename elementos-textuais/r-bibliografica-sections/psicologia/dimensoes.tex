\subsection{Dimensões Afetivas Negativas}
Se caracteriza \textbf{dimensão afetiva}, também conhecida como \textbf{afetividade}, um conjunto de sentimentos que nos afetam positiva ou negativamente \cite{pinto2009afetos}. Logo, quando abordado as \textbf{negativas}, pode-se pensar em tristeza, inveja e desesperança por exemplo, porem, o contexto da pesquisa será limitado as dimensões afetivas negativas: depressão, ansiedade e stress.

A \textbf{depressão} é uma forma atenuada de \textbf{melancolia} \cite{roudinesco2000psicanalise}, que por sua vez, seria uma condição maléfica de enfraquecimento da saúde mental de um ser. Classificada como \textbf{transtorno de humor}, a depressão, diferente de outras variações mais regulares de humor, pode causar grandes danos a vida cotidiana uma vez que, por definição, altera a percepção de si mesmo maximizando o peso dos seus problemas diante de sua própria perspectiva \cite{esteves2006depressao}. A melancolia e a depressão compartilham de sintomas em comum como: desanimo, perca de interesse, inibição, bloqueio de sentimentos e outros mais \cite[28]{freud2014livro}. No Brail uma média de 11 milhões de pessoas são afetadas por esse transtorno \cite{paho2017-letstalk}.

Segundamente existe a \textbf{ansiedade}, muitas das desordem relacionadas a ansiedade são categorizadas pelo medo e evasão do usuário a um assunto especifico \cite[393]{dsmiv}. Esse sentimento é primitivo e tem fortes semelhanças as reações animais de defesa ao se colocarem em um ambiente hostil. A inibição, assim como na depressão, é um dos sintomas dos transtornos relacionados a ansiedade, porem, o comportamento do ser usuário assume um formato mais similar a um excesso de vigilância e preocupação \cite{margis2003relaccao}.

Durante os estudos das dimensões afetivas negativas, foram expostas algumas dificuldades em analisar casos de ansiedade e depressão por alguns fatores em comum que apresentavam menor impacto ao estado clinico, porem, conseguiria prever e até serem pontos inicias para o desenvolvimento de outras desordem, esse agrupamento de sintomas como tensão, irritabilidade e dificuldades para relaxar foi denominada \textbf{Stress}. \cite{lovibond1995structure, ribeiro2004contribuiccao, margis2003relaccao}

Já estabelecidas as diferenças e relações entre as dimensões tornar-se questionável como mensurar seu impacto em um ser. Essas dimensões foram selecionadas devido a existência de métodos já testados para analisar e avaliar pacientes que as portem.
