\subsection{Psicologia Cognitiva}
Os estudos que visavam entender dos animais e humanos a capacidade de pensar, memorizar, perceber e no caso humano utilizar um dialeto (linguagem), foi uma área que surgiu antes mesmo da aparição das pesquisas sobre inteligencia artifical e sua vontade de entender os mesmos préceitos. Essa área da psicologia ficou conhecida como \textbf{psicologia cognitiva}.

O poder humano de realizar mutiplas tarefas, o questionamento de um cadeia de eventos físicos cancelar ou não uma outra e a recente popularidade e divergência da língua, fez com que em 1958 surgisse o primeiro modelo psicológico \cite[4-7]{broadbent1958perception}.

As técnicas experimentais, como os modelos psicológicos, impactaram toda a área interdiciplinar das \textbf{ciencias cognitivas}. Novas abordagens como a junção de modelos criados em computação para IA e técnicas experiementais da psicologia foram criadas a fim do entendimento da mente humana. Durante a criação do \textit{General Problem Solver}, por exemplo, Newell e Simon propuseram não só a implementação de algoritimos descritos por antigos estudiosos afim de resolver um leque de problemas, mas tambem, entender como a maquina estava realizando isso, assim, seria possivel uma comparação com a mente humana \cite[3-5]{newell1961gps, russell2003artificial}.

Nessa pesquisa, a psicologia cognitiva tera uma grande impacto em entender as \textit{features} da IA, uma vez que, sera necessário entender quais parametros serão analisados em pró de demonstrar a compatibilidade entre dois textos. Porem, alem disso, entender o conceito da psicologia nos ajuda a entender um dos vários fundamentos da IA que serão apresentados na próxima sessão.