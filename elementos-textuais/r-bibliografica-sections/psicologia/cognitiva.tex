\subsection{Psicologia Cognitiva}
Dentro da Psicologia existe um ramo cujo o objetivo é entender a capacidade animal de pensar, memorizar, perceber e no caso humano utilizar um dialeto (linguagem), essa área de estudo ficou conhecida como psicologia cognitiva. \cite[3-9]{neisser2014cognitive}

O ato de entender os procedimentos cognitivos adotados pelo ser humano se tornou mais relevante após a proposta da criação de uma inteligência artificial. O questionamento de como uma decisão poderia gerar, afetar ou cancelar uma cadeia de eventos, seja ela durante um pensamento ou uma ação, partindo da capacidade humana de realizar múltiplas tarefas, algumas delas complexas como interpretar linguagens com tantas divergências, fez com que em 1958\cite[4-7]{broadbent1958perception} surgisse o primeiro modelo psicológico que propunha um fluxo similar ao processamento de informações de um computador.

As técnicas experimentais, como os modelos psicológicos, impactaram toda a área interdisciplinar das ciencias cognitivas. Novas abordagens como a junção de modelos criados em computação para IA e técnicas experimentais da psicologia foram criadas a fim do entendimento da mente humana. Durante a criação do \textit{General Problem Solver} ou em tradução literal Solucionador Geral de Problemas, por exemplo, Newell e Simon propuseram não só a implementação de algoritmos descritos por antigos estudiosos afim de resolver um série de problemas, mas também, entender como a máquina estava realizando isso, assim, seria possível uma comparação com a mente humana \cite{newell1961gps, russell2003artificial}.

Nessa pesquisa, a psicologia cognitiva terá um grande impacto em entender como as amostras que contem ou não sintomas causados pelas dimensões afetivas negativas se comportam e pensam. Esse entendimento é essencial para o mapeamento dos atributos que serão processados pela IA afim de gerar os modelos para predição em dados ainda não analisados. Além disso, a psicologia cognitiva faz parte de todo o processo qualitativo de criação, devido a ser um dos vários campos que compões a multidisciplinaridade dentro da IA.
