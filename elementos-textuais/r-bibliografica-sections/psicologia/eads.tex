\subsection{Escala Depressão, Ansiedade e Stress}
Adotando um modelo dividido em três sub-escalas a Escala de Depressão, Ansiedade e Stress ou EADS foi proposta na premissa de uma maior assertividade na analise das dimensões afetivas negativas. Já existiam outras pesquisas com proposta similar como a Escala de Ansiedade de Beck e a Escala de Depressão de Beck. As diferenças entre essas escalas e a EADS são: a proposta de correlação entre os transtornos, a introdução de uma escala para o stress e a execução para obter os resultados partindo dessas novas abordagens.

A EADS completa é composta de 42 itens, para fins de facilitar a geração de dados será utilizado o modelo de 21 itens. Essa escala, como já foi citado, é dividida em três sub-escalas representando a depressão, ansiedade e stress. Essa divisão é simétrica, ou seja, cada sub-escala contem sete itens. O mais interessante é o fato, também ja abordado, da EADS propor uma correlação entre as dimensões, logo, um item pode pertencer a uma sub-escala, porem afetar uma segunda. Os itens dentro dessas sub-escalas são afirmações que podem ser respondidas por números de 1 a 4 e representam desde "não se aplica a mim" até "se aplicou a mim na maior parte das vezes". Representam as dimensões mais negativas os maiores valores gerados pela soma dos itens de cada sub-escala \cite{lovibond1995structure, ribeiro2004contribuiccao}.

Diferente da proposta de auto avaliação ou avaliação mediada\footnote{Nesses casos os usuários são responsáveis por responder as perguntas por si mesmos ou por meio de um mediador que ira preencher o formulário.}, como é proposta pela EADS,  um dos objetivos da pesquisa é inferir os resultados das escalas utilizando textos cotidianos de uma amostra coletada no Twitter. Isso nos leva a entender os conceitos cognitivos por trás da psicologia que levaram os autores a propor suas escalas e pesquisas.
