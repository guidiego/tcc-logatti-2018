\subsection{Escala Depressão, Ansiedade e Stress}
Adotando um modelo dividido em 3 sub-escalas a \textbf{Escala de Depressão, Ansiedade e Stress} ou \textbf{EADS} foi proposta na premissa de uma maior assertividade na analise das dimensões afetivas negativas, uma vez que existiam outras pesquisas com proposta similar como o \textit{ Beck Anxiety Inventory} (BAI) e \textit{Beck Depression Inventory} (BDI). As diferenças entre essas escalas são: sua execução, o fator de correlação das sub-escalas propostas pelo EADS e o inventário de Stress introduzido durante o estudo (como mencionado na sessão anterior).

A EADS completa é composta de 42 itens, por fins de facilitar a geração de dados iremos usar o modelo de 21 itens, que são divididos igualmente entre as escalas. Mesmo que um dos itens pertença a uma escala ele pode ter correlação com alguma outra. Esses itens são afirmações que podem ser respondidas por números de 1 a 4 e representam desde "não se aplica a mim" até "se aplicou a mim na maior parte das vezes". Representam as dimensões mais negativas os maiores valores gerados pela soma dos itens de cada subcategoria \cite{lovibond1995structure, ribeiro2004contribuiccao}.

Diferente da proposta de auto avaliação ou avaliação mediada\footnote{Nesses casos os usuários são responsáveis por responder as perguntas por si mesmos ou por meio de um mediador que ira preencher o formulário.}, como é proposta pela EADS,  uma das premissas da pesquisa é inferir os resultados das escalas utilizando textos cotidianos de uma amostra em rede social. Isso nos leva a entender os conceitos cognitivos por trás da psicologia que levaram os autores a propor suas escalas e pesquisas.


