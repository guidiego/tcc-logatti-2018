\subsection{Analise de Discurso}
Ao decorrer de um  texto (que é algo concreto), pode-se caracterizar diversos níveis de geração de sentido. O \textbf{fundamental}, responsável pela primeira formulação de sentido a partir do discernimento de termos dentro de um contexto. O \textbf{narrativo}, onde o autor utiliza dos valores fundamentais através de um sujeito tomando a direção da prosa. E por fim, o \textbf{discursivo}, relacionado as escolhas de tempo, espaço, pessoa e figura durante a narrativa dos fundamentos, dando a essa narrativa um ponto de vista. Logo, o termo \textbf{discurso} é dado como um suporte abstrato por trás do texto, afim da concretização da sua ideia central \cite[13-17]{gregolin1995ad}.

A \textbf{analise de discurso} é, de forma sucinta, uma analise do que foi dito, de como foi dito e qual o sentido do que foi dito. As primeiras manifestações do assunto foram no século XX com autores russos que, além de isolar e definir elementos de uma linguagem poética queriam definir determinantes por trás do perfil artístico do escritor. A evolução dessa área de estudo a quebrou em varias vertentes como a francesa, que apoia a possibilidade de automatizar essa analise por meio da informática, a AD continua sendo um campo complexo e de continuo estudo por trás das definições e metodologias para abordar e sustentar as novas unidades de analise. \cite[22]{souza2006ad}.

Os discursos se diferem de pessoa para pessoa devido ao nível discursivo, a necessidade de expressar um determinado sentido leva o autor a se colocar em um ponto de vista durante sua narrativa. Do contexto da pesquisa, entender o discurso do usuário para mapear o motivo do seu estado mental é um fator de total relevância para entender o estado dele. A pesquisa realizada por Modesto Leite \cite[134]{modesto2005adepre}, mostra em seus resultados que os discursos apresentados pelos pacientes fundamentavam o motivo psicológico do por que os mesmo teriam o transtorno. Sem uma analise coerente dos discursos de nossa amostra ainda se faz possível a geração de dados relevantes, porem, as chances de não avançarmos em assertividade se torna grande.
