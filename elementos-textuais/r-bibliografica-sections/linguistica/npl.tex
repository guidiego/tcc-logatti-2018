\subsection{Processamento de Linguagem Natural}
Partindo desse principio o que impediria o computador de entender o que lhe foi dito? Mesmo parecendo óbvio, entender palavras não significa entender o contexto, é necessário se familiarizar com o ambiente e o momento afim de idealizar o que está sendo transmitido. Essa conexão entre elementos é tratada no estudo do \textit{connectivism}\footnote{Integração dos princípios de rede, caos, complexidade e teorias de auto-organização. Seu objetivo é entender decisões baseado nas mudanças de componentes fundamentais \cite[5]{siemens2014connectivism}.}. De acordo com a linha de pensamento, estabelecida pelo estudo, os neurônios seriam os agentes cognitivos responsáveis por planejar, construir e representar essas informações que nosso cérebro recebe \cite[22]{brandura1996}. Mesmo com toda essa relevância o assunto não tinha tanta visibilidade até o nascimento  do campo hibrido chamado de Processamento de Linguagem Natural. \cite[16]{russell2003artificial}