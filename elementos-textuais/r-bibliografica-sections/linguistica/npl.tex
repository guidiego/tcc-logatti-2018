\subsection{Processamento de Linguagem Natural}
Partindo do principio de um dicurso, por mais que as palavras sejam localizadas, como inferir o sentido da frase? Mesmo parecendo óbvio, entender palavras não significa entender o contexto, é necessário se familiarizar com o ambiente e o momento afim de idealizar o que está sendo transmitido. Essa conexão entre elementos é tratada no estudo do \textit{connectivism}\footnote{Integração dos princípios de rede, caos, complexidade e teorias de auto-organização. Seu objetivo é entender decisões baseado nas mudanças de componentes fundamentais \cite{siemens2014connectivism}.}. De acordo com a linha de pensamento, estabelecida pelo estudo, os neurônios seriam os agentes cognitivos responsáveis por planejar, construir e representar essas informações que nosso cérebro recebe. Criar soluções para para problemas pontuais que envolvam a ligua que é utilizado no dia-a-dia se uma pessoa, essa é a definição por trás do \textbf{Processamento de Linguagem Natural} \cite{brandura1996, maria2015npl}.

\begin{quote}
Além disso, esse é o ponto principal da contribuição da Linguística para o PLN, qual seja, o de fornecer dados linguísticos que a máquina não é capaz de inferir, mas pode, em parte, processar, melhorando o seu desempenho.
\end{quote}
