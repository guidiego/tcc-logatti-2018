\section{Fundamentos da Inteligencia Artificial}

Psicologia, Linguística, Economia, Filosofia e Matemática, esses são alguns dos fundamentos abordados até então. É possível entender o conceito por trás da inteligencia, porem, para que a IA seja bem sucedida e necessário um artefato, e esse artefato tem sido o computador. A engenharia da computação é um outro fundamento, que até agora foi tratado implicitamente, seu foco é construir maquinas eficientes, sendo assim, além dos hardwares, novos sistemas operacionais, linguagens e ferramentas surgiram fortalecendo o campo de IA \cite[13-14]{russell2003artificial}.


Nas últimas sessões foi abordados os temas \textbf{linguistica} e \textbf{psicologia}, e claramente é possivel vislumbrar sua relevancia dentro da área de Inteligencia Artifical. O motivo dessa relevancia é que ambas são consideradas fundamentos da IA assim como outras áreas que serão abordadas aqui.

% Filosofia e intro a agentes
\textit{Law of thought} (lei do pensamento) é uma lei psicológica que acompanha um processo mental, logo, necessita estar de acordo com uma razão ou lógica \cite[289-291]{frege1956thought}. Para as pessoas a lógica parece ser algo extremamente banal, porem, a teoria por trás do que é logico e o que é racional é algo bem extenso. Desde a Grécia antiga filósofos e/ou matemáticos estudam o conceito de lógica. \cite[4-5]{russell2003artificial}. Durante todo esse período varias escolas nos proporcionaram diversas visões sobre o assunto. O racionalismo por exemplo, visa que podemos adquirir conhecimento independente da nossa experiencia sensorial \cite{rationalismvsempiricism}. Dentro dessa ideia nasceram o dualismo, onde afirmamos que nossa mente é algo natural e sem conhecimento do mundo externo \cite[7]{descartes2013rene}, e o materialismo, que contrariando o dualismo afirma que a mente é formada pelas operações do nosso cérebro \cite[6]{russell2003artificial}.
Em contradição com o racionalismo, existe o empirismo, onde a experiencia sensorial é a fonte final de conhecimento \cite{rationalismvsempiricism}.

% Economia
Para que se torne possível realizar ações racionais é necessário entender o ambiente aonde você esta e suas variáveis \cite[99]{simon1955behavioral}. Diferente do que muitos pensam o estudo da economia se trata de dinheiro porem se trata de como guiamos nossas decisões baseadas nos retornos esperados. Os estudos da economia dentro da IA ainda se aplicam a questões como como devemos agir quando o que esperamos esta em um futuro distante ou então se devemos continuar quando outros fatores não continuarem a nos favorecer \cite[9]{russell2003artificial}. A ação racional descrita é feita por um agente, que é descrito como autônomo e racional, uma vez que não necessita diretamente de um humano para agir e é construído com propósito de retirar a melhor performance com base em seu objetivo \cite[2]{ wooldridge1994agent}.

% Matematica
Esse agente esta dentro de um ambiente, é necessário que esteja claro o que pode ou não ser computado, quais são as regras que podemos aplicar e principalmente como conseguimos obter algo racional em caso informações incertas \cite[7]{russell2003artificial}. Os princípios matemáticos aplicados aqui partem das pequisas em cima das proposições, para que seja possível deduzir uma proposição não é necessário apenas inferência, em alguns casos, é necessário que esses dados sejam convertidos a fatores numéricos, afim de utilizar funções aritméticas para resolver os problemas \cite[2-4]{boole1854investigation}. Além disso, existe a probabilidade, em pouco tempo sua capacidade de resolver teoremas inacabados e ajudar a mensurar problemas a tornou indispensável para a IA \cite[9]{russell2003artificial}.

A cibernética e a teoria de controle é um outro fundamento da IA, o conceito por trás desse fundamento foi criado por em 1961 por Wiener \cite[15]{russell2003artificial}. Muitas vezes é necessário que a maquina tenha conhecimento de fatores externos, que usualmente não pode ser transmitido pela ausência de transmissores de informação humanos, como olhos, ouvidos e boca. Se esses estímulos são tão relevantes seria necessário que de forma elétrica fossem criados sensores capazes de controlar essas informações e integrar com os softwares de maneira que passassem dados numéricos e não exatamente seus sentimentos em relação a algo \cite[3-7]{wiener1961cybernetics}. Diferente dos demais, esse fundamento não impacta nossa pesquisa, logo, não será aprofundado como os demais.

A neurociência, é responsável por estudar o nosso cérebro, ou sendo mais exato, como nossas redes neurais funcionam, e é o ultimo fundamento abordado. \cite[10]{russell2003artificial}. A biologia em si era em grande parte descritiva, propor um modelo matemático para o funcionamento de um cérebro vislumbraria em entender o metabolismo como centro de transmissão e calcular a forças emitidas durante esse processo, para que então, fosse transformar a ação ocorrida em um cérebro humano em uma função aritmética \cite[1-3]{rashevsky1960mathematical}. Criar entidades mais inteligentes necessitária de avanços em campos externos ao de computação, como a biologia, para entender se é possível maximizar a inteligencia humana, outras definições como até onde entidades ficariam mais inteligentes, como implementar interfaces para suportar algo talvez superior a inteligencia humana ou então aumentar a capacidade de nossos computadores para que eles fossem capazes de suprir o potencial de um cérebro dependeriam em grande parte de avanços de hardware \cite[1-2]{vinge1993coming}. Mesmo com tantas discussões em torno do assunto, a proposta de singularidade de Vinge não tem uma comparação informativa, e mesmo que fossemos capaz de ter memória e processamento infinito ainda não é possível entender como armazenar e replicar os padrões encontrados na neurociência \cite[11-12]{russell2003artificial}. Os estudos cognitivos tornaram possível entender melhor o funcionamento da mente humana; o método racional e a proposta de pensar nos meios que nos levam a um fim fez com que fosse capaz de abstrair modelos inteligentes através de agentes; a neurociência, por sua vez, nos deu a magnitude de como transmitir conhecimento e aprimorar os modelos, o fator inteligencia e aprendizado passaram a ser mais vistos dentro do ramo de IA, e nos levaram para ascensão da aprendizagem de máquina.