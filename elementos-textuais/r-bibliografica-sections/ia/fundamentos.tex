\subsection{Fundamentos da Inteligencia Artificial}
Nas últimas sessões foram abordados os temas linguística e psicologia, e claramente é possível vislumbrar sua relevância dentro da área de Inteligência Artificial. A razão é o fato de ambas serem fundamentos da IA assim como outras áreas que serão dissertadas nessa sessão.

A engenharia da computação tem como foco construir maquinas eficientes, seja impactando com novos \textit{hardwares}, afetando diretamente o potencial de processamento e armazenamento das máquinas, ou com a criação de novos sistemas operacionais, linguagens e ferramentas que podem desde auxiliar na \textit{performance}, organização ou complexidade do problema até possibilitar criação de novas abordagens e implementações de algoritmos mais complexos. \cite[13-14]{russell2003artificial}.

Na busca de um melhor resultado durante a criação de um sistema inteligente, o processamento de dados é algo essencial. Em certos momentos os dados não forneceram diretamente o que é necessário, e o ato de assimilar uma verdade baseado em outra verdade já conhecida, também chamado de inferência, já não será suficiente para gerar bons resultados, sendo assim, é necessário utilizar de recursos fundamentados e probabilísticos como equações matemáticas para resolver o problema. Para que isso se torne possível, será necessários extrair um dado exato de algo abstrato, por exemplo o \textit{sentistrength}\footnote{http://sentistrength.wlv.ac.uk/}, que retira o sentimento a partir de um trecho de texto \cite{boole1854investigation}.

Além dos conceitos de matemática, existem fatores da economia que fundamentam a IA. Diferente do que sugerido, essa área não se trata de dinheiro, mas de como é guiada as decisões baseadas nos retornos esperados. Os estudos da economia ainda aplicam-se a como agir perante expectativas de curto, médio ou longo prazo e se é viável continua-las quando outros fatores não estiverem favorecendo o ambiente \cite[9]{russell2003artificial}. Abordando o ambiente não só como o \textit{software} e sim como um mundo externo a ele, existe um fundamento, que não será aprofundado por não ter impacto na nossa pesquisa, nomeado cibernética e a teoria de controle. Foi criado por Wiener\cite{wiener1961cybernetics} e visa o uso de componentes elétricos para coletar informações externas á maquina e traduzi-los para uma linguagem (numérica) cuja a ela seja capaz de compreender.

Existem campos, como a própria linguística citada, que procuram sair da forma descritiva que atuam, propondo-se adotar modelos e definições exatas para guiar seus estudos. Esse foi o caso da neurociência, que é responsável por estudar o nosso cérebro, ou sendo mais exato, como nossas redes neurais funcionam \cite[10]{russell2003artificial}. A biologia em si era em grande parte descritiva, embasada por anos de observação e pesquisas que até hoje apoiam suas definições. Sair desses padrões levou os cientistas a vislumbrarem a possibilidade de tratar nosso metabolismo como um centro de transmissão. Esse centro enviaria diversas transmissões que por sua vez emitiriam uma força. Essa afirmação levaria os cientistas a poderem calcular essas a força dessas transmissões afim de chegar em um modelo matemático plausível para replicar o funcionamento de nossas redes neurais \cite[1-3]{rashevsky1960mathematical}. Entretanto, duvidas como até onde pode-se um computador suportar ou superar o processamento de um cérebro humano foram levantadas. Mesmo com tantas discussões em torno do assunto, a proposta de Vinge sobre uma super-máquina que superaria a inteligência humana, também chamada de singularidade tecnológica, continua não tendo uma comparação informativa atualmente, e mesmo que atualmente existisse a capacidade de ter-se memória e processamento infinito ainda não é possível entender como armazenar e replicar os padrões encontrados na neurociência \cite[11-12]{vinge1993coming,russell2003artificial}. Os estudos cognitivos tornaram possível entender melhor o funcionamento da mente humana. O método racional e a proposta de pensar nos meios que nos levam a um fim fez com que fosse capaz de abstrair modelos inteligentes através de agentes. A neurociência, por sua vez, nos deu a magnitude de como transmitir conhecimento e aprimorar os modelos, o fator inteligência e aprendizado passaram a ser mais vistos dentro do ramo de IA, e nos levaram para ascensão da aprendizagem de máquina.

Diferente das ciências onde temos fatos, teorias e pesquisas conclusivas, existem áreas capazes de fomentar intelectualmente assuntos diversos. O conceito de lógica, por exemplo, é algo relativamente descritível e aplicável aos conceitos da ciência, porem a teoria por trás do que viria ser lógico e o que faria um pensamento ser racional é algo muito mais complexo, em suma, o conceito de lógica já era debatido por estudiosos na Grécia antiga. A filosofia tem sua relevância pra IA pois contribuiu com ideias como o \textit{Law of thought} (lei do pensamento), que em contra partida ao pensar humano, é uma lei psicológica que acompanha um processo mental, logo, necessita estar de acordo com uma razão ou lógica \cite{frege1956thought, russell2003artificial}. Além disso, a filosofia foi responsável por estudos como o racionalismo onde é dito que podemos adquirir conhecimento independente da nossa experiência sensorial. Originada dessa ideia nasceram o dualismo, onde é afirmado que a mente é algo natural e sem conhecimento do mundo externo, e o materialismo, que contrariando o dualismo afirma que a mente é formada pelas operações do cérebro. Em negativa ao racionalismo, existe o empirismo, onde é definido que a experiência sensorial é a fonte final de conhecimento \cite[6]{rationalismvsempiricism, descartes2013rene, russell2003artificial}. Estes estudos tiveram sua relevância sobre o pensamento racional e como seria possível criar maquinas que pensassem e agissem racionalmente.
