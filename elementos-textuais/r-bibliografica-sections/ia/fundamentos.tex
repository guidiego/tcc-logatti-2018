\section{Fundamentos da Inteligencia Artificial}
Nas últimas sessões foram abordados os temas \textbf{linguística} e \textbf{psicologia}, e claramente é possível vislumbrar sua relevância dentro da área de Inteligência Artificial. O motivo disso vem de ambas serem consideradas \textbf{fundamentos} da IA assim como outras áreas que serão abordadas nessa sessão.

A \textbf{engenharia da computação} tem como foco construir maquinas eficientes além dos hardwares, sendo assim, novos sistemas operacionais, linguagens e ferramentas que fortalecem o campo de IA \cite[13-14]{russell2003artificial}.

Durante a criação de uma inteligência artificial existem dois fatores muito importantes: a relação de probabilidades juntamente a transformação de dados abstratos para fatores numéricos, seguido de, como será processado os dados afim de gerar um melhor resultado. O primeiro fator esta ligado a conceitos da \textbf{matemática}, a inferência não será capaz de resolver todos os problemas, as vezes utilizar funções matematicas para melhores resultados é necessário \cite{boole1854investigation}. O segundo fator vem da \textbf{economia} que diferente do que sugerido, não trata-se de dinheiro, mas de como é guiada as decisões baseadas nos retornos esperados. Os estudos da economia dentro da IA ainda se aplicam a como agir perante expectativas de curto, médio ou longo prazo e se é viável continuar algo quando outros fatores não continuarem a favorecer o ambiente \cite[9]{russell2003artificial}.

A \textbf{cibernética e a teoria de controle} é um outro fundamento da IA, o conceito por trás desse fundamento foi criado por em 1961 por Wiener \cite[15]{russell2003artificial}. Muitas vezes é necessário que a maquina tenha conhecimento de fatores externos, que usualmente não pode ser transmitido pela ausência de transmissores de informação humanos, como olhos, ouvidos e boca. Se esses estímulos são tão relevantes seria necessário que de forma elétrica fossem criados sensores capazes de controlar essas informações e integrar com os softwares de maneira que passassem dados numéricos e não exatamente seus sentimentos em relação a algo \cite[3-7]{wiener1961cybernetics}. Diferente dos demais, esse fundamento não impacta nossa pesquisa, logo, não será aprofundado como os demais.

A \textbf{neurociência}, é responsável por estudar o nosso cérebro, ou sendo mais exato, como nossas redes neurais funcionam, e é o ultimo fundamento abordado. \cite[10]{russell2003artificial}. A biologia em si era em grande parte descritiva, propor um modelo matemático para o funcionamento de um cérebro vislumbraria em entender o metabolismo como centro de transmissão e calcular a forças emitidas durante esse processo, para que então, fosse transformar a ação ocorrida em um cérebro humano em uma função aritmética \cite[1-3]{rashevsky1960mathematical}. Criar entidades mais inteligentes necessitária de avanços em campos externos ao de computação, como a biologia, para entender se é possível maximizar a inteligência humana, outras definições como até onde entidades ficariam mais inteligentes, como implementar interfaces para suportar algo talvez superior a inteligência humana ou então aumentar a capacidade de nossos computadores para que eles fossem capazes de suprir o potencial de um cérebro dependeriam em grande parte de avanços de hardware \cite[1-2]{vinge1993coming}. Mesmo com tantas discussões em torno do assunto, a proposta de singularidade de Vinge não tem uma comparação informativa, e mesmo que fossemos capaz de ter memória e processamento infinito ainda não é possível entender como armazenar e replicar os padrões encontrados na neurociência \cite[11-12]{russell2003artificial}. Os estudos cognitivos tornaram possível entender melhor o funcionamento da mente humana; o método racional e a proposta de pensar nos meios que nos levam a um fim fez com que fosse capaz de abstrair modelos inteligentes através de agentes; a neurociência, por sua vez, nos deu a magnitude de como transmitir conhecimento e aprimorar os modelos, o fator inteligência e aprendizado passaram a ser mais vistos dentro do ramo de IA, e nos levaram para ascensão da aprendizagem de máquina.

Diferente das ciências onde temos fatos, teorias e pesquisas conclusivas, existem áreas capazes de fomentar intelectualmente assuntos diversos, o conceito de lógica, por exemplo, é algo relativamente descritível e aplicável aos conceitos da ciência, porem a teoria por trás do que viria ser lógico e o que faria um pensamento ser racional é algo muito mais complexo. Desde a Grécia antiga filósofos estudam o conceito de lógica \cite[4-5]{russell2003artificial}. A \textbf{filosofia} tem sua relevância pra IA pois contribuiu com ideias como o \textit{Law of thought} (lei do pensamento), que em contra partida ao pensar humano, é uma lei psicológica que acompanha um processo mental, logo, necessita estar de acordo com uma razão ou lógica \cite[289-291]{frege1956thought}. Alem disso, a filosofia foi responsavel por estudos como o racionalismo onde é dito que podemos adquirir conhecimento independente da nossa experiencia sensorial. Dentro dessa ideia nasceram o dualismo, onde afirmamos que nossa mente é algo natural e sem conhecimento do mundo externo, e o materialismo, que contrariando o dualismo afirma que a mente é formada pelas operações do nosso cérebro. Em negativa ao racionalismo, existe o empirismo, onde a experiencia sensorial é a fonte final de conhecimento \cite[6]{rationalismvsempiricism, descartes2013rene, russell2003artificial}. Todos esses estudos tiveram sua relevância sobre o pensamento racional e como seria possível criar maquinas que pensassem e agissem racionalmente.
