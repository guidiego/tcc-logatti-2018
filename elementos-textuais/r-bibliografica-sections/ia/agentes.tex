\subsection{Agentes Racionais}
%Tranformar em intro
Mas seria possível uma maquina pensar como nós e entender as mesmas descrições? Esse foi o questionamento inicial de Turing, a partir disso ele propôs o \textit{Imitation Game},que anos após se tornaria o famoso Teste de Turing. Além disso, algumas diretrizes para construção da maquina capaz de passar em seu teste e objeções a suas próprias afirmações foram dadas. Duas dessas objeções seriam o tamanho finito do armazenamento das maquinas seguido da explicação que apenas reconhecer frases, buscar a informação necessária e levá-la ao usuário não seria o suficiente para passar no teste, as maquinas teriam que ser capazes de guardar instruções e se aprimorar do mesmo modo que um ser humano, assim, seria possível conseguir se adaptar a novas situações \cite[144-155]{turing1950}. Logo, questionar se uma maquina pode pensar é o mesmo que questionar se um submarino pode nadar, essa frase foi utilizada por Dijsktra \cite{dijkstra898} ao explicar que um submarino nunca realizaria o ato de nadar, porem, continuaria executando seu objetivo e propósito da melhor maneria possível.

Quando foi proposto à máquina entender um conjunto de texto e responder corretamente a isso \cite[146]{turing1950}, entramos indiretamente no campo da linguística.

Logo para passar no Teste de Turing, seria necessário criarmos uma maquina capaz de aprender e se adaptar a diversas situações usando Processamento de Linguagem Natural e Referencias de Conhecimento para entender o que foi requisitado além de um algorítimo capaz de automaticamente tomar decisões baseados nos dados fornecidos. \cite[2]{russell2003artificial}. Todas essas etapas podem ser/ou não solucionadas com agentes, que serão explicados a seguir. Durante essa mesma época, em 1956, foi proposto que a quantidade de informações a serem referenciadas seria uma das maiores dificuldades em entender as relações entre as variantes \cite[81-82]{miller1956magical}, de acordo com Russel \cite[13]{russell2003artificial} o inicio do campo da Ciência Cognitiva ocorreu no \textit{workshop} onde esse artigo foi publicado. Essa parceria entre as áreas de computação e psicologia trouxeram grandes avanços ao juntar modelos computacionais com teorias experimentais de psicologia, essa abordagem causou divergências entre estudiosos que discutiam se esse modelo ou um algorítimo performariam melhor na execução de uma tarefa. \cite[3]{russell2003artificial}. O campo da Ciência Cognitiva,  que tinha vasta abrangência como é possível vislumbrar em \textit{“The MIT encyclopedia of the cognitive science”} \cite{wilson2001encyclopedia} que reuniu seis diferentes áreas para abordar o tema, ganhou tração.

%-------[]

O racionalismo por exemplo, visa que podemos adquirir conhecimento independente da nossa experiencia sensorial \cite{rationalismvsempiricism}. Dentro dessa ideia nasceram o dualismo, onde afirmamos que nossa mente é algo natural e sem conhecimento do mundo externo \cite[7]{descartes2013rene}, e o materialismo, que contrariando o dualismo afirma que a mente é formada pelas operações do nosso cérebro \cite[6]{russell2003artificial}.
Em contradição com o racionalismo, existe o empirismo, onde a experiencia sensorial é a fonte final de conhecimento \cite{rationalismvsempiricism}.

Para que se torne possível realizar ações racionais é necessário entender o ambiente aonde você esta e suas variáveis \cite[99]{simon1955behavioral}.

A ação racional descrita é feita por um agente, que é descrito como autônomo e racional, uma vez que não necessita diretamente de um humano para agir e é construído com propósito de retirar a melhor performance com base em seu objetivo \cite[2]{wooldridge1994agent}.

Esse agente esta dentro de um ambiente, é necessário que esteja claro o que pode ou não ser computado, quais são as regras que podemos aplicar e principalmente como conseguimos obter algo racional em caso informações incertas \cite[7]{russell2003artificial}. 

Em resumo o agente recebera um \textit{input}\footnote{Entrada de dados} e será responsável por gerar um \textit{output}\footnote{Dado de Saída}, ao longo do tempo o mesmo agente gerara múltiplas percepções e essas formarão uma sequencia de percepções\footnote{Não serão todos os modelos que seguirão a proposta sequencial, existem casos em que a linha temporal não afeta o desenvolvimento da decisões tornando-as episódicas}. \cite[34-35]{russell2003artificial} Existem definições dadas aos agentes racionais, iremos definir algumas delas a seguir \cite[42-45]{russell2003artificial}:

\begin{itemize}
 \item \textbf{Totalmente, parcialmente ou não observador:} essa definição é gerada pela quantidade de fatores do ambiente que seu agente recebe, um agente que tem todas as informações do ambiente é totalmente observador enquanto um que não recebe nada, precisando assim manter alguns estados, é não observador.
 \item \textbf{Estocástico ou Determinístico:} quando é impossível determinar o próximo estado através do anterior o agente é Estocástico, caso ao contrario ele é Determinístico.
 \item \textbf{Episódicos ou Sequenciais:} já foi dito que em diversas abordagens são gerados sequencias de percepção, quando essa sequencia é alterado a partir de alguma mudança de estado chamamos o agente de sequencial, caso ao contrario o agente é Episódico.
 \item \textbf{Estáticos ou Dinâmicos} essa definição é referente ao ambiente, quando nosso ambiente não infere alterações chamamos o agente de estático, caso ao contrario Dinâmico.
 \item \textbf{Continuo ou Distinto:} Quando existem finitas possibilidades de estado pode se afirmar que o agente é Distinto, quando as possibilidades são infinitas é dado o nome de Continuo.
 \item \textbf{Conhecido ou Desconhecido:} Quando o agente necessita aprender algo e não consegue realizar a ação por si só ele é um agente desconhecido, caso ao contrário ele é conhecido.
\end{itemize}

Para que seja possível definir se o agente está ou não gerando os dados esperados é necessário medir sua performance, então, é necessário que analisar o ambiente gerado a partir das percepções e conferir se os dados são os esperados ou não \cite[294-295]{frege1956thought}.

Racional é algo baseado ou acordado com uma razão ou lógica\footnote{Oxford Dictionarie:  https://en.oxforddictionaries.com/definition/rational}, existem dois pilares para a lógica: a Conversão responsável por expressar a mesma proposição em diferentes formas e o Silogismo responsável por localizar um termo em comum que conecte duas dessas proposições. \cite[175]{boole1854investigation}.