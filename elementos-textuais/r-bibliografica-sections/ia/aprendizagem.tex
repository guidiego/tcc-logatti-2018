\section{Aprendizado de Maquina}
Se maquinas pensam ou não, o enfoque da IA é executar diretamente um processo que um ser humano aprendeu ao longo dos anos, e não, algo que implicitamente nasceu com ele. Logo, o fluxo mais sistemático para conseguir que uma maquina abstrai-se formalmente um conhecimento seria mapear um estado inicial, aplicar dados para sua aprendizagem e submete-la a outras experiências afim de reafirmar esses conhecimentos. Essa abordagem foi proposta por Turing em 1950, entretanto, o campo de pesquisa conhecido como \textit\textbf{machine learning}} ou \textbf{aprendizado de maquina} surgiu apenas em 1959 quando o termo nasceu ao ser aplicado em um estudo de redes neurais. Portanto, pode-se brevemente descrever o aprendizado de máquina como métodos computacionais designados a utilizarem de algoritmos precisos de predição juntamente a uma base de conhecimento para aumentar a sua assertividade\cite[1]{turing1950, samuel1959some, mohri2012foundations}.
