\subsection{Algoritmos de Aprendizado de Máquina}
Os atributos presente nos dados podem ser utilizados com a premissa de predizer uma determinada hipótese. O ato de coletar um conjunto de dados para treino baseados em entra-saida e, posteriormente, utiliza-lo para predizer uma ou mais hipóteses é nomeado \textbf{aprendizagem supervisionada} e pode-se entender melhor o contexto apresentado na <figura 3>. O mais comum cenário dentro da aprendizagem supervisionada consiste em utilizar de um atributo discreto\footnote{todo} ou qualitativo\footnote{todo}, a partir de um indutor, para gerar o melhor classificador para aquele problema, esse método é chamado \textbf{classificação}. Ainda existe um outro método muito comum tambem, que se baseia em atributos continuos\footnote{todo}, com isso é possivel utilizar de predição numérica afim de identificar um modelo dentro do plano cartesiano e prever futuros valores para o mesmo atributo em outras situações, esse métdo se chama \textbf{regressão}. \cite{hastie2009unsupervised, russell2003artificial}

Em contra partida, existem casos onde os atributos não estaram anotados e a IA terá que literalmente inferir a probabilidade numérica em toda a base. Esse estilo de aprendizagem se chama \textbf{aprendizagem não supervisionada}, e o enfoque é utilizar da segregação dos atributos e suas dimensões para inferir resultados. O método mais comum dentro desse estilo é a \textbf{clusterização}, em resumo, ela se baseia em particionar os dados em grupos dentro de um plano cartesiano a partir do dado de inferencia. Outra pratica utilizada nesse tipo de aprendizagem é \textbf{redução dimensional}, ela se baseia em resumir o estado atual de um item para um estado de menor complexidade de dimensões com base nas propriedades chaves \cite{hastie2009unsupervised, mohri2012foundations}.

Ainda co-relacionado com as últimas explicações, existe um tipo onde é inserido no agente dados anotados e não anotados para pedrição de todos os items da base. A \textbf{aprendizagem semi-supervisionada}, sucintamente explicando, é a mescla das duas outras aprendizagens, visando o resultado, utiliza da parte anotada (supervisionada) e da lógica de grupos (não-supervisionada), para normalizar e optimizar o resultado final \cite[7]{mohri2012foundations}.

Por último, existem casos onde não teremos dados suficientes para executar outros tipos de aprendizagem, nesse momento a aprendizagem baseada no tradicional tentativa e error. Nomeada \textbf{aprendizado por reforço}, se observar a figura:3, pode-se notar que o funcionamento desse tipo de aprendizado se consiste em um \textbf{ambiente} (A), responsavel por emitir um estado para o \textbf{component} (C), feito isso o é aplicado uma \textbf{entrada de dados} (e) ao nosso \textbf{agente} (G) que será responsavel por tomar a decisão de que \textbf{ação} (a) tomara para a entrada recebida. Essa ação modifica o estado do ambiente e transmite um sinal de reforço visando através do componente (r) para que a aplicação tome a melhor escolha ao longo prazo \cite{kaelbling1996reinforcement, russell2003artificial}.


Não será definido exemplos práticos dessos algoritimos dados, durante nosso resultado e conclusão abordaremos a utilização de algumas das abordagens surgeridas dentro de nossa pesquisa e o motivo da escolha.

Indiferente do tipo ou situação em que o algoritimo se enquadra, a mutiplicidade de escolhas propostas pelo machine learning gerou várias propostas de aprendizado, algumas até tentando utilizar da estrutura proposta pela neurociencia para replicar nossoas redes neurais.




