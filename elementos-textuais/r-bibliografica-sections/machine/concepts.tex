\subsection{Conceitos do Aprendizado de Máquina}
Previamente foi dito que, seriam necessários dados de treino, esse dado bruto, como já dito também, contem várias propriedades que podem ser ou não processadas afim de gerar informações, esse dado é conhecido com dados de treinamento e um dado que contém as informações processadas necessárias, inserida por algoritimos ou por fatores externos, é chamado de dado rotulado ou anotado.

As propriedades podem ser utilizadas para classificar um determinado dado, isso as concede o nome de atributo. Também conhecido com \textit{feature} é considerada como um dos fatores mais relevantes para o aprendizado da maquina. O termo conhecido como \textit{feature engineering} ou, em tradução literal, engenharia de atributos, diferente do processo de aprendizagem em si, são baseados diretamente a suas entidades e em como escolher e pré-processar seus atributos chaves afim de otimizar o aprendizado \cite{domingos2012few}.

Esses atributos, dentre outras propriedades, estão em uma massa de dados. Por sua vez serão utilizadas como dado de entrada para um agente. Esse agente tambem pode ser representado, por fins explicativos, como uma simples função \textit{f(x) = y}, o objetivo é treinar o agente para que ele seja capaz de generalizar. Ou seja, capaz de predizer o melhor valor para \textit{y} a partir da entrada \textit{x}. Quando o resultado final depende totalmente do dado de entrada, ou seja, não existem muitas variações que levem a saída a grandes divergências, dizemos que o problema generaliza bem.

Essa função gerada que sucintamente traduz os padrões encontrados entre os dados, gera o que conhecemos como modelo. Muitos deles são utilizados para classificar o dado dentro de um grupo de possibilidades, esse tipo de modelo é descrito como classificador.

Os problemas a serem processados pelos classificadores ou outros modelos possuem um conjunto de  hipóteses. Quando é concebida uma preferencia a uma dessas hipóteses com a premissa de induzir o modelo a tomar uma decisão chamamos esse ato de tendência.

Uma quantidade extensa de hipóteses e dados de entrada não significam uma maquina mais assertiva, quando um agente é treinado para executar mais do que o necessário, um fenômeno peculiar chamado de \textit{overfitting} ou sobre-ajustes acontece, esse evento faz com que a maquina seja capaz de tomar decisões assertivas apenas para um conjunto de dados a qual foi submetida e não consegue predizer novas entradas.

As descrições dadas por Russel \cite[693]{russell2003artificial} ainda se aplicam a conceitos mais complexos que não cabem a essa pesquisa explicitamente. O jeito que é utilizado os dados ou os atributos previamente ditos, podem resultar em diversos tipos de abordagens diferentes.
