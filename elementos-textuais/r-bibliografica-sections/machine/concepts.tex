\subsection{Conceitos do Aprendizado de Máquina}
Préviamente foi dito que, seria necessário dados de treino, esse dado bruto, como ja dito tambem, contem várias propriedades que podem ser ou não processadas afim de gerar informações, esse dado é conhecido com \textbf{dados de treinamento}.

As propriedades citadas podem ser utilizadas para classificar o dado, isso as concede o nome de \textbf{atributo}. Tambem conhecido com \textbf{\textit{feature}} .....

Esses atributos, dentre outros dados, serão dados de entrada \textit{x} afim de \textbf{generalizar} os problemas apresentados esperando um valor de saida \textit{y}. Um problema generaliza bem, quando seu valor depende diretamente do valor de \textit{x} tornando condificionais excedentes desnecessária para obter o valor experado.

Essa função gerada que sucintamente traduz os padrões encontrados entre os dados, gera o que conhecemos como \textbf{modelo}. Muitos deles são utilizados para classificar o dado dentro de um grupo de possibilidades, esse tipo de modelo é descrito como \textbf{classificador}.

O problemas a serem processados pelos classificadores ou outros modelos possuem um conjunto de \textbf{hipoteses}. Quando é concebida uma preferencia a uma dessas hipoteses com a premissa de induzir o modelo a tomar uma decisão chamamos esse ato de \textbf{tendencia}.

Uma quantidade extensa de hipoteses e dados de entrada não significam uma maquina mais acertiva, quando um agente é treinado para executar mais do que o necessário, uma fenomeno peculiar chamado de \textbf{\textit{overfitting}} acontece, esse evento impossibilita a maquina seja capaz de tormar decisões acertivas e criei novos grupos discondizentes.

As discrições dadas por Russel \cite[693]{russell2003artificial} ainda se aplicão a conceitos mais complexos que não cabem a essa pesquisa explicitamente. Entendido conceitos importante como dados de treinamento e \textit{features}, pode-se passar a entender como a divisão e aplicação em torno de ambos itens geram diferentes abordagens.