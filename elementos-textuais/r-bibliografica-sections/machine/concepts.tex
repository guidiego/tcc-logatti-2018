\subsection{Conceitos do Aprendizado de Máquina}
Previamente foi dito que, seria necessário dados de treino, esse dado bruto, como já dito também, contem várias propriedades que podem ser ou não processadas afim de gerar informações, esse dado é conhecido com \textbf{dados de treinamento}.

As propriedades citadas podem ser utilizadas para classificar o dado, isso as concede o nome de \textbf{atributo}. Também conhecido com \textbf{\textit{feature}} é considerada como um dos fatores mais relevantes para o aprendizado da maquina. O termo conhecido como \textbf{\textit{feature engineering}} ou, em tradução literal, \textbf{engenharia de atributos}, diferente do processo de aprendizagem em si, são baseados diretamente a suas entidades e em como escolher e pré-processar seus atributos chaves afim de otimizar o aprendizado \cite{domingos2012few}.

Esses atributos, dentre outros dados, serão dados de entrada \textit{x} afim de \textbf{generalizar} os problemas apresentados esperando um valor de saída \textit{y}. Um problema generaliza bem, quando seu valor depende diretamente do valor de \textit{x} tornando condicionais excedentes desnecessária para obter o valor esperado.

Essa função gerada que sucintamente traduz os padrões encontrados entre os dados, gera o que conhecemos como \textbf{modelo}. Muitos deles são utilizados para classificar o dado dentro de um grupo de possibilidades, esse tipo de modelo é descrito como \textbf{classificador}.

O problemas a serem processados pelos classificadores ou outros modelos possuem um conjunto de \textbf{ hipóteses }. Quando é concebida uma preferencia a uma dessas hipóteses com a premissa de induzir o modelo a tomar uma decisão chamamos esse ato de \textbf{tendência}.

Uma quantidade extensa de hipóteses e dados de entrada não significam uma maquina mais assertiva, quando um agente é treinado para executar mais do que o necessário, uma fenômeno peculiar chamado de \textbf{\textit{overfitting}} acontece, esse evento impossibilita a maquina seja capaz de tomar decisões assertivas e crie novos grupos não coerentes.

As discrições dadas por Russel \cite[693]{russell2003artificial} ainda se aplicam a conceitos mais complexos que não cabem a essa pesquisa explicitamente. Entendido conceitos importante como dados de treinamento e \textit{features}, pode-se passar a entender como a divisão e aplicação em torno de ambos itens geram diferentes abordagens.

