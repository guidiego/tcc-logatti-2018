\subsection{Análise de Dados}
O passo descrito como interpretação no KDD se refere a entender os dados de saída vindos da mineração afim de afirmar ou descartar hipóteses, com isso é possível refinar o processo e/ou explorar novas possibilidades. Existem 3 fortes termos quando tratamos a analise de dados:
- \textit{Data Exploration}: O ato de explorar os dados (conhecimento) gerados afim de localizar pontos relevantes e elaborar ou embasar hipóteses.
- \textit{Data Storytelling}: O ato do dado contar uma história. Abordar a hipótese de maneira empírica e demonstrar a veracidade dela discutindo abordagem e algoritmos gerados é um dos objetivos dessa área.
- \textit{Data Visualization}: O ato de demonstrar o dado através de representações gráficas ou tabelas, normalmente utilizados para embasar e demonstrar dados durante o \textit{Storytelling}.
