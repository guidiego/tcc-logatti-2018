\subsection{Mineração de Dados}
Nessa subseção propõe-se que seja dissertado sobre os processos do KDD a partir da seleção até a interpretação. Vale ressaltar que para obter conhecimento não é necessário uma IA, sistemas de tomada de decisão trabalham com probabilidade matemática sob dados exatos, o que em muitos casos, já seria o suficiente para obter a informação do dado. Entretanto, o foco da pesquisa se baseia na implementação de um sistema inteligente e isso inclina essa explicação para o fato de: o aprendizado de máquina é uma das possíveis formas de se minerar um dado.

A mineração de dados ou popularmente conhecida como \textit{data mining}, baseia-se em retirar os valores mais relevantes e valiosos para se inferir um conhecimento. Os demais passos como pré-processamento e mutação se assemelham a alguns processos citados anteriormente como a redução dimensional ou ainda a engenharia de atributos.

