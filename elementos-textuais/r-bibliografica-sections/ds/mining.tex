\subsection{Mineração de Dados}
A mineração é um passo que ocorre após os dados terem sido devidamente normalizados. Vale ressaltar que para obter conhecimento não é necessário uma IA, sistemas de tomada de decisão trabalham com probabilidade matemática sob dados exatos, o que em muitos casos, já seria o suficiente para obter a informação do dado. Entretanto, o foco da pesquisa se baseia na implementação de um sistema inteligente e isso inclina essa explicação para o fato de: o aprendizado de máquina é uma das possíveis formas de se minerar um dado.

A mineração de dados ou popularmente conhecida como \textit{data mining}, baseia-se em retirar os valores mais relevantes e valiosos para se inferir um conhecimento. Os demais passos como pré-processamento e mutação se assemelham a alguns processos citados anteriormente como a redução dimensional ou ainda a engenharia de atributos.

No caso dessa pesquisa, a inferência de uma EADS baseada nas publicações de um usuário seria um tipo de dado extraído a partir de um modelo inteligente treinado por uma base de dados pré-processada. A caracterização de um perfil baseado em seus atributos, incluindo o próprio valor da EADS inferido, seria um outro exemplo do uso de IA para minerar a probabilidade de existir um certo nível de uma ou mais dimensão afetiva baseada em informações pessoais como sexo, localidade, trabalho e gostos pessoais.

Com os dados em mãos, é possível, baseado no conhecimento fornecido pelo dado, realizar vários tipos de análise. O enfoque desse trabalho não é realizar hipóteses ou explorar os dados gerados, e sim, certificar a possibilidade da criação dessa base e a assertividade dos modelos, garantindo, que outros pesquisadores possam utilizar da base gerada nesse trabalho para realizar futuras pesquisas. Mesmo assim se torna relevante citar, mesmo que brevemente, a última parte do ciclo de um dado referente a consumir o conhecimento fornecido pelo mesmo.
