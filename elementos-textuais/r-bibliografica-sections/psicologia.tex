\section{Psicologia}

A psicologia é, descrita como, a ciência da vida mental, capaz de analisando os desejos, sentimentos, razões, sentimentos, decisões entre outras faculdades mentais entender o posicionamento e o estado emocional do ser. Entender o nosso estado e como isso impacta em a vida é o grande desafio da área \cite[4-8]{william1890principles}.

Nossa pesquisa utiliza da psicologia em dois pontos distintos, porem, interligados. A primeira delas é o envolvimento da psicologia com a depressão, em segundo a participação da área da psicologia cognitiva na evolução da Inteligencia Artificial. Ambos os pontos se interligam ao se questionar o motivos de alguem ter depressão, ou o quão plausivel é o mapeamento disso através de técnicas desenvolvidas dentro da área nos ultimos anos.

% sections
\subsection{Depressão}
Desanimo, perca de interesse, inibição e bloqueio de sentimentos são alguns sintomas exibidos por pessoas melancólicas \cite[276]{freud}. A \textbf{melancolia} seria uma condição maléfica de enfraquecimento da sáude mental de um ser. Partindo do principio de Fairbain onde o ser humano busca por gratificação, a não gratificação poderia ser o motivo de um estado melancólico.

A \textbf{depressão} é uma forma atenuada de melancolia \cite{roudinesco2000}. Classificada como \textbf{transtorno de humor}, diferente de outras variações mais regulares de humor, pode causar grandes danos a vida cotidiana uma vez que, por definição, altera a percepção de si mesmo maximizando o peso dos seus problemas diante de sua própria pespectiva. Por tais motivos, a melancolia e a depressão compartilham de sintomas similares, entretando, a dinamica de suas origens, relações e concepções podem criar diversas perspectivas o que leva ao ponto de como se pode medir algo tão abstrado. \cite{}

\input{elementos-textuais/r-bibliografica-sections/psicologia/ansiedade}
\subsection{Escala Depressão, Ansiedade e Stress}
Adotando um modelo dividido em 3 sub-escalas a \textbf{Escala de Depressão, Ansiedade e Stress} ou \textbf{EADS} foi proposta na premissa de uma maior assertividade na analise das dimensões afetivas negativas, uma vez que existiam outras pesquisas com proposta similar como o \textit{ Beck Anxiety Inventory} (BAI) e \textit{Beck Depression Inventory} (BDI). As diferenças entre essas escalas são: sua execução, o fator de correlação das sub-escalas propostas pelo EADS e o inventário de Stress introduzido durante o estudo (como mencionado na sessão anterior).

A EADS completa é composta de 42 itens, por fins de facilitar a geração de dados iremos usar o modelo de 21 itens, que são divididos igualmente entre as escalas. Mesmo que um dos itens pertença a uma escala ele pode ter correlação com alguma outra. Esses itens são afirmações que podem ser respondidas por números de 1 a 4 e representam desde "não se aplica a mim" até "se aplicou a mim na maior parte das vezes". Representam as dimensões mais negativas os maiores valores gerados pela soma dos itens de cada subcategoria \cite{lovibond1995structure, ribeiro2004contribuiccao}.

Diferente da proposta de auto avaliação ou avaliação mediada\footnote{Nesses casos os usuários são responsáveis por responder as perguntas por si mesmos ou por meio de um mediador que ira preencher o formulário.}, como é proposta pela EADS,  uma das premissas da pesquisa é inferir os resultados das escalas utilizando textos cotidianos de uma amostra em rede social. Isso nos leva a entender os conceitos cognitivos por trás da psicologia que levaram os autores a propor suas escalas e pesquisas.



\subsection{Psicologia Cognitiva}
A psicologia é, descrita como, a ciência da saúde mental, capaz de analisar os desejos, sentimentos, razões, decisões dentre outras faculdades mentais afim de entender o posicionamento e estado emocional do ser \cite[4-6]{william1890principles}. Os estudos que visavam entender dos animais e humanos a capacidade de pensar, memorizar, perceber e no caso humano utilizar um dialeto (linguagem), foi uma área que surgiu antes mesmo da aparição das pesquisas sobre Inteligencia Artifical e sua vontade de entender os mesmos préceitos. Essa área da psicologia ficou conhecida como \textbf{Psicologia Cognitiva}.

O poder humano de realizar mutiplas tarefas, o questionamento de um cadeia de eventos físicos cancelar ou não uma outra e a recente popularidade e divergência da língua, fez com que em 1958 surgisse o primeiro modelo psicológico \cite[4-7]{broadbent1958perception}.

Os estudos da psicologia cognitiva gerou grandes avanços para área interdiciplinar das \textbf{ciencias cognitivas}. Novas abordagens como a junção de modelos criados em computação para IA e técnicas experiementais da psicologia foram criadas a fim do entendimento da mente humana. Durante a criação do \textit{General Problem Solver}, por exemplo, Newell e Simon propuseram não só a implementação de algoritimos descritos por antigos estudiosos afim de resolver um leque de problemas, mas tambem, entender como a maquina estava realizando isso, assim, seria possivel uma comparação com a mente humana na premissa de aprimorar conceitos ja existentes dentro da ciencia \cite[3-5]{newell1961gps, russell2003artificial}.

Nessa pesquisa, a psicologia cognitiva tera uma grande impacto em entender as \textit{features} da IA, uma vez que, sera necessário entender quais parametros serão analisados em pró de demonstrar a compatibilidade entre dois textos. Porem, alem disso, entender o conceito da psicologia nos ajuda a entender um dos vários fundamentos da IA que serão apresentados na próxima sessão.