\section{Psicologia}

A psicologia é, descrita como, a ciência da vida mental, capaz de analisando os desejos, sentimentos, razões, sentimentos, decisões entre outras faculdades mentais entender o posicionamento e o estado emocional do ser. Entender o nosso estado e como isso impacta em a vida é o grande desafio da área \cite[4-8]{william1890principles}.

Nossa pesquisa utiliza da psicologia em dois pontos distintos, porem, interligados. A primeira delas é o envolvimento da psicologia com a depressão, em segundo a participação da área da psicologia cognitiva na evolução da Inteligencia Artificial. Ambos os pontos se interligam ao se questionar o motivos de alguem ter depressão, ou o quão plausivel é o mapeamento disso através de técnicas desenvolvidas dentro da área nos ultimos anos.

% sections
\subsection{Dimensões Afetivas Negativas}
Se caracteriza \textbf{Dimensão Afetiva}, tambem conhecida como afetividade, um conjunto de sentimentos que nos afetão positiva ou negativamente \cite{pinto2009afetos}. Logo, quando abordado as \textbf{negativas}, pode-se pensar em tristeza, inveja e desesperança por exemplo, porem, o contexto da pesquisa sera limitado as dimensões afetivas negativas: depressão, ansiedade e stress.

A \textbf{depressão} é uma forma atenuada de \textbf{melancolia} \cite{roudinesco2000psicanalise}, que por sua vez, seria uma condição maléfica de enfraquecimento da sáude mental de um ser. Classificada como \textbf{transtorno de humor}, a depressão, diferente de outras variações mais regulares de humor, pode causar grandes danos a vida cotidiana uma vez que, por definição, altera a percepção de si mesmo maximizando o peso dos seus problemas diante de sua própria pespectiva \cite{esteves2006depressao}. A melancolia e a depressão compartilham de sintomas em comum como: Desanimo, perca de interesse, inibição, bloqueio de sentimentos e outros mais \cite[28]{freud2014livro}.

Segundamente existe a \textbf{Ansiedade}, muitas das disordem relacionadas a ansiedade são categorizadas pelo medo e evasão do usuario a um assunto especifico \cite[393]{dsmiv}. Esse sentimento é primitivo e tem fortes semelhanças as reações animais de defesa ao se colocarem em um abiente hostil. A inibição, assim como na depressão, é um dos sintomas dos transtornos relacionados a ansiedade, porem, o comportamento do ser assume um formato mais similar a um excesso de vigilancia e preocupação \cite{margis2003relaccao}.

\subsection{Escala Depressão, Ansiedade e Stress}
Adotando um modelo dividido em 3 sub-escalas a \textbf{Escala de Depressão, Ansiedade e Stress} ou \textbf{EADS} foi proposta na premissa de uma maior assertividade na analise das dimensões afetivas negativas, uma vez que existiam outras pesquisas com proposta similar como o \textit{ Beck Anxiety Inventory} (BAI) e \textit{Beck Depression Inventory} (BDI). As diferenças entre essas escalas são: sua execução, o fator de correlação das sub-escalas propostas pelo EADS e o inventário de Stress introduzido durante o estudo (como mencionado na sessão anterior).  

A EADS completa é composta de 42 itens, por fins de facilitar a geração de dados iremos usar o modelo de 21 itens, que são divididos igualmente entre as escalas e mesmo que um deles pertença a uma escala ele pode ter correlação com alguma outra. Esses itens são afirmações que podem ser respondidas por números de 1 a 4 que representam desde "não se aplica a mim" até "se aplicou a mim na maior parte das vezes" e no final serão somados a fim de gerar um resultado para a sub-escala, os maiores valores representam as dimensões mais negativas \cite{lovibond1995structure, ribeiro2004contribuiccao}.

Diferente da proposta de auto avaliação ou avaliação mediada\footnote{Nesses casos os usuários são responsáveis por responder as perguntas por si mesmos ou por meio de um mediador que ira preencher o formulário.}, como é proposta pela EADS,  uma das premissas da pesquisa é inferir os resultados das escalas utilizando textos cotidianos de uma amostra em rede social. Isso nos leva a entender os conceitos cognitivos por trás da psicologia que levaram os autores a propor suas escalas e pesquisas.



\subsection{Psicologia Cognitiva}
Dentro da Psicologia existe um ramo cujo o objetivo é entender a capacidade animal de pensar, memorizar, perceber e no caso humano utilizar um dialeto (linguagem), essa área de estudo ficou conhecida como psicologia cognitiva.

O ato de entender os procedimentos cognitivos adotados pelo ser humano se tornou mais relevante após a proposta da criação de uma inteligência artificial. O questionamento de como uma decisão poderia gerar, afetar ou cancelar uma cadeia de eventos, seja ela durante um pensamento ou uma ação, partindo da capacidade humana de realizar múltiplas tarefas, algumas delas complexas como interpretar linguagens com tantas divergências, fez com que em 1958\cite[4-7]{broadbent1958perception} surgisse o primeiro modelo psicológico que propunha um fluxo similar ao processamento de informações de um computador.

As técnicas experimentais, como os modelos psicológicos, impactaram toda a área interdisciplinar das ciencias cognitivas. Novas abordagens como a junção de modelos criados em computação para IA e técnicas experimentais da psicologia foram criadas a fim do entendimento da mente humana. Durante a criação do \textit{General Problem Solver}, por exemplo, Newell e Simon propuseram não só a implementação de algoritmos descritos por antigos estudiosos afim de resolver um série de problemas, mas também, entender como a maquina estava realizando isso, assim, seria possível uma comparação com a mente humana \cite{newell1961gps, russell2003artificial}.

Nessa pesquisa, a psicologia cognitiva terá um grande impacto em entender como as amostras que contem ou não sintomas causados pelas dimensões afetivas negativas se comportam e pensam. Esse entendimento é essencial para o mapeamento dos atributos que serão processados pela IA afim de gerar os modelos para predição em dados ainda não analisados. Além disso, a psicologia cognitiva faz parte de todo o processo qualitativo de criação devido a ser um dos vários campos que compões a multidisciplinaridade dentro da IA.
