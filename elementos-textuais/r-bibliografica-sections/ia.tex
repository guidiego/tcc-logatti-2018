\section{Inteligencia Artificial}

Inteligência é, por definição, uma coleção sistemática de habilidades e funções com objetivo de processar diferentes tipos de informações de diversas maneiras \cite[49]{guilford1982cognitive}. Simular essa "coleção sistemática de habilidades e funções", reconhecer formatos, trajetórias, sinais, deduzir proposições entre outras, criando entidades inteligentes é o foco da inteligência artificial.

Do momento em que a IA passou a ser um campo de estudo até o momento atual, a quantidade de abordagens apresentadas por diversos pesquisadores afim de construir uma máquina inteligente apenas aumentou. Esses métodos se auto-contribuíram propondo novas visões e gerando novas abordagens a partir de descobertas dentro das suas linhas de pensamento. Isso levou a IA para seu estado atual. \cite[1-2]{russell2003artificial}.

Ao decorrer da sessão serão apresentados os fundamentos visando introduzir as demais áreas que colaborarão com a computação para a criação da inteligência artificial, seguido pelas definições de agentes racionais.

\subsection{Fundamentos da Inteligencia Artificial}
Nas últimas sessões foram abordados os temas linguística e psicologia, e claramente é possível vislumbrar sua relevância dentro da área de Inteligência Artificial. A razão é o fato de ambas serem fundamentos da IA assim como outras áreas que serão dissertadas nessa sessão.

A engenharia da computação tem como foco construir maquinas eficientes, seja impactando com novos \textit{hardwares}, afetando diretamente o potencial de processamento e armazenamento das máquinas, ou com a criação de novos sistemas operacionais, linguagens e ferramentas que podem desde auxiliar na \textit{performance}, organização ou complexidade do problema até possibilitar criação de novas abordagens e implementações de algoritmos mais complexos. \cite[13-14]{russell2003artificial}.

Na busca de um melhor resultado durante a criação de um sistema inteligente, o processamento de dados é algo essencial. Em certos momentos os dados não forneceram diretamente o que é necessário, e o ato de assimilar uma verdade baseado em outra verdade já conhecida, também chamado de inferência, já não será suficiente para gerar bons resultados, sendo assim, é necessário utilizar de recursos fundamentados e probabilísticos como equações matemáticas para resolver o problema. Para que isso se torne possível, será necessários extrair um dado exato de algo abstrato, por exemplo o \textit{sentistrength}\footnote{http://sentistrength.wlv.ac.uk/}, que retira o sentimento a partir de um trecho de texto \cite{boole1854investigation}.

Além dos conceitos de matemática, existem fatores da economia que fundamentam a IA. Diferente do que sugerido, essa área não se trata de dinheiro, mas de como é guiada as decisões baseadas nos retornos esperados. Os estudos da economia ainda aplicam-se a como agir perante expectativas de curto, médio ou longo prazo e se é viável continua-las quando outros fatores não estiverem favorecendo o ambiente \cite[9]{russell2003artificial}. Abordando o ambiente não só como o \textit{software} e sim como um mundo externo a ele, existe um fundamento, que não será aprofundado por não ter impacto na nossa pesquisa, nomeado cibernética e a teoria de controle. Foi criado por Wiener\cite{wiener1961cybernetics} e visa o uso de componentes elétricos para coletar informações externas á maquina e traduzi-los para uma linguagem (numérica) cuja a ela seja capaz de compreender.

Existem campos, como a própria linguística citada, que procuram sair da forma descritiva que atuam, propondo-se adotar modelos e definições exatas para guiar seus estudos. Esse foi o caso da neurociência, que é responsável por estudar o nosso cérebro, ou sendo mais exato, como nossas redes neurais funcionam \cite[10]{russell2003artificial}. A biologia em si era em grande parte descritiva, embasada por anos de observação e pesquisas que até hoje apoiam suas definições. Sair desses padrões levou os cientistas a vislumbrarem a possibilidade de tratar nosso metabolismo como um centro de transmissão. Esse centro enviaria diversas transmissões que por sua vez emitiriam uma força. Essa afirmação levaria os cientistas a poderem calcular essas a força dessas transmissões afim de chegar em um modelo matemático plausível para replicar o funcionamento de nossas redes neurais \cite[1-3]{rashevsky1960mathematical}. Entretanto, duvidas como até onde pode-se um computador suportar ou superar o processamento de um cérebro humano foram levantadas. Mesmo com tantas discussões em torno do assunto, a proposta de Vinge sobre uma super-máquina que superaria a inteligência humana, também chamada de singularidade tecnológica, continua não tendo uma comparação informativa atualmente, e mesmo que atualmente existisse a capacidade de ter-se memória e processamento infinito ainda não é possível entender como armazenar e replicar os padrões encontrados na neurociência \cite[11-12]{vinge1993coming,russell2003artificial}. Os estudos cognitivos tornaram possível entender melhor o funcionamento da mente humana. O método racional e a proposta de pensar nos meios que nos levam a um fim fez com que fosse capaz de abstrair modelos inteligentes através de agentes. A neurociência, por sua vez, nos deu a magnitude de como transmitir conhecimento e aprimorar os modelos, o fator inteligência e aprendizado passaram a ser mais vistos dentro do ramo de IA, e nos levaram para ascensão da aprendizagem de máquina.

Diferente das ciências onde temos fatos, teorias e pesquisas conclusivas, existem áreas capazes de fomentar intelectualmente assuntos diversos. O conceito de lógica, por exemplo, é algo relativamente descritível e aplicável aos conceitos da ciência, porem a teoria por trás do que viria ser lógico e o que faria um pensamento ser racional é algo muito mais complexo, em suma, o conceito de lógica já era debatido por estudiosos na Grécia antiga. A filosofia tem sua relevância pra IA pois contribuiu com ideias como o \textit{Law of thought} (lei do pensamento), que em contra partida ao pensar humano, é uma lei psicológica que acompanha um processo mental, logo, necessita estar de acordo com uma razão ou lógica \cite[4-5]{frege1956thought russell2003artificial }. Além disso, a filosofia foi responsável por estudos como o racionalismo onde é dito que podemos adquirir conhecimento independente da nossa experiência sensorial. Originada dessa ideia nasceram o dualismo, onde é afirmado que a mente é algo natural e sem conhecimento do mundo externo, e o materialismo, que contrariando o dualismo afirma que a mente é formada pelas operações do cérebro. Em negativa ao racionalismo, existe o empirismo, onde é definido que a experiência sensorial é a fonte final de conhecimento \cite[6]{rationalismvsempiricism, descartes2013rene, russell2003artificial}. Estes estudos tiveram sua relevância sobre o pensamento racional e como seria possível criar maquinas que pensassem e agissem racionalmente.

\subsection{Agentes Racionais}
%Tranformar em intro
Mas seria possível uma maquina pensar como nós e entender as mesmas descrições? Esse foi o questionamento inicial de Turing, a partir disso ele propôs o \textit{Imitation Game},que anos após se tornaria o famoso Teste de Turing. Além disso, algumas diretrizes para construção da maquina capaz de passar em seu teste e objeções a suas próprias afirmações foram dadas. Duas dessas objeções seriam o tamanho finito do armazenamento das maquinas seguido da explicação que apenas reconhecer frases, buscar a informação necessária e levá-la ao usuário não seria o suficiente para passar no teste, as maquinas teriam que ser capazes de guardar instruções e se aprimorar do mesmo modo que um ser humano, assim, seria possível conseguir se adaptar a novas situações \cite[144-155]{turing1950}. Logo, questionar se uma maquina pode pensar é o mesmo que questionar se um submarino pode nadar, essa frase foi utilizada por Dijsktra \cite{dijkstra898} ao explicar que um submarino nunca realizaria o ato de nadar, porem, continuaria executando seu objetivo e propósito da melhor maneria possível.

Quando foi proposto à máquina entender um conjunto de texto e responder corretamente a isso \cite[146]{turing1950}, entramos indiretamente no campo da linguística.

Logo para passar no Teste de Turing, seria necessário criarmos uma maquina capaz de aprender e se adaptar a diversas situações usando Processamento de Linguagem Natural e Referencias de Conhecimento para entender o que foi requisitado além de um algorítimo capaz de automaticamente tomar decisões baseados nos dados fornecidos. \cite[2]{russell2003artificial}. Todas essas etapas podem ser/ou não solucionadas com agentes, que serão explicados a seguir. Durante essa mesma época, em 1956, foi proposto que a quantidade de informações a serem referenciadas seria uma das maiores dificuldades em entender as relações entre as variantes \cite[81-82]{miller1956magical}, de acordo com Russel \cite[13]{russell2003artificial} o inicio do campo da Ciência Cognitiva ocorreu no \textit{workshop} onde esse artigo foi publicado. Essa parceria entre as áreas de computação e psicologia trouxeram grandes avanços ao juntar modelos computacionais com teorias experimentais de psicologia, essa abordagem causou divergências entre estudiosos que discutiam se esse modelo ou um algorítimo performariam melhor na execução de uma tarefa. \cite[3]{russell2003artificial}. O campo da Ciência Cognitiva,  que tinha vasta abrangência como é possível vislumbrar em \textit{“The MIT encyclopedia of the cognitive science”} \cite{wilson2001encyclopedia} que reuniu seis diferentes áreas para abordar o tema, ganhou tração.

%-------[]

O racionalismo por exemplo, visa que podemos adquirir conhecimento independente da nossa experiencia sensorial \cite{rationalismvsempiricism}. Dentro dessa ideia nasceram o dualismo, onde afirmamos que nossa mente é algo natural e sem conhecimento do mundo externo \cite[7]{descartes2013rene}, e o materialismo, que contrariando o dualismo afirma que a mente é formada pelas operações do nosso cérebro \cite[6]{russell2003artificial}.
Em contradição com o racionalismo, existe o empirismo, onde a experiencia sensorial é a fonte final de conhecimento \cite{rationalismvsempiricism}.

Para que se torne possível realizar ações racionais é necessário entender o ambiente aonde você esta e suas variáveis \cite[99]{simon1955behavioral}.

A ação racional descrita é feita por um agente, que é descrito como autônomo e racional, uma vez que não necessita diretamente de um humano para agir e é construído com propósito de retirar a melhor performance com base em seu objetivo \cite[2]{wooldridge1994agent}.

Esse agente esta dentro de um ambiente, é necessário que esteja claro o que pode ou não ser computado, quais são as regras que podemos aplicar e principalmente como conseguimos obter algo racional em caso informações incertas \cite[7]{russell2003artificial}. 

Em resumo o agente recebera um \textit{input}\footnote{Entrada de dados} e será responsável por gerar um \textit{output}\footnote{Dado de Saída}, ao longo do tempo o mesmo agente gerara múltiplas percepções e essas formarão uma sequencia de percepções\footnote{Não serão todos os modelos que seguirão a proposta sequencial, existem casos em que a linha temporal não afeta o desenvolvimento da decisões tornando-as episódicas}. \cite[34-35]{russell2003artificial} Existem definições dadas aos agentes racionais, iremos definir algumas delas a seguir \cite[42-45]{russell2003artificial}:

\begin{itemize}
 \item \textbf{Totalmente, parcialmente ou não observador:} essa definição é gerada pela quantidade de fatores do ambiente que seu agente recebe, um agente que tem todas as informações do ambiente é totalmente observador enquanto um que não recebe nada, precisando assim manter alguns estados, é não observador.
 \item \textbf{Estocástico ou Determinístico:} quando é impossível determinar o próximo estado através do anterior o agente é Estocástico, caso ao contrario ele é Determinístico.
 \item \textbf{Episódicos ou Sequenciais:} já foi dito que em diversas abordagens são gerados sequencias de percepção, quando essa sequencia é alterado a partir de alguma mudança de estado chamamos o agente de sequencial, caso ao contrario o agente é Episódico.
 \item \textbf{Estáticos ou Dinâmicos} essa definição é referente ao ambiente, quando nosso ambiente não infere alterações chamamos o agente de estático, caso ao contrario Dinâmico.
 \item \textbf{Continuo ou Distinto:} Quando existem finitas possibilidades de estado pode se afirmar que o agente é Distinto, quando as possibilidades são infinitas é dado o nome de Continuo.
 \item \textbf{Conhecido ou Desconhecido:} Quando o agente necessita aprender algo e não consegue realizar a ação por si só ele é um agente desconhecido, caso ao contrário ele é conhecido.
\end{itemize}

Para que seja possível definir se o agente está ou não gerando os dados esperados é necessário medir sua performance, então, é necessário que analisar o ambiente gerado a partir das percepções e conferir se os dados são os esperados ou não \cite[294-295]{frege1956thought}.

Racional é algo baseado ou acordado com uma razão ou lógica\footnote{Oxford Dictionarie:  https://en.oxforddictionaries.com/definition/rational}, existem dois pilares para a lógica: a Conversão responsável por expressar a mesma proposição em diferentes formas e o Silogismo responsável por localizar um termo em comum que conecte duas dessas proposições. \cite[175]{boole1854investigation}.
