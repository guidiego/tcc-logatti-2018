\subsection{Coleta}
A coleta será feita utilizando a API publica do twitter, o link para a documentação é \url{https://developer.twitter.com/en/docs}. Será trabalhado no projeto duas entidades: Tweet e Usuário. O tweet é a entidade que representa a publicação do usuário, enquanto o usuário contém informações necessárias sobre o seu perfil.

Para que seja possível acessar a API é necessário criar uma conta de desenvolvimento e gerar o \textit{token} de acesso\footnote{\url{https://apps.twitter.com/app/new}}. Com a chave em mãos é possível replicar o arquivo /collector/client/.env\_sample dentro do projeto Dumont para /collector/client/.env, e conforme demonstrado na Figura \ref{fig:twitteropts}, completar os campos necessários.

Para rodar basta executar o comando \textit{node collector/twitter.js}, e verá a saída de dados. Existem duas observações relevantes a serem feitas nessa etapa:
\begin{itemize}
  \item !AppaTag(tweet) e !AppaTag(user): Se notar, algumas linhas começarão com essas duas anotações, são essas anotações que serão responsáveis por fazer com que o APPA identifique qual entidade de processamento será responsável por processar cada tipo de dado.
  \item Dados de Emoji: É possível notar que alguns dados mapeados não existem na API do Twitter. Como observado, um dos problemas durante a mineração de dados é o uso de \textit{emojis} em textos, já citado também, é possível algumas tarefas de pré-processamento serem executadas durante a própria coleta. Sabendo que \textit{emojis} podem expressar sentimentos \cite{novak2015sentiment}, e que armazenar e tratar esse dado poderia ser relevante na hora de confirmar sentimento em frases, o autor também criou a biblioteca \textit{Emojinator}\footnote{https://github.com/getdumont/emojinator}, além do texto tradado, também será obtida informações do \textit{emojis} utilizados no meio do texto.
\end{itemize}

\begin{figure}
    \centering
    \includegraphics[width=.8\textwidth]{imagens/twitteropts.png}
    \caption{Imagem demonstrando onde cada chave deve ser inserida no código}
    \label{fig:twitteropts}
\end{figure}

Uma vez configurado o coletor, ainda é necessário entender e configurar as outras tarefas para que o APPA funcione apropriadamente, logo, é necessário entender como essas entidades ficarão no final e quais as tarefas que realizarão essa manipulação.
