\section{Tarefas de Processamento}
Como explicado, o APPA ira gerenciar nossas tarefas de processamento, ja foi abordado também alguns atributos que serão necessários para 14BIS executar os modelos lógicos. Tendo isso em vista é necessário que seja implementado \textit{scripts} de mineração e pré-processamento para que a base de dados de dados do projeto tenha a estrutura necessária para que as IAs possam agir.

Essa sessão tem como objetivo detalhar os atributos e \textit{scripts} criados para gerar-los. Além disso em alguns casos, onde são utilizados recursos externos como APIs, será detalhado a configuração necessária.

É importante ressaltar que todo o projeto é \textit{Open Source}, conhecidos como projeto de código aberto, e todo o código criado para esse projeto pode ser encontrado na plataforma de versionamento do GitHub\footnote{\url{https://github.com/getdumont}}.

Antes de iniciarmos as explicações é necessário que para reprodução dessa pesquisa você tenha inicialmente o Docker\footnote{\url{https://www.docker.com/}} instalado. Caso deseje executar as tarefas individualmente, como será citado aqui, é necessário que você tenha instalado na sua máquina o Python 3.6\footnote{\url{https://www.python.org/downloads/release/python-360/}} e o NodeJS 9.11\footnote{\url{https://nodejs.org/en/blog/release/v9.11.1/}}. Vale lembrar que todo código aqui demonstrado foi escrito, executado e testado em um sistema operacional com base Unix, logo a portabilidade com Windows não é garantida.

Uma vez com as ferramentas instaladas no sistema operacional, torna-se possível a reprodução dessa pesquisa. A primeira etapa como já discursada até então se refere a coleta de dados.

\subsection{Coleta}
Para isso existem ainda algumas configurações pendentes, como podemos observar na figura \ref{fig:creds}, existem duas telas, a primeira é o \textit{dumont/sample\_env} e a segunda a réplica criada anteriormente \textit{dumont/dev.env}, foi removido todas as variáveis referentes aos serviços da AWS, já que não será utilizado local como já abordado, além disso foi retirado o usuário e senha do mongo já que o serviço no docker foi configurado para não precisar do mesmo. Além disso a variável \textit{COLLECTOR\_REQUEST\_TOKEN} pois a intuição dela é criar um mínimo de autenticação caso utilize essa API pública na internet. Por fim o \textit{COLLECTOR\_LIMIT} irá limitar a quantidade de usuarios a serem coletados a cada requisição.

\begin{figure}
    \centering
    \includegraphics[width=1\textwidth]{imagens/creds.png}
    \caption{Imagem demonstrando passo a passo de como gerar o JSON de credencial}
    \label{fig:creds}
\end{figure}

É importante lembrar dos limites da API publica do twitter, e tambem que devido a implementação (que pode ser acompanhada no arquivo \textit{dumont/collector/twitter/index.js}) na linha 74 é possível ver o momento onde a \textit{stream} é parada, entretando, podem chegar inumeros tweets de diferentes usuarios simuntaneamente, fazendo assim, com que sobrecarregue a quantiadade de usuarios permitidas na API. É sugerido rodar valores baixos entre 20 a 45 para evitar problemas.

Uma vez configurado, é possivel executar o comando \textit{docker-compose -f docker-compose.dev.yml}, uma vez que o docker inicie o serviço você pode iniciar a coleta utilizando desde um navegador, ferramentas (um exemplo é o Postman\footnote{https://www.getpostman.com/}) ou até mesmo o \textit{curl} utilizando o uri \textit{\url{http://127.0.0.1:8080/}}

Depois que os dados foram coletados, ja é possível minerar algumas informações deles, para isso existem processos a serem detalhados.
