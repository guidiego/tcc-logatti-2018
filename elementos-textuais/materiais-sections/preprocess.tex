\subsection{Tarefas de Pré-Processamento}
O segundo passo após coletar os dados é rodar scripts de pré-processamento e mineração. Uma das maiores dificuldades é como manipular os dados de maneira incremental. Desde que a pesquisa teve inicio, muitas ideias surgiram, novos pontos de vistas e novos dados a serem minerados. Foi adotado uma propriedade chamada \textit{processing\_version}, essa propriedade marca o documento com a versão do processamento dele.

Dentro da pasta \textit{dumont/tasks/processing} existe duas pastas, uma para processar usuários e outra para processar tweets, ambas exportam vários estágios de processamento, esse estágio é exportado e passado para uma classe chamada \textit{Processor} localizada no arquivo \textit{dumont/tasks/processing/\_\_init\_\_.py}, o nível de processamento base é o 0 (nível inserido na hora que o coletor salva do dado no banco), a partir disso é possível atualizar o processamento por um script.

As tarefas de processamento são responsáveis por:

\begin{itemize}
    \item Normalização de Palavras: Transformar girias e erros conhecidos em palavras corretas.
    \item Remoção de \textit{stop-words}: Existem palavras que prejudicam a analise textual por não serem essenciais ou estarem colocadas de maneira equivocada. Um dos processos retira esse tipo de ruído do texto.
    \item Arvore Léxica: Criar uma arvore léxica baseada na frase original do tweet e na frase que já foi tratada removendo as \textit{stop-words}.
    \item Analise de Sentimento: Utilizando a API do Google Language é retirado o sentimento da frase original e da frase tratada também.
\end{itemize}

Para obter esses dados basta rodar o comando \textit{docker-compose up tasks}. Com esses dados já é possível ter alguma noção de informações relevantes dos textos, porém ainda é necessário de embasamento técnico, ou seja, um dado especialista que possa orientar a máquina a utilizar as demais propriedades mapeadas para localizar um delta em comum, entretanto, exigir que todos os dados da massa sejam analisados é algo inviavel. Para isso é necessário gerar uma pré-amostra.