\chapter{Docker}
\label{app:docker}

A modernidade e o avanço na programação levaram os desenvolvedores a publicarem suas aplicações cada vez mais rápido e com maior frequência. Com o tempo se criou o termo DevOps, a abreviação para o que em português seria: “Desenvolvimento e Operação”. Usualmente os times de DevOps eram compostos pelos indivíduos responsáveis por operacionalizar toda a parte de publicação e monitoramento das aplicações. As dificuldades encontradas por times era, e até hoje ainda é, a divergência entre tecnologia perante diferentes ambientes, e principalmente, o tempo que a aplicação demora em seu \textit{Deploy} e \textit{Rollback}\footnote{\textit{Deploy} é o termo utilizado para publicar algo na nuvem, enquanto \textit{rollback} é o termo para retroceder uma versão recém publicada}.

O Docker nasceu para suprir não só esse problema, padronizando ambientes, mas também para isolar as dependências e ferramentas instaladas através dele. Uma vez executado o Docker cria o que chamamos de imagem, é basicamente uma versão minimalista do Linux que tem todas as dependências e códigos da sua aplicação. Com essa imagem gerada é possível executa-la e dar origem a um \textit{container}. Podem existir vários containers rodando simultaneamente, e o mais importante, interligados. Com isso subir um banco de dados especifico sem instala-lo em sua maquina, ou testar sua aplicação com outras versões de uma linguagem, se tornou algo simples. O Docker fez tanto sucesso que as principais fornecedoras de aplicação em nuvem como a Amazon, Google e Azure tem serviços dedicados a rodarem a partir da ferramenta.

Para facilitar a execução, foram configuradas todas as imagens necessárias para que a execução seja rápida e direta. Obviamente coordenar uma grande leva de containers seria um problema, para isso, foi criado o Docker Compose, basicamente um arquivo que administra todas as imagens e interliga elas tornando assim mais fácil a conexão entre os containers. Dentro da raiz do projeto existe um \textit{dumont/docker-compose.yml}, com todas as partes da aplicação. Existe também um \textit{dumont/docker-compose.dev.yml} que é o que será utilizado para rodar a pesquisa na máquina local.

Para baixar a ferramenta basta acessar o site\footnote{\url{https://www.docker.com/products/docker-desktop}} e baixa-la para seu sistema operacional. Uma vez com o Docker e o código em mãos é necessário configurar alguns outros elementos. Obviamente antes mesmo de coletar dados é necessário um lugar para guarda-los, como já abordado utilizaremos o MongoDB.

