\subsection{Processamento de Linguagem Natural}
Se entender palavras não significa entender o contexto, logo, se familiarizar com o ambiente e o momento afim de idealizar o que está sendo transmitido é algo necessário. Essa conexão entre elementos é tratada no estudo do \textit{connectivism}\footnote{Integração dos princípios de rede, caos, complexidade e teorias de auto-organização. Seu objetivo é entender decisões baseado nas mudanças de componentes fundamentais \cite{siemens2014connectivism}.}. De acordo com a linha de pensamento, estabelecida pelo estudo, os neurônios seriam os agentes cognitivos responsáveis por planejar, construir e representar essas informações que o cérebro humano recebe. Criar soluções para problemas pontuais que envolvam a língua que é utilizado no dia-a-dia de uma pessoa, essa é a definição por trás do Processamento de Linguagem Natural (PLN). Fornecer dados linguísticos que a maquina não é capaz de inferir, ou que seja necessário uma ajuda para seu melhor desempenho, é o ponto principal dessa área \cite{brandura1996, maria2015npl}.

Já que o PLN será inicialmente utilizado para análise de palavras-chaves e padrões, é necessário vislumbrar que será necessário um conjunto de regras a fim de melhorar uma determinada métrica durante o processo de aprendizagem. Para que isso seja possível, um conhecimento dentro da área de psicologia se torna altamente relevante.
