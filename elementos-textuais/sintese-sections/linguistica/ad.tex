\subsection{Análise de Discurso}
No decorrer de um texto (que é algo concreto), pode-se caracterizar diversos níveis de geração de sentido.
A primeira formulação de sentido vem do discernimento de termos dentro de um contexto, esse nível é chamado de fundamental. Após distinguir esse primeiro sentido ele é aplicado pelo autor através de um sujeito fazendo com que a prosa tome uma direção, esse nível é chamado narrativo. Por fim, existe o nível discursivo, relacionado as escolhas de tempo, espaço, pessoa e figura durante a narrativa dos fundamentos, dando a essa narrativa um ponto de vista. Logo, o termo discurso é dado como um suporte abstrato por trás do texto, afim da concretização da sua idéia central \cite[13-17]{gregolin1995ad}.

A análise de discurso é, de forma sucinta, uma análise do que foi dito, de como foi dito e qual o sentido do que foi dito. As primeiras manifestações do assunto foram no século XX com autores russos que, além de isolar e definir elementos de uma linguagem poética queriam definir determinantes por trás do perfil artístico do escritor. O tempo fez com que a análise de discurso se desenvolvese e ramificasse em várias vertentes, uma delas a francesa, que apoia a possibilidade de automatizar essa análise por meio da informática. A área continua sendo um campo complexo e de contínuo estudo por trás das definições e metodologias para abordar e sustentar as novas unidades de análise. \cite[22]{souza2006ad}.

Os discursos se diferem de pessoa para pessoa devido ao nível discursivo, a necessidade de expressar um determinado sentido leva o autor a se colocar em um ponto de vista durante sua narrativa. Do contexto da pesquisa, entender o discurso do usuário para mapear o motivo do seu estado mental é um fator de total relevância para entender o estado dele. A pesquisa realizada por Modesto Leite \cite[134]{modesto2005adepre}, mostra em seus resultados que os discursos apresentados pelos pacientes fundamentavam o motivo psicológico do por que os mesmo teriam o transtorno. 

Partindo dos principio apresentados sobre um discurso, por mais que as palavras sejam localizadas, o ponto chave da discussão está em como um computador seria capaz de inferir o sentido da frase. Existem áreas, seguindo os campos multidiciplinares que envolvem linguística e computação, responsáveis por garantir que o processamento dos textos gerará os resultados esperados.
