\section{Psicologia}

A psicologia é descrita como a ciência da vida mental, capaz de entender o posicionamento e o estado emocional do ser, analisando os desejos, razões, sentimentos, decisões entre outras faculdades mentais. Esse entendimento de como o estado mental das pessoas impacta em sua vida é o grande desafio da área \cite[4-8]{william1890principles}.

A pesquisa utiliza da psicologia em dois pontos distintos. O primeiro deles é o envolvimento da psicologia com transtornos de humor, seguido da participação na área da psicologia cognitiva e seu impacto na evolução dos modelos computacionais. Existem vários transtornos de humor descritos pelo DSM-IV\footnote{É um manual famoso que auxilia no diagnóstico de desordens mentais\cite{dsmiv}}, entretanto, as suas causas são variadas. Entender o conceito de como alguns sentimentos impactam esses transtornos é fundamental nesse momento.

% sections
\subsection{Dimensões Afetivas Negativas}
Se caracteriza dimensão afetiva, também conhecida como afetividade, um conjunto de sentimentos que nos afetam positiva ou negativamente \cite[2-6]{pinto2009afetos}. Logo, quando abordado as negativas, pode-se pensar em tristeza, inveja e desesperança por exemplo, porém, o contexto da pesquisa será limitado as dimensões afetivas negativas: depressão, ansiedade e stress.

A depressão é uma forma atenuada de melancolia \cite{roudinesco2000psicanalise}, que por sua vez, seria uma condição maléfica de enfraquecimento da saúde mental de um ser. Classificada como transtorno de humor, a depressão, diferente de outras variações mais regulares de humor, pode causar grandes danos a vida cotidiana uma vez que, por definição, altera a percepção de si mesmo maximizando o peso dos seus problemas diante de sua própria perspectiva \cite{esteves2006depressao}. A melancolia e a depressão compartilham de sintomas em comum como: desânimo, perca de interesse, inibição, bloqueio de sentimentos e outros mais \cite[28]{freud2014livro}. No Brasil há uma média de 11 milhões de pessoas afetadas por esse transtorno \cite{paho2017-letstalk}, sem contar os demais transtornos correlacionados, a partir desses dados já se nota a importância do assunto e as propostas feitas para auxiliar no seu combate.

Existe também a ansiedade. Muitas das desordem relacionadas a ansiedade são categorizadas pelo medo e evasão do usuário a um assunto especifico \cite[393]{dsmiv}. Esse sentimento é primitivo e tem fortes semelhanças as reações animais de defesa ao se colocarem em um ambiente hostil. A inibição, assim como na depressão, é um dos sintomas dos transtornos relacionados a ansiedade, porém, o comportamento do ser usuário assume um formato mais similar a um excesso de vigilância e preocupação \cite{margis2003relaccao}.

Durante os estudos das dimensões afetivas negativas, algumas dificuldades em analisar casos de ansiedade e depressão foram encontradas. Com um menor impacto ao estado clinico, porém, podendo apresentar estados inicias dos outros dois transtornos, o agrupamento de sintomas composto por tensão, irritabilidade e dificuldades para relaxar foi denominado Stress. \cite{lovibond1995structure, ribeiro2004contribuiccao, margis2003relaccao}

Já estabelecidas as diferenças e relações entre a ansiedade, depressão e o stress, torna-se questionável como mensurar seu impacto em um ser humano. Essas dimensões foram selecionadas devido a existência de escalas já testadas para analisar e avaliar pacientes que as portem.



\subsection{Escala de Depressão, Ansiedade e Stress (EADS)}
Adotando um modelo dividido em três sub-escalas a Escala de Depressão, Ansiedade e Stress ou EADS foi proposta na premissa de uma maior assertividade na analise das dimensões afetivas negativas. Já existiam outras pesquisas com propostas similares como a Escala de Ansiedade de Beck e a Escala de Depressão de Beck. As diferenças entre essas escalas e a EADS são: a proposta de correlação entre os transtornos, a introdução de uma escala para o stress e a execução para obter os resultados partindo dessas novas abordagens.

A EADS completa é composta de 42 itens, para fins de facilitar a geração de dados será utilizado o modelo de 21 itens. Essa escala, como já foi citado, é dividida em três sub-escalas representando a depressão, ansiedade e stress. Essa divisão é simétrica, ou seja, cada sub-escala contém sete itens. O mais interessante é o fato, também ja abordado, da EADS propor uma correlação entre as dimensões, logo, um item pode pertencer a uma sub-escala, porém afetar uma segunda. Os itens dentro dessas sub-escalas são afirmações que podem ser respondidas por números de 1 á 4 e representam desde "não se aplica a mim" até "se aplicou a mim na maior parte das vezes". Representam as dimensões mais negativas os maiores valores gerados pela soma dos itens de cada sub-escala \cite{lovibond1995structure, ribeiro2004contribuiccao}.

Diferente da proposta de auto avaliação ou avaliação mediada\footnote{Nesses casos os usuários são responsáveis por responder as perguntas por si mesmos ou por meio de um mediador que ira preencher o formulário.}, como é proposta pela EADS,  um dos objetivos da pesquisa é inferir os resultados das escalas utilizando textos cotidianos de uma amostra coletada no Twitter. Isso nos leva a entender os conceitos cognitivos por trás da psicologia que levaram os autores a propor suas escalas e pesquisas.

\subsection{Psicologia Cognitiva}
Dentro da Psicologia existe um ramo cujo objetivo é entender a capacidade animal de pensar, memorizar, perceber e no caso humano utilizar um dialeto (linguagem), essa área de estudo ficou conhecida como psicologia cognitiva. \cite[3-9]{neisser2014cognitive}

O ato de entender os procedimentos cognitivos adotados pelo ser humano se tornou mais relevante após a proposta da criação de uma inteligência artificial. O questionamento de como uma decisão poderia gerar, afetar ou cancelar uma cadeia de eventos, seja ela durante um pensamento ou uma ação, partindo da capacidade humana de realizar múltiplas tarefas, algumas delas complexas como interpretar linguagens com tantas divergências, fez com que em 1958\cite[4-7]{broadbent1958perception} surgisse o primeiro modelo psicológico que propunha um fluxo similar ao processamento de informações de um computador.

As técnicas experimentais, como os modelos psicológicos, impactaram toda a área interdisciplinar das ciencias cognitivas. Novas abordagens como a junção de modelos criados em computação para IA e técnicas experimentais da psicologia foram criadas a fim do entendimento da mente humana. Durante a criação do \textit{General Problem Solver} ou em tradução literal Solucionador Geral de Problemas, por exemplo, Newell e Simon propuseram não só a implementação de algoritmos descritos por antigos estudiosos afim de resolver um série de problemas, mas também, entender como a máquina estava realizando isso, assim, seria possível uma comparação com a mente humana \cite{newell1961gps, russell2003artificial}.

Nessa pesquisa, a psicologia cognitiva terá um grande impacto em entender como as amostras que contem ou não sintomas causados pelas dimensões afetivas negativas se comportam e pensam. Esse entendimento é essencial para o mapeamento dos atributos que serão processados pela IA afim de gerar os modelos para predição em dados ainda não analisados. Além disso, a psicologia cognitiva faz parte de todo o processo qualitativo de criação, devido a ser um dos vários campos que compões a multidisciplinaridade dentro da IA.
