\chapter*[Introdução]{Introdução}
\addcontentsline{toc}{chapter}{Introdução}

Atualmente a saúde mental é um assunto cada vez mais recorrente e relevante na sociedade, porém os estudos sobre o assunto e seus transtornos já é algo antigo. Um dos marcos nos estudos sobre estados emocionais enfraquecidos foi quando Freud publicou sua obra \textit{“Luto e Melancolia”}\cite{freud2014livro} que teve sua primeira publicação feita em 1916. A psicologia e a psiquiatria avançaram gradualmente no campo referente aos transtornos de humor, a comoção e relevância sobre o assunto teve impacto positivo na sociedade que se engajou mais em prol ao assunto estabelecendo uma maior visibilidade ao tema.

Analisar um paciente com transtornos é uma tarefa cotidiana para um psicólogo ou psiquiatra. A forma com que as pessoas utilizam recursos como fala e escrita deixam traços que podem ser utilizados para analisar padrões emocionais e comportamentais. Cotidianamente, principalmente depois da ascensão das redes sociais, a quantidade de usuários interagindo cresceu, expressar-se publicamente se tornou mais simples.

Devido aos compartilhamentos em redes sociais, a quantidade de conteúdo disponível vem apenas crescendo, basta um rápido acesso a um perfil para mapear informações pessoais, sejam elas exatas como nome, idade, endereço e até familiares ou conteúdos mais abstratos como gostos musicais e literários. O ápice tecnológico citado, que começou aproximadamente no início da década de 2010, deu inicio a era dos dados. O dado organizado é capaz de se mutacionar em informações muitas vezes desejas e necessárias para os usuários, esse potencial juntamente a vontade recente, gerada pelos avanços informática, de obter mais praticidade e facilidade no dia-a-dia, fez com que o dado se tornasse ainda mais valioso. Empresas como Facebook, Google, Uber e Airbnb ganharam o mercado exatamente por conseguirem coletar os dados importantes e administrá-los em forma de soluções para seus usuários.

Aproveitando o momento outras empresas começaram a utilizar dados públicos para criar ferramentas dedicas a analise de mercado e tomada de decisão. A facilidade em mapear visitações, curtidas, compartilhamentos e até publicações opinando sobre algum produto ou serviço impactou o crescimento dessas ferramentas. Aplicar a mesma lógica para analise de perfil com a premissas de caracterizar se o usuário tem algum transtorno é mais complexo, porem factível.

Para mapear tais perfis seriam necessários especialistas analisando e gerando os dados desejados. O maior problema dessa abordagem, que se repete em quase todo processo que é automatizado, é o fator humano. A quantidade de profissionais não é escalável, ou seja, para que seja possível analisar mais dados seria necessário mais profissionais, e a quantidade de profissionais é finita. Além disso, a logística para registrar e manter os dados seria desgastante e não forneceria garantia de dados consistentes.

No entanto a automatização tem seus pontos de dificuldade como o fato da maquina ser lógica. Nesse caso analise de dados abstratos como textos, só seriam possíveis com um forte mapeamento de propriedades como sentimento, sentido e palavras-chaves. O aprendizado de maquina e algoritmos de extração das propriedades citadas seriam de suma relevância, entretanto, uma vez que os dados fossem validados, e a maquina se tornasse capaz de replicar com certa assertividade a analise profissional, a quantidade de dados gerados seria maior e mais concisa devida a escalabilidade de uso de máquinas e a padronização gerada pela automatização do processo.

Partindo da premissa de que se é possível extrair informações necessárias, inferi-las em modelos psicológicos conceituados e, a partir disso, obter-se dados condizentes para mensurar o impacto dos transtornos em um determinado perfil, a proposta deste trabalho é exatamente utilizar de dados públicos do Twitter para coletar perfis e suas respectivas postagens recentes, seguido, em transformar essa base dados em uma base de conhecimento inserindo as propriedades citadas anteriormente, entre outras, com processamento e ajuda de profissionais, para isso além da mineração será implementado um sistema capaz de recolher as analises de psicólogos e psiquiatras.

Para maior veracidade será solicitado ajuda de diversos profissionais, da área de psicologia e psiquiatria para que a base seja consistente e capaz de gerar dados reais e relevantes. Por último, será utilizada essa base de conhecimento para treinar uma IA afim de que ela seja capaz de replicar as analises dos profissionais, fazendo assim, com que a maquina seja capaz de identificar transtornos nos usuários.

Para a pesquisa, foi selecionado o paradigma funcional utilizada nas linguagens como Javascript, Python e Rust para construção do \textit{softwares} e \textit{scripts} responsáveis pelo funcionamento do sistema. Além disso o banco utilizado será o RocksDB para armazenar nosso conjunto de dados pela alta \textit{performance} em leitura e indexação de dados.


