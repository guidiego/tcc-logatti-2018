\chapter*[Introdução]{Introdução}
\addcontentsline{toc}{chapter}{Introdução}

Nos dias atuais a depressão é um assunto cada vez mais recorrente e relevante na sociedade, porém os estudos sobre tal transtorno já é algo antigo. Um dos marcos do estudo da depressão foi quando Freud publicou sua obra \textit{“Mourning and Melancholia”} em 1916. Desde as primeiras citações sobre melancolia e depressão os avanços nas áreas de psicologia e psiquiatria fizeram com que o assunto conquistasse cada vez mais espaço e visibilidade entre as pessoas. Hoje no Brasil temos uma média de 11 milhões de brasileiros afetados por esse transtorno \cite{paho2017-letstalk}.

Analisar um paciente com índices de depressão é uma tarefa cotidiana para um psicólogo ou psiquiatra. A forma com que as pessoas utilizam recursos como fala e escrita deixam traços que podem ser utilizados para analisar padrões emocionais e comportamentais. Na escrita a construção de uma frase já é o suficiente para dizer muito sobre a pessoa. Existem vários estudos que analisam discursos em prol de achar padrões emocionais. A escrita esta presente todos os dias em nosso cotidiano, principalmente depois da ascensão das redes sociais quando o jeito que nos expressamos publicamente se tornou mais simples.

Graças aos compartilhamentos em redes sociais, a quantidade de conteúdo disponível vem apenas aumentando, basta um rápido acesso a um perfil para mapearmos informações pessoais, sejam elas exatas como nome, idade, endereço e até familiares ou conteúdos mais abstratas como gostos musicais e literários. Todo esse ápice tecnológico que começou a aproximadamente quatro anos atrás e deu inicio a era dos dados. Ja que nas últimas duas décadas a facilidade fornecida pelo computador fez com que o ser humano procura-se cada vez mais praticidade e informação, o dado que é capaz de ser organizado a fim de se transformar em informação relevante se tornou algo ainda mais valioso. Empresas como Facebook, Google, Uber e Airbnb ganharam o mercado exatamente por conseguirem coletar os dados importantes e administrá-los em forma de soluções para seus usuários.

Obviamente outras pessoas se aproveitaram desse momento e começaram a utilizar de dados públicos para fazer ferramentas voltadas para analise de mercado e tomada de decisão. Isso se deve a facilidade em mapear visitações, curtidas, compartilhamentos e até publicações opinando sobre algum produto ou serviço. Aplicar a mesma lógica para analisar um perfil e tentar descobrir se a pessoa tem algum índice de depressão é um pouco mais complexo, porem factivel.

A resposta mais lógica para mapear perfis com depressão é chamar especialistas para analisar os perfis e gerar então os dados desejados. O maior problema, que se repete em quase todo processo que é automatizado, é o fator humano. Para analisarmos uma quantidade de dados satisfatória seria necessário um numero de profissionais muito grande, sem contar a logística para registrar esses dados e mante-los consistentes.

Por outro lado, a maquina é lógica. Nesse caso analise de dados abstratos como textos, só seriam possíveis com aprendizado de máquina. Ainda assim existiria outro problema que é o fator da saúde mental. Analisar dados quando o contexto tem pouca variação como em “gostei de x” ou “odiei y”, onde a maior barreira poderia ser o sarcasmo, é bem diferente de analisar uma frase como “Queria estar morta”, onde além do sarcasmo existe a chance da mesma representar um estado emocional ou simplesmente um meme\footnote{Uma imagem, video, ou frase, normalmente de humor, transmitida por usuários na internet. Um elemento cultural transferido por meios não genéticos}.

A proposta do trabalho é utilizar de dados públicos do Twitter para coletar perfis e suas respectivas postagens recentes. Em seguida, será necessário transformar essa base dados em uma base de conhecimento inserindo informações como indice de depressão e palavras chaves, para que isso seja possivel, será implementado um sistema capaz de recolher as analises de psicólogos e psiquiatras e através de funções auxiliares gerar o indicadores falados anteriormente. Para maior veracidade será solicitado ajuda de diversos profissionais, de diversos niveis, da area de psicologia e psiquiatria para que nossa base seja conscistente e capaz de gerar dados reais e relevantes. Por último, sera utilizado essa base de conhecimento para treinar uma rede neural afim de que ela seja capaz de replicar as analises dos profissionais, fazendo assim, com que a maquina seja capaz de identificar índices de depressão autonomamente.

Para a pesquisa, foi selecionado o paradigma funcional utilizada nas linguagens como Javascript, Python e Golang para construção do softwares e \textit{scripts} responsáveis pelo funcionamento do sistema. Além disso o banco utilizado será o MongoDB para armazenar nosso conjunto de dados pela alta performance em leitura e indexação de dados.
