O projeto iniciou com a premissa de se seria possível caracterizar dimensões afetivas negativas em perfis do twitter utilizando inteligência artificial. O avanço da psicologia nos últimos anos, devido a proporção que a saúde mental tem tomado, fez com que cada vez mais demandas por dados e pesquisas quantitavas aparececem. Automatizar essas pesquisas e habilitar os psicólogos, psiquiatras e até mesmo cientistas sociais a terem bases maiores com taxas de acerto alta seria de grande auxilo e magnitude para evolução das pesquisas na área. Com a ascenção das redes sociais esses dados e automatização se tornaram mais viaveis uma vez que o jeito com que as pessoas publicão e se expressão é mais direto, principalmente devido a facilidade de publicar qualquer tipo de coisa de um celular.

Entretanto, chegar a esse ponto requer que etapas sejam desbravadas. Até o presente momento, utilizando dos recursos da Escala de Ansiedade, Depressão e Stress e de conceitos de mineração de dados e inteligência artificial, foi proposto a criação de um \textit{dataset} capaz de inferir questões da EADS em um determinado tweet para que posteriormente fosse feito um ponderamento e inferido a EADS completa para um perfil, caracterizando assim as dimensões afetivas negativas para um determinado usuário.

De toda a base coletada, aproximadamente 1.3\% continha dados relevantes para a pesquisa, 683 perguntas foram coletadas dentre 310 tweets e foi possível criar um modelo inicial de 60\% de presição, 44\% de recuo somando uma pontuação de 42\%. Isso prova que com mais dados e avanços na mineração e no modelo é possível inferir questões da EADS em um tweet, além disso, com a recorrencia das questões nos tweets de um determinado periodo se faz possível predizer as dimensões afetivas negativas pela formula da própria EADS.

É possível chegar a conclusão que a caracterização de dimensões afetivas negativas em perfis do Twitter utilizando inteligência artificial é possível, entretanto, os recursos presentes até o momento nessa pesquisa apenas evidenciaram a possíbilidade e geraram resultados para provar o conceito. A pesquisa estipulada em 9 usuários comprovou que 3 continham conteudo para uma possível análise, entretanto, a base coletada deve ser pelo menos 20 vezes maior, em uma média de 60 perfis válidos seria necessário aproximadamente 200 perfis acompanhados. Mesmo com o fator de falta de dados, foi possível evidenciar que existe uma relação entre os items mais frequentes em um perfil com a taxa de identificação da escala, além disso, foi possível notar algumas relações de constructos entre os 3 perfis análisados, infelizmente, a ausencia de dados torna ilegitima um afirmação mais precisa das relações provando apenas que existe uma veracidade em inferir a frequencia de items com seu valor em uma escala.


Para futuras pesquisas, que darão continuidade ao projeto aqui batizado de Dumont, é necessário um grupo de psicólogos para gerar um base coesa para a máquina consumir, assim podendo reduzir algumas divergencias dentro dos dados analisados. Para isso será executada novas "rodadas" de mineração, validação com usuários e testes de modelos utilizando outros algoritmos e ferramentas. Também é necessário estudar a utilização de outros algoritmos de aprendizagem supervisionada além de testar novos conjuntos de atributos para melhorar a \textit{performance} do modelo inicial. A quantidade de dados também precisa ser revista, pesquisas com massas maiores precisam ser feitas a fim entender as similaridades entre as escalas e dados das respostas para poder inferir a EADS de maneira correta, além de rever a base de respostas para igualar os dados e segregar-los melhor tendo em vista a diminuição do \textit{recall} no primeiro modelo. Por fim, utilizar os dados para predizer as dimensões para um determinado grupo dividido por características como sexo, idade, localidade, frequencia de postagens, interação, assuntos chaves entre outros se torna relevante para gerar dados para futuras pesquisas além da área de T.I.