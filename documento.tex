\documentclass[12pt,oneside,a4paper,chapter=TITLE, english, french,	spanish, brazil]{abntex2-logatti}

\titulo{Caracterização das dimensões afetivas negativas em perfis do Twitter por meio de inteligência artificial.}
\curso{Sistemas de Informação}
\autor{Guilherme Diego Albino Francisco}
\RG{40.741.310-8}
\local{Araraquara - SP}
\data{\MONTH\space / \the\year}
\orientador[Orientadora:]{Cristina Cibeli Vidotti Ivo de Medeiros}
\preambulo{Trabalho de Conclusão de Curso, apresentado às Faculdades Integradas de Araraquara, como requisito parcial para obtenção do título de Bacharel em }

\begin{document}

% Retira espaço extra obsoleto entre as frases.
\frenchspacing 

% ----------------------------------------------------------
% ELEMENTOS PRÉ-TEXTUAIS
% ----------------------------------------------------------
 \pretextual


\imprimircapa

\folhaderosto



\newcommand{\listofgraficosname}{Lista de Gráficos}
\newcommand{\graficoname}{Gráfico}
\newfloat[chapter]{grafico}{logr}{\graficoname}
\newlistof{listofgraficos}{logr}{\listofgraficosname}
\newlistentry{grafico}{logr}{0}

\counterwithout{grafico}{chapter}
\renewcommand{\cftgraficoname}{\graficoname\space}
\renewcommand*{\cftgraficoaftersnum}{\hfill--\hfill}


\pagebreak
\begin{dedicatoria}
 Este trabalho é dedicado à todos que um dia sonharam em melhorar a vida de alguém com grandes ou pequenos atos.
\end{dedicatoria}


\begin{agradecimentos}
Primeiramente gostaria de agradecer aos meus pais, minha irmã e a toda minha família pelo apoio nesses últimos anos. Todas as vezes que tranquei a faculdade ou troquei de curso e mesmo assim sempre acreditarem no meu potencial e me incentivaram.

Gostaria de agradecer a todos os professores da Logatti. Em especial ao Fabio Fornazzari Papini que além de um excelente coordenador sempre foi um grande amigo e me deu grandes conselhos e orientação nessa caminhada profissional. 

Meu agradecimento mais que especial a minha orientadora Cristina Cibeli Vidotti Ivo de Medeiros, que me introduziu a inteligência artificial e sempre acredita em tudo o que me proponho a fazer por mais maluco que seja. 

Por último, porém de longe menos importante a todos os meus amigos, que sempre estiveram do meu lado me levando para frente, gostaria de nomear alguns nesse agradecimento. 

Jéssica Temporal, Leticia Portella e Gustavo Coelho que me levaram a ter interesse por análise de dados e me ensinaram e ainda continuam ensinando muito sobre o mesmo.

João Daher e Mateus Freira que sempre me sedem um pouco do seu tempo para me passar algum conhecimento valioso de inteligência artificial.

Alex Cortes que me ajudarou com toda a parte visual do projeto. E por fim Jaqueline Alves que me ajudou demais com a parte de psicologia cuja a mesma não tinha conhecimento nenhum, foi graças a suas referências que pude tocar essa idéia.



\end{agradecimentos}


\begin{epigrafe}
“You don't understand anything until you learn it more than one way”

\textbf{MINSKY, Marvin. Managing an Information Security and Privacy Awareness and Training
Program (2005)}
\end{epigrafe}



% resumo em português
\setlength{\absparsep}{18pt} % ajusta o espaçamento dos parágrafos do resumo, espaço entre o resumo e as palavras-chave
% \begin{resumo}
%  Segundo a ABNT, o resumo deve ressaltar o
%  objetivo, o método, os resultados e as conclusões do documento. A ordem e a extensão
%  destes itens dependem do tipo de resumo (informativo ou indicativo) e do
%  tratamento que cada item recebe no documento original. O resumo deve ser
%  precedido da referência do documento, com exceção do resumo inserido no
%  próprio documento. (\ldots) As palavras-chave devem figurar logo abaixo do
%  resumo, antecedidas da expressão Palavras-chave:, separadas entre si por
%  ponto e finalizadas também por ponto.

%  \textbf{Palavras-chave}: latex. abntex. editoração de texto.
% \end{resumo}


% resumo em inglês
% \begin{resumo}[Abstract]
%  \begin{otherlanguage*}{english}
%    This is the english abstract. 

%    \textbf{Key-words}: latex. abntex. text editoration.
%  \end{otherlanguage*}
% \end{resumo}



% ---
% inserir lista de ilustrações
% ---
\pdfbookmark[0]{\listfigurename}{lof}
\listoffigures*
\cleardoublepage
% ---


% Lista de Gráficos
% \pdfbookmark[0]{\listofgraficosname}{log}
% \listofgraficos*
% \cleardoublepage

% ---
% inserir lista de tabelas
% ---
% \pdfbookmark[0]{\listtablename}{lot}
% \listoftables*
% \cleardoublepage
% ---

% Lista de Quadros
% \pdfbookmark[0]{\listofquadrosname}{loq}
% \listofquadros*
% \cleardoublepage


\begin{siglas}
  \item[IA] Inteligência Artificial
  \item[EADS] Escala de Ansiedade, Depressão e Stress
  \item[PLN] Processamento de Linguagem Natural
\end{siglas}


% ---
% inserir lista de símbolos
% ---
% \begin{simbolos}
%   \item[$ \Gamma $] Letra grega Gama
%   \item[$ \Lambda $] Lambda
%   \item[$ \zeta $] Letra grega minúscula zeta
%   \item[$ \in $] Pertence
% \end{simbolos}
% ---


% inserir o sumario
% ----
% \setcounter{tocdepth}{0}
\pdfbookmark[0]{\contentsname}{toc}
\setcounter{tocdepth}{0}
\tableofcontents
\cleardoublepage
% ---



% ----------------------------------------------------------
% ELEMENTOS TEXTUAIS
% ----------------------------------------------------------
\textual

\chapter*[Introdução]{Introdução}
\addcontentsline{toc}{section}{\protect\numberline{}Introdução}%
\chapter*[Introdução]{Introdução}
\addcontentsline{toc}{chapter}{Introdução}

Nos dias atuais a depressão é um assunto cada vez mais recorrente e relevante na sociedade, porém os estudos sobre tal transtorno já é algo antigo. Um dos marcos do estudo da depressão foi quando Freud publicou sua obra \textit{“Mourning and Melancholia”} em 1916. Desde as primeiras citações sobre melancolia e depressão os avanços nas áreas de psicologia e psiquiatria fizeram com que o assunto conquistasse cada vez mais espaço e visibilidade entre as pessoas. Hoje no Brasil temos uma média de 11 milhões de brasileiros afetados por esse transtorno \cite{paho2017-letstalk}.

Analisar um paciente com índices de depressão é uma tarefa cotidiana para um psicólogo ou psiquiatra. A forma com que as pessoas utilizam recursos como fala e escrita deixam traços que podem ser utilizados para analisar padrões emocionais e comportamentais. Na escrita a construção de uma frase já é o suficiente para dizer muito sobre a pessoa. Existem vários estudos que analisam discursos em prol de achar padrões emocionais. A escrita esta presente todos os dias em nosso cotidiano, principalmente depois da ascensão das redes sociais quando o jeito que nos expressamos publicamente se tornou mais simples.

Graças aos compartilhamentos em redes sociais, a quantidade de conteúdo disponível vem apenas aumentando, basta um rápido acesso a um perfil para mapearmos informações pessoais, sejam elas exatas como nome, idade, endereço e até familiares ou conteúdos mais abstratas como gostos musicais e literários. Todo esse ápice tecnológico que começou a aproximadamente quatro anos atrás e deu inicio a era dos dados. Ja que nas últimas duas décadas a facilidade fornecida pelo computador fez com que o ser humano procura-se cada vez mais praticidade e informação, o dado que é capaz de ser organizado a fim de se transformar em informação relevante se tornou algo ainda mais valioso. Empresas como Facebook, Google, Uber e Airbnb ganharam o mercado exatamente por conseguirem coletar os dados importantes e administrá-los em forma de soluções para seus usuários.

Obviamente outras pessoas se aproveitaram desse momento e começaram a utilizar de dados públicos para fazer ferramentas voltadas para analise de mercado e tomada de decisão. Isso se deve a facilidade em mapear visitações, curtidas, compartilhamentos e até publicações opinando sobre algum produto ou serviço. Aplicar a mesma lógica para analisar um perfil e tentar descobrir se a pessoa tem algum índice de depressão é um pouco mais complexo, porem factivel.

A resposta mais lógica para mapear perfis com depressão é chamar especialistas para analisar os perfis e gerar então os dados desejados. O maior problema, que se repete em quase todo processo que é automatizado, é o fator humano. Para analisarmos uma quantidade de dados satisfatória seria necessário um numero de profissionais muito grande, sem contar a logística para registrar esses dados e mante-los consistentes.

Por outro lado, a maquina é lógica. Nesse caso analise de dados abstratos como textos, só seriam possíveis com aprendizado de máquina. Ainda assim existiria outro problema que é o fator da saúde mental. Analisar dados quando o contexto tem pouca variação como em “gostei de x” ou “odiei y”, onde a maior barreira poderia ser o sarcasmo, é bem diferente de analisar uma frase como “Queria estar morta”, onde além do sarcasmo existe a chance da mesma representar um estado emocional ou simplesmente um meme\footnote{Uma imagem, video, ou frase, normalmente de humor, transmitida por usuários na internet. Um elemento cultural transferido por meios não genéticos}.

A proposta do trabalho é utilizar de dados públicos do Twitter para coletar perfis e suas respectivas postagens recentes. Em seguida, será necessário transformar essa base dados em uma base de conhecimento inserindo informações como indice de depressão e palavras chaves, para que isso seja possivel, será implementado um sistema capaz de recolher as analises de psicólogos e psiquiatras e através de funções auxiliares gerar o indicadores falados anteriormente. Para maior veracidade será solicitado ajuda de diversos profissionais, de diversos niveis, da area de psicologia e psiquiatria para que nossa base seja conscistente e capaz de gerar dados reais e relevantes. Por último, sera utilizado essa base de conhecimento para treinar uma rede neural afim de que ela seja capaz de replicar as analises dos profissionais, fazendo assim, com que a maquina seja capaz de identificar índices de depressão autonomamente.

Para a pesquisa, foi selecionado o paradigma funcional utilizada nas linguagens como Javascript, Python e Golang para construção do softwares e \textit{scripts} responsáveis pelo funcionamento do sistema. Além disso o banco utilizado será o MongoDB para armazenar nosso conjunto de dados pela alta performance em leitura e indexação de dados.

\cleardoublepage
\chapter{SÍNTESE BIBLIOGRÁFICA}
A pesquisa proposta, assim como os temas abordados nela, é multidisciplinar. As próximas sessões abordarão tópicos necessários para o entendimento do projeto e suas abordagens.

Primeiramente será introduzido a linguística devido ao tema da analise de discurso, essencial para que seja possível analisar como as amostras se expressam em seus textos, além de uma introdução ao processamento de linguagem natural.

Em seguida, será necessário entender sobre Psicologia, dentro dela os tópicos traumas emocionais e modelos de mensuração que serão brevemente abordados devido ao cunho dessa pesquisa. Além disso, será introduzido ao conceito de Psicologia Cognitiva.

Os tópicos já explicados serão utilizados para dar fundamento a Inteligência Artificial. Nessa sessão também seram apresentados os conseitos básicos da área e o conceito de agentes. Em seguida será aprofundado o tema aprendizado de máquina, onde sera explicado conceitos e abordagens utilizadas por essa ramificação da IA.

Por fim, será tratado o tema \textit{data science}, os tópicos relacionados a manipulação de dados, desde a mineração e gestão até a sua representação e organização.

\section{Linguística}
A gramática é composta por: um conjunto finito de letras que formam o chamado alfabeto e um conjunto de regras e normas. Utilizando das regras e do finito número de palavras formadas a partir das letras, é possível se expressar através de uma sentença. O estado representado por essa sentença pode variar de acordo com a regra aplicada. É impossível cobrir todos os estados com uma única regra pelo motivo de existirem números infinitos de sentenças a serem formadas \cite[13-25]{chomsky2002syntactic}. Essa capacidade de obter descrições de forma simplificada através da linguagem é a primeira área cognitiva do ser humano \cite[131]{putnam1975mind}.

No dia-a-dia, existem multiplos fatores que ajudam a entender o sentido de uma frase, porém, em uma máquina os mesmos fatores muitas vezes não se aplicam. Nessa sessão o enfoque é em introduzir alguns estudos fomentados pela linguística. Na inteligência artificial, o ato de juntar símbolos (padrões físicos) em expressões (estruturas) utilizando um conjunto de regras (processos), é considerado um sistema de símbolos físicos. Acredita-se que um sistema desse formato possui os meios necessários e suficientes para realizar ações inteligentes de forma geral \cite[116]{newell1976ComputerSA}. Entretanto, o que foi escrito pode ser compreendido de forma diferente, vide duplo sentidos, isso torna o contexto extremamente relevante para o entendimento. Dentro da linguística existem estudos que permitem analisar o que foi redigido afim de entender o sentido que o autor quis transmitir.

% sections
\subsection{Analise de Discurso}
Ao decorrer de um texto (que é algo concreto), pode-se caracterizar diversos níveis de geração de sentido. O fundamental, responsável pela primeira formulação de sentido a partir do discernimento de termos dentro de um contexto. O narrativo, onde o autor utiliza dos valores fundamentais através de um sujeito tomando a direção da prosa. E por fim, o discursivo, relacionado as escolhas de tempo, espaço, pessoa e figura durante a narrativa dos fundamentos, dando a essa narrativa um ponto de vista. Logo, o termo discurso é dado como um suporte abstrato por trás do texto, afim da concretização da sua ideia central \cite[13-17]{gregolin1995ad}.

A analise de discurso é, de forma sucinta, uma analise do que foi dito, de como foi dito e qual o sentido do que foi dito. As primeiras manifestações do assunto foram no século XX com autores russos que, além de isolar e definir elementos de uma linguagem poética queriam definir determinantes por trás do perfil artístico do escritor. O tempo fez com que a analise de discurso se se desenvolve e se ramificasse em varias vertentes, uma delas a francesa que apoia a possibilidade de automatizar essa analise por meio da informática. A área continua sendo um campo complexo e de continuo estudo por trás das definições e metodologias para abordar e sustentar as novas unidades de analise. \cite[22]{souza2006ad}.

Os discursos se diferem de pessoa para pessoa devido ao nível discursivo, a necessidade de expressar um determinado sentido leva o autor a se colocar em um ponto de vista durante sua narrativa. Do contexto da pesquisa, entender o discurso do usuário para mapear o motivo do seu estado mental é um fator de total relevância para entender o estado dele. A pesquisa realizada por Modesto Leite \cite[134]{modesto2005adepre}, mostra em seus resultados que os discursos apresentados pelos pacientes fundamentavam o motivo psicológico do por que os mesmo teriam o transtorno. Partindo dos principio apresentados sobre um discurso, por mais que as palavras sejam localizadas, o como um computador seria capaz de inferir o sentido da frase se torna o ponto chave em discussão.

\subsection{Processamento de Linguagem Natural}
Partindo do principio de um dicurso, por mais que as palavras sejam localizadas, como inferir o sentido da frase? Mesmo parecendo óbvio, entender palavras não significa entender o contexto, é necessário se familiarizar com o ambiente e o momento afim de idealizar o que está sendo transmitido. Essa conexão entre elementos é tratada no estudo do \textit{connectivism}\footnote{Integração dos princípios de rede, caos, complexidade e teorias de auto-organização. Seu objetivo é entender decisões baseado nas mudanças de componentes fundamentais \cite{siemens2014connectivism}.}. De acordo com a linha de pensamento, estabelecida pelo estudo, os neurônios seriam os agentes cognitivos responsáveis por planejar, construir e representar essas informações que nosso cérebro recebe. Criar soluções para para problemas pontuais que envolvam a ligua que é utilizado no dia-a-dia se uma pessoa, essa é a definição por trás do \textbf{Processamento de Linguagem Natural} \cite{brandura1996, maria2015npl}.

\begin{quote}
Além disso, esse é o ponto principal da contribuição da Linguística para o PLN, qual seja, o de fornecer dados linguísticos que a máquina não é capaz de inferir, mas pode, em parte, processar, melhorando o seu desempenho.
\end{quote}


\section{Psicologia}

A psicologia é descrita como a ciência da vida mental, capaz de entender o posicionamento e o estado emocional do ser, analisando os desejos, razões, sentimentos, decisões entre outras faculdades mentais. Esse entendimento de como o estado mental das pessoas impacta em sua vida é o grande desafio da área \cite[4-8]{william1890principles}.

A pesquisa utiliza da psicologia em dois pontos distintos. O primeiro deles é o envolvimento da psicologia com transtornos de humor, seguido da participação na área da psicologia cognitiva e seu impacto na evolução dos modelos computacionais. Existem vários transtornos de humor descritos pelo DSM-IV\footnote{É um manual famoso que auxilia no diagnóstico de desordens mentais\cite{dsmiv}}, entretanto, as suas causas são variadas. Entender o conceito de como alguns sentimentos impactam esses transtornos é fundamental nesse momento.

% sections
\subsection{Dimensões Afetivas Negativas}
Se caracteriza \textbf{Dimensão Afetiva}, tambem conhecida como afetividade, um conjunto de sentimentos que nos afetão positiva ou negativamente \cite{pinto2009afetos}. Logo, quando abordado as \textbf{negativas}, pode-se pensar em tristeza, inveja e desesperança por exemplo, porem, o contexto da pesquisa sera limitado as dimensões afetivas negativas: depressão, ansiedade e stress.

A \textbf{depressão} é uma forma atenuada de \textbf{melancolia} \cite{roudinesco2000psicanalise}, que por sua vez, seria uma condição maléfica de enfraquecimento da sáude mental de um ser. Classificada como \textbf{transtorno de humor}, a depressão, diferente de outras variações mais regulares de humor, pode causar grandes danos a vida cotidiana uma vez que, por definição, altera a percepção de si mesmo maximizando o peso dos seus problemas diante de sua própria pespectiva. A melancolia e a depressão compartilham de sintomas em comum como: Desanimo, perca de interesse, inibição, bloqueio de sentimentos e outros mais \cite[28]{freud2014livro}.
\subsection{Escala Depressão, Ansiedade e Stress}
Adotando um modelo dividido em 3 sub-escalas a \textbf{Escala de Depressão, Ansiedade e Stress} ou \textbf{EADS} foi proposta na premissa de uma maior assertividade na analise das dimensões afetivas negativas, uma vez que existiam outras pesquisas com proposta similar como o \textit{ Beck Anxiety Inventory} (BAI) e \textit{Beck Depression Inventory} (BDI). As diferenças entre essas escalas são: sua execução, o fator de correlação das sub-escalas propostas pelo EADS e o inventário de Stress introduzido durante o estudo (como mencionado na sessão anterior).

A EADS completa é composta de 42 itens, por fins de facilitar a geração de dados iremos usar o modelo de 21 itens, que são divididos igualmente entre as escalas. Mesmo que um dos itens pertença a uma escala ele pode ter correlação com alguma outra. Esses itens são afirmações que podem ser respondidas por números de 1 a 4 e representam desde "não se aplica a mim" até "se aplicou a mim na maior parte das vezes". Representam as dimensões mais negativas os maiores valores gerados pela soma dos itens de cada subcategoria \cite{lovibond1995structure, ribeiro2004contribuiccao}.

Diferente da proposta de auto avaliação ou avaliação mediada\footnote{Nesses casos os usuários são responsáveis por responder as perguntas por si mesmos ou por meio de um mediador que ira preencher o formulário.}, como é proposta pela EADS,  uma das premissas da pesquisa é inferir os resultados das escalas utilizando textos cotidianos de uma amostra em rede social. Isso nos leva a entender os conceitos cognitivos por trás da psicologia que levaram os autores a propor suas escalas e pesquisas.



\subsection{Psicologia Cognitiva}
Dentro da Psicologia existe um ramo cujo objetivo é entender a capacidade animal de pensar, memorizar, perceber e no caso humano utilizar um dialeto (linguagem), essa área de estudo ficou conhecida como psicologia cognitiva. \cite[3-9]{neisser2014cognitive}

O ato de entender os procedimentos cognitivos adotados pelo ser humano se tornou mais relevante após a proposta da criação de uma inteligência artificial. O questionamento de como uma decisão poderia gerar, afetar ou cancelar uma cadeia de eventos, seja ela durante um pensamento ou uma ação, partindo da capacidade humana de realizar múltiplas tarefas, algumas delas complexas como interpretar linguagens com tantas divergências, fez com que em 1958\cite[4-7]{broadbent1958perception} surgisse o primeiro modelo psicológico que propunha um fluxo similar ao processamento de informações de um computador.

As técnicas experimentais, como os modelos psicológicos, impactaram toda a área interdisciplinar das ciencias cognitivas. Novas abordagens como a junção de modelos criados em computação para IA e técnicas experimentais da psicologia foram criadas a fim do entendimento da mente humana. Durante a criação do \textit{General Problem Solver} ou em tradução literal Solucionador Geral de Problemas, por exemplo, Newell e Simon propuseram não só a implementação de algoritmos descritos por antigos estudiosos afim de resolver um série de problemas, mas também, entender como a máquina estava realizando isso, assim, seria possível uma comparação com a mente humana \cite{newell1961gps, russell2003artificial}.

Nessa pesquisa, a psicologia cognitiva terá um grande impacto em entender como as amostras que contem ou não sintomas causados pelas dimensões afetivas negativas se comportam e pensam. Esse entendimento é essencial para o mapeamento dos atributos que serão processados pela IA afim de gerar os modelos para predição em dados ainda não analisados. Além disso, a psicologia cognitiva faz parte de todo o processo qualitativo de criação, devido a ser um dos vários campos que compões a multidisciplinaridade dentro da IA.

\section{Inteligencia Artificial}

Inteligência é, por definição, uma coleção sistemática de habilidades e funções com objetivo de processar diferentes tipos de informações de diversas maneiras \cite[49]{guilford1982cognitive}. Simular essa "coleção sistemática de habilidades e funções", reconhecer formatos, trajetórias, sinais, deduzir proposições entre outras, criando entidades inteligentes é o foco da inteligência artificial.

Do momento em que a IA passou a ser um campo de estudo até o momento atual, a quantidade de abordagens apresentadas por diversos pesquisadores afim de construir uma máquina inteligente apenas aumentou. Esses métodos se auto-contribuíram propondo novas visões e gerando novas abordagens a partir de descobertas dentro das suas linhas de pensamento. Isso levou a IA para seu estado atual. \cite[1-2]{russell2003artificial}.

Ao decorrer da sessão serão apresentados os fundamentos visando introduzir as demais áreas que colaborarão com a computação para a criação da inteligência artificial, seguido pelas definições de agentes racionais.

\section{Fundamentos da Inteligencia Artificial}

Psicologia, Linguística, Economia, Filosofia e Matemática, esses são alguns dos fundamentos abordados até então. É possível entender o conceito por trás da inteligencia, porem, para que a IA seja bem sucedida e necessário um artefato, e esse artefato tem sido o computador. A engenharia da computação é um outro fundamento, que até agora foi tratado implicitamente, seu foco é construir maquinas eficientes, sendo assim, além dos hardwares, novos sistemas operacionais, linguagens e ferramentas surgiram fortalecendo o campo de IA \cite[13-14]{russell2003artificial}.


Nas últimas sessões foi abordados os temas \textbf{linguistica} e \textbf{psicologia}, e claramente é possivel vislumbrar sua relevancia dentro da área de Inteligencia Artifical. O motivo dessa relevancia é que ambas são consideradas fundamentos da IA assim como outras áreas que serão abordadas aqui.

% Filosofia e intro a agentes
\textit{Law of thought} (lei do pensamento) é uma lei psicológica que acompanha um processo mental, logo, necessita estar de acordo com uma razão ou lógica \cite[289-291]{frege1956thought}. Para as pessoas a lógica parece ser algo extremamente banal, porem, a teoria por trás do que é logico e o que é racional é algo bem extenso. Desde a Grécia antiga filósofos e/ou matemáticos estudam o conceito de lógica. \cite[4-5]{russell2003artificial}. Durante todo esse período varias escolas nos proporcionaram diversas visões sobre o assunto. O racionalismo por exemplo, visa que podemos adquirir conhecimento independente da nossa experiencia sensorial \cite{rationalismvsempiricism}. Dentro dessa ideia nasceram o dualismo, onde afirmamos que nossa mente é algo natural e sem conhecimento do mundo externo \cite[7]{descartes2013rene}, e o materialismo, que contrariando o dualismo afirma que a mente é formada pelas operações do nosso cérebro \cite[6]{russell2003artificial}.
Em contradição com o racionalismo, existe o empirismo, onde a experiencia sensorial é a fonte final de conhecimento \cite{rationalismvsempiricism}.

% Economia
Para que se torne possível realizar ações racionais é necessário entender o ambiente aonde você esta e suas variáveis \cite[99]{simon1955behavioral}. Diferente do que muitos pensam o estudo da economia se trata de dinheiro porem se trata de como guiamos nossas decisões baseadas nos retornos esperados. Os estudos da economia dentro da IA ainda se aplicam a questões como como devemos agir quando o que esperamos esta em um futuro distante ou então se devemos continuar quando outros fatores não continuarem a nos favorecer \cite[9]{russell2003artificial}. A ação racional descrita é feita por um agente, que é descrito como autônomo e racional, uma vez que não necessita diretamente de um humano para agir e é construído com propósito de retirar a melhor performance com base em seu objetivo \cite[2]{ wooldridge1994agent}.

% Matematica
Esse agente esta dentro de um ambiente, é necessário que esteja claro o que pode ou não ser computado, quais são as regras que podemos aplicar e principalmente como conseguimos obter algo racional em caso informações incertas \cite[7]{russell2003artificial}. Os princípios matemáticos aplicados aqui partem das pequisas em cima das proposições, para que seja possível deduzir uma proposição não é necessário apenas inferência, em alguns casos, é necessário que esses dados sejam convertidos a fatores numéricos, afim de utilizar funções aritméticas para resolver os problemas \cite[2-4]{boole1854investigation}. Além disso, existe a probabilidade, em pouco tempo sua capacidade de resolver teoremas inacabados e ajudar a mensurar problemas a tornou indispensável para a IA \cite[9]{russell2003artificial}.

A cibernética e a teoria de controle é um outro fundamento da IA, o conceito por trás desse fundamento foi criado por em 1961 por Wiener \cite[15]{russell2003artificial}. Muitas vezes é necessário que a maquina tenha conhecimento de fatores externos, que usualmente não pode ser transmitido pela ausência de transmissores de informação humanos, como olhos, ouvidos e boca. Se esses estímulos são tão relevantes seria necessário que de forma elétrica fossem criados sensores capazes de controlar essas informações e integrar com os softwares de maneira que passassem dados numéricos e não exatamente seus sentimentos em relação a algo \cite[3-7]{wiener1961cybernetics}. Diferente dos demais, esse fundamento não impacta nossa pesquisa, logo, não será aprofundado como os demais.

A neurociência, é responsável por estudar o nosso cérebro, ou sendo mais exato, como nossas redes neurais funcionam, e é o ultimo fundamento abordado. \cite[10]{russell2003artificial}. A biologia em si era em grande parte descritiva, propor um modelo matemático para o funcionamento de um cérebro vislumbraria em entender o metabolismo como centro de transmissão e calcular a forças emitidas durante esse processo, para que então, fosse transformar a ação ocorrida em um cérebro humano em uma função aritmética \cite[1-3]{rashevsky1960mathematical}. Criar entidades mais inteligentes necessitária de avanços em campos externos ao de computação, como a biologia, para entender se é possível maximizar a inteligencia humana, outras definições como até onde entidades ficariam mais inteligentes, como implementar interfaces para suportar algo talvez superior a inteligencia humana ou então aumentar a capacidade de nossos computadores para que eles fossem capazes de suprir o potencial de um cérebro dependeriam em grande parte de avanços de hardware \cite[1-2]{vinge1993coming}. Mesmo com tantas discussões em torno do assunto, a proposta de singularidade de Vinge não tem uma comparação informativa, e mesmo que fossemos capaz de ter memória e processamento infinito ainda não é possível entender como armazenar e replicar os padrões encontrados na neurociência \cite[11-12]{russell2003artificial}. Os estudos cognitivos tornaram possível entender melhor o funcionamento da mente humana; o método racional e a proposta de pensar nos meios que nos levam a um fim fez com que fosse capaz de abstrair modelos inteligentes através de agentes; a neurociência, por sua vez, nos deu a magnitude de como transmitir conhecimento e aprimorar os modelos, o fator inteligencia e aprendizado passaram a ser mais vistos dentro do ramo de IA, e nos levaram para ascensão da aprendizagem de máquina.
\subsection{Agentes Racionais}
%Tranformar em intro
Mas seria possível uma maquina pensar como nós e entender as mesmas descrições? Esse foi o questionamento inicial de Turing, a partir disso ele propôs o \textit{Imitation Game},que anos após se tornaria o famoso Teste de Turing. Além disso, algumas diretrizes para construção da maquina capaz de passar em seu teste e objeções a suas próprias afirmações foram dadas. Duas dessas objeções seriam o tamanho finito do armazenamento das maquinas seguido da explicação que apenas reconhecer frases, buscar a informação necessária e levá-la ao usuário não seria o suficiente para passar no teste, as maquinas teriam que ser capazes de guardar instruções e se aprimorar do mesmo modo que um ser humano, assim, seria possível conseguir se adaptar a novas situações \cite[144-155]{turing1950}. Logo, questionar se uma maquina pode pensar é o mesmo que questionar se um submarino pode nadar, essa frase foi utilizada por Dijsktra \cite{dijkstra898} ao explicar que um submarino nunca realizaria o ato de nadar, porem, continuaria executando seu objetivo e propósito da melhor maneria possível.

Quando foi proposto à máquina entender um conjunto de texto e responder corretamente a isso \cite[146]{turing1950}, entramos indiretamente no campo da linguística.

Logo para passar no Teste de Turing, seria necessário criarmos uma maquina capaz de aprender e se adaptar a diversas situações usando Processamento de Linguagem Natural e Referencias de Conhecimento para entender o que foi requisitado além de um algorítimo capaz de automaticamente tomar decisões baseados nos dados fornecidos. \cite[2]{russell2003artificial}. Todas essas etapas podem ser/ou não solucionadas com agentes, que serão explicados a seguir. Durante essa mesma época, em 1956, foi proposto que a quantidade de informações a serem referenciadas seria uma das maiores dificuldades em entender as relações entre as variantes \cite[81-82]{miller1956magical}, de acordo com Russel \cite[13]{russell2003artificial} o inicio do campo da Ciência Cognitiva ocorreu no \textit{workshop} onde esse artigo foi publicado. Essa parceria entre as áreas de computação e psicologia trouxeram grandes avanços ao juntar modelos computacionais com teorias experimentais de psicologia, essa abordagem causou divergências entre estudiosos que discutiam se esse modelo ou um algorítimo performariam melhor na execução de uma tarefa. \cite[3]{russell2003artificial}. O campo da Ciência Cognitiva,  que tinha vasta abrangência como é possível vislumbrar em \textit{“The MIT encyclopedia of the cognitive science”} \cite{wilson2001encyclopedia} que reuniu seis diferentes áreas para abordar o tema, ganhou tração.

%-------[]
Em resumo o agente recebera um \textit{input}\footnote{Entrada de dados} e será responsável por gerar um \textit{output}\footnote{Dado de Saída}, ao longo do tempo o mesmo agente gerara múltiplas percepções e essas formarão uma sequencia de percepções\footnote{Não serão todos os modelos que seguirão a proposta sequencial, existem casos em que a linha temporal não afeta o desenvolvimento da decisões tornando-as episódicas}. \cite[34-35]{russell2003artificial} Existem definições dadas aos agentes racionais, iremos definir algumas delas a seguir \cite[42-45]{russell2003artificial}:

\begin{itemize}
 \item \textbf{Totalmente, parcialmente ou não observador:} essa definição é gerada pela quantidade de fatores do ambiente que seu agente recebe, um agente que tem todas as informações do ambiente é totalmente observador enquanto um que não recebe nada, precisando assim manter alguns estados, é não observador.
 \item \textbf{Estocástico ou Determinístico:} quando é impossível determinar o próximo estado através do anterior o agente é Estocástico, caso ao contrario ele é Determinístico.
 \item \textbf{Episódicos ou Sequenciais:} já foi dito que em diversas abordagens são gerados sequencias de percepção, quando essa sequencia é alterado a partir de alguma mudança de estado chamamos o agente de sequencial, caso ao contrario o agente é Episódico.
 \item \textbf{Estáticos ou Dinâmicos} essa definição é referente ao ambiente, quando nosso ambiente não infere alterações chamamos o agente de estático, caso ao contrario Dinâmico.
 \item \textbf{Continuo ou Distinto:} Quando existem finitas possibilidades de estado pode se afirmar que o agente é Distinto, quando as possibilidades são infinitas é dado o nome de Continuo.
 \item \textbf{Conhecido ou Desconhecido:} Quando o agente necessita aprender algo e não consegue realizar a ação por si só ele é um agente desconhecido, caso ao contrário ele é conhecido.
\end{itemize}

Para que seja possível definir se o agente está ou não gerando os dados esperados é necessário medir sua performance, então, é necessário que analisar o ambiente gerado a partir das percepções e conferir se os dados são os esperados ou não \cite[294-295]{frege1956thought}.

Racional é algo baseado ou acordado com uma razão ou lógica\footnote{Oxford Dictionarie:  https://en.oxforddictionaries.com/definition/rational}, existem dois pilares para a lógica: a Conversão responsável por expressar a mesma proposição em diferentes formas e o Silogismo responsável por localizar um termo em comum que conecte duas dessas proposições. \cite[175]{boole1854investigation}.

\section{Aprendizado de Maquina}
Se maquinas pensam ou não, o enfoque da IA é executar diretamente um processo que um ser humano aprendeu ao longo dos anos, e não, algo que implicitamente nasceu com ele. Logo, o fluxo mais sistemático para conseguir que uma maquina abstrai-se formalmente um conhecimento seria mapear um estado inicial, aplicar dados para sua aprendizagem e submete-la a outras experiências afim de reafirmar esses conhecimentos. Essa abordagem foi proposta por Turing em 1950, entretanto, o campo de pesquisa conhecido como \textit{\textbf{machine learning}} ou \textbf{aprendizado de maquina} surgiu apenas em 1959 quando o termo nasceu ao ser aplicado em um estudo de redes neurais. Portanto, pode-se brevemente descrever o aprendizado de máquina como métodos computacionais designados a utilizarem de algoritmos precisos de predição juntamente a uma base de conhecimento para aumentar a sua assertividade\cite[1]{turing1950, samuel1959some, mohri2012foundations}.

Nessa sessão será dado um enfoque maior em \textit{machine learning}, primeiramente será introduzido alguns conceitos que serão utilizados a partir dessa sessão. Com os conceitos passados, sera apresentado, de maneira lacônica, conceitos de utilização dos dados e conceitos dentro do aprendizado de maquina, juntamente com sua aplicação por algumas abordagens. Por fim, será sugerido algumas ferramentas utilizadas para implementação do \textit{machine learning} e um breve enfoque em como guiaremos esse campo dentro de nossa pesquisa.

\subsection{Conceitos do Aprendizado de Máquina}
Previamente foi dito que, seriam necessários dados de treino, esse dado bruto, como já dito também, contem várias propriedades que podem ser ou não processadas afim de gerar informações, esse dado é conhecido com dados de treinamento e um dado que contém as informações processadas necessárias, inserida por algoritimos ou por fatores externos, é chamado de dado rotulado ou anotado.

As propriedades podem ser utilizadas para classificar um determinado dado, isso as concede o nome de atributo. Também conhecido com \textit{feature} é considerada como um dos fatores mais relevantes para o aprendizado da maquina. O termo conhecido como \textit{feature engineering} ou, em tradução literal, engenharia de atributos, diferente do processo de aprendizagem em si, são baseados diretamente a suas entidades e em como escolher e pré-processar seus atributos chaves afim de otimizar o aprendizado \cite{domingos2012few}.

Esses atributos, dentre outras propriedades, estão em uma massa de dados. Por sua vez serão utilizadas como dado de entrada para um agente. Esse agente tambem pode ser representado, por fins explicativos, como uma simples função \textit{f(x) = y}, o objetivo é treinar o agente para que ele seja capaz de generalizar. Ou seja, capaz de predizer o melhor valor para \textit{y} a partir da entrada \textit{x}. Quando o resultado final depende totalmente do dado de entrada, ou seja, não existem muitas variações que levem a saída a grandes divergências, dizemos que o problema generaliza bem.

Essa função gerada que sucintamente traduz os padrões encontrados entre os dados, gera o que conhecemos como modelo. Muitos deles são utilizados para classificar o dado dentro de um grupo de possibilidades, esse tipo de modelo é descrito como classificador.

Os problemas a serem processados pelos classificadores ou outros modelos possuem um conjunto de  hipóteses. Quando é concebida uma preferencia a uma dessas hipóteses com a premissa de induzir o modelo a tomar uma decisão chamamos esse ato de tendência.

Uma quantidade extensa de hipóteses e dados de entrada não significam uma maquina mais assertiva, quando um agente é treinado para executar mais do que o necessário, um fenômeno peculiar chamado de \textit{overfitting} ou sobre-ajustes acontece, esse evento faz com que a maquina seja capaz de tomar decisões assertivas apenas para um conjunto de dados a qual foi submetida e não consegue predizer novas entradas.

As descrições dadas por Russel \cite[693]{russell2003artificial} ainda se aplicam a conceitos mais complexos que não cabem a essa pesquisa explicitamente. O jeito que é utilizado os dados ou os atributos previamente ditos, podem resultar em diversos tipos de abordagens diferentes.

\section{Ciência por trás dos dados}
Como um ser humano, os dados têm seu ciclo de vida, esse ciclo pode alterar-se dependendo de sua finalidade dentro da aplicação. O tema e o objetivo desta pesquisa é coberto por uma matéria, já antiga, nomeada KDD ou \textit{Knowledge Discovery in Database}, como o próprio nome traduzido sugere, a Descoberta de Conhecimento em Base de Dados abrange varias subclasses como mineração de dados, analise de dados, probabilidade e até mesmo a própria IA. A KDD se justifica pela sua metodologia de manipulação de dados afim de extrair o conhecimento almejado de um base. Se observar o fluxo descrito na figura \ref{fig:kdd}, pode-se notar que uma cadeia de eventos como: seleção, pré-processamento, mutação, mineração e interpretação. Esses eventos geram novos estados para o dado, e os moldam focando o objetivo de retirar o conhecimento pela interpretação no final do processo.

\begin{figure}[H]
    \centering
    \includegraphics[width=.8\textwidth]{imagens/kdd.png}
    \caption{Representação da metodologia defendida pela KDD. Fonte: o autor.}
    \label{fig:kdd}
\end{figure}

Com o passar do tempo, várias abordagens surgiram para tratar dados e essa massa de pesquisas e definições deu-se por bagunçar em partes as nomenclaturas e definições que estão atrelados a ciência por trás dos dados. Para evitar confusões as definições apresentadas nessa sessão serão sucintas e tem como objetivo futuros tópicos e discussões apresentados durante os resultados dessa pesquisa. Devido a grande divergência, ocorrida pela disseminação rápida de algumas termologias, os termos abordados serão guiados pelas seguintes referencias \cite{laender2002brief, fayyad1996kdd, hand2007principles}.

\subsection{Seleção de Dados}
O ato de selecionar os dados com enfoque no conhecimento que se deseja obter deles afim de gerar um conjunto de dados especifico é também conhecido como \textit{data collection}, durante o processo do KDD é o primeiro passo responsável por gerar um amostra mais focada no problema.

\subsection{Mineração de Dados}
Nessa subseção propõe-se que seja dissertado sobre os processos do KDD a partir da seleção até a interpretação. Vale ressaltar que para obter conhecimento não é necessário uma IA, sistemas de tomada de decisão trabalham com probabilidade matemática sob dados exatos, o que em muitos casos, já seria o suficiente para obter a informação do dado. Entretanto, o foco da pesquisa se baseia na implementação de um sistema inteligente e isso inclina essa explicação para o fato de: o aprendizado de máquina é uma das possíveis formas de se minerar um dado.

A mineração de dados ou popularmente conhecida como \textit{data mining}, baseia-se em retirar os valores mais relevantes e valiosos para se inferir um conhecimento. Os demais passos como pré-processamento e mutação se assemelham a alguns processos citados anteriormente como a redução dimensional ou ainda a engenharia de atributos.


\subsection{Análise de Dados}
E finalmente, a análise. O passo descrito como interpretação no KDD se refere a entender os dados de saída vindos da mineração afim de afirmar ou descarta hipóteses, com isso é possível refinar o processo e explorar novas possibilidades. Uma das palavras que hoje se tornaram popular dentro dá área é o \textit{data storytelling}, ou, o ato do dado contar uma história. Abordar a hipótese de maneira empírica e demonstrar a veracidade dela discutindo abordagem e algoritmos gerados é um dos objetivos dessa área.




\cleardoublepage
\chapter{MATERIAIS E MÉTODOS}
O objetivo da pesquisa é caracterizar dimensões afetivas negativas em perfis do twitter localizando pontos comuns entre usúarios portadores de afetividades negativas (stress, ansiedade e depressão) de mesmo nível. Esse tipo de processo se asemelha com a aprendizagem não supervisionada, logo, antes dessa etapa é necessário mapearmos atributos, sendo assim, é necessário a identificação de usúarios com dimensões afetivas negativas em primeiro momento.

Como observado, existem vários passos para conclusão desse projeto, abertamente estrutura de processamento contára com um processo de mineração e dois processos de inteligencia artificial afim de gerar dois modelos lógicos. O primeiro modelo responsavel por inferir valores da EADS em um perfil, e o segundo, de predizer, a partir de dados do perfil, a probabilidade de existir um determinado nível de afetividade utilizando dados do perfil.

Projeto em geral tem alguns outros pontos sociais envolvidos, sendo um deles captação de dados embasados através de profissionais, entretando, nosso objetivo inicial é fazer a máquina coletar e acertar as previsões com qualquer tipo de dado (mesmo que fornercido por pessoas não capacitadas, no caso uma base anotada pelo próprio autor), sendo assim, será detalhado o sistema em si (coleta de dados e aprendizagem de maquina) e as metodologias utilizadas para desenvolve-lo.

Pode-se observar na Figura \ref{fig:tecnologias}, o sistema é divido em dois núcleos, o Dumont responsável por minerar e gerar toda a base de dados, e o 14BIS que será responsavel pelas Inteligencias Artificias.

\begin{figure}
    \centering
    \includegraphics[width=0.9\textwidth]{imagens/tecnologias.png}
    \caption{Desenho apresentando os núcleos do projeto}
    \label{fig:tecnologias}
\end{figure}

Existem basicamente 5 técnologias que estão sendo utilizadas nesse projeto:
\begin{itemize}
 \item Python: É a linguagem atual mais utilizada no mundo de Aprendizado de Máquina, sua simplicidade ja á torna simples de usar, porém, a quantidade de materiais, bibliotecas e artigos sobre PLN e Aprendizado de Máquina á tornam a principal linguagem nesse projeto.
 \item Node/Javascript: Node é o interpretador que permite com que seja possivel executar o Javascript (linguagem originalmente de navegar no servidor). A linguagem tem um grande ganho com integrações e será utilizada para consumir recursos vindos de APIs.
 \item Go: Linguagem fortemente e estaticamente tipada, que fornecesse um poder de processamento equivalente a da linguagem C, entretando, muito mais simples de escrever e manter código, será utilizado para scripts onde serão necessário processamento de muitos dados.
 \item MongoDB: O Mongo é um banco não relacionado, em resumo, um grande banco de documentos indexados por atributos especificos, fornece um grande poder de leitura alem do fato de não ser prezo a estruturas pré-definidas como banco relacionais (isso facilita para que sejam inseridos novos dados futuramente sem grandes custos).
 \item Docker: Docker é uma ferramenta para infra-estrutura, será utilizado para rodar a aplicação em containers e facilitar o \textit{deploy}\footnote{Vindo do termo em inglês "lançar" é utilizado para o ato de colocar uma aplicação em ambiente de produção}.
\end{itemize}

Todo o código dos núcleos está disponível no GitHub utilizando o GIT. Para entender como funciona a plataforma de versionamento, que pode ser tambem utilizada para achar informações mais técnicas alem de baixar o código basta seguir as instruções conforme o \autoref{app:git}. Além disso como mostrado ambos os núcleos utilizam Docker como base para executar as aplicações, isso devido ao resultado de tal abordagem reduzir alguns problemas possíveis em relação a ambiente de execução. É possível obter as informações necessárias para entendimento e instalação conforme o \autoref{app:docker}.

Uma vez obtido conhecimento sobre os núcleos do projeto e suas técnologias, é possível idealizar o fluxo completo e interação entre elas. Se observar a Figura \ref{fig:tcc_caso_de_uso}, notara que o Dumont ira utilizar da API do twitter para coletar dados públicos, posteriormente esses dados serão processados e mutacionados a fim de gerar uma base de dados, por final essa base dados será salva em um banco de dados.

\begin{figure}[!h]
    \centering
    \includegraphics[width=.5\textwidth]{imagens/tcc_caso_de_uso.png}
    \caption{Diagrama de caso de uso do sistema}
    \label{fig:tcc_caso_de_uso}
\end{figure}

Para salvar os durante a coleta ou o pré-processamento será utilizado o MongoDB. O MongoDB  é um banco não relacional, ou seja, um banco que não tem \textit{transactions}\footnote{Simboliza uma atividade realizada em um banco relacional} e é baseado inteiramente em documentos. Diferente de um banco não relacional, os dados inseridos nesse tipo de banco não tem uma normalização ou qualquer tipo de padrão. No caso do Mongo a unica identificação utilizada para interligar os documentos é a nomeada Coleção\footnote{As Coleções servem para agrupar tipos especificos de documentos.}.

Dentro do trabalho existem algumas estruturas mais fechadas e outras que irão variar mais com o passar do desenvolvimento, pode-se notar na Figura \ref{fig:entities}, a visão inicial do que seria a estrutura de nossos documentos. Teremos duas coleções mais relevantes \textit{Tweet} e \textit{User} aqui ficaram armazenados os dados do Twitter, as demais estruturas são estruturas periféricas criadas para suportar o sistema de armazenamento de dados especialistas. Para isso existe a estrutura \textit{Answer} que serve para que os especilistas inserirem a possibilidade de um mapeamento do mesmo dentro de algum dos items da EADS. Por final existem mais duas coleções a \textit{Specialist} que armazena os especialistas registrados no nosso sistema e o \textit{List} que é uma coleção auxiliar para agrupar uma quantidade de tweets tornando mais facil para nossa aplicação apresenta-los aos especialista e coletar analise.

\begin{figure}[!h]
    \centering
    \includegraphics[width=.65\textwidth]{imagens/entities.png}
    \caption{Mapa de entidades do projeto}
    \label{fig:entities}
\end{figure}

Vale resaltar que o objetivo é gerar uma base de conhecimento, e isso implica na ausencia de profissionais agindo na base nesse primeiro trabalho. Além disso aplica-la a uma aprendizagem supervisionada e em seguida aplicar esse dado a uma não supervisionada. Logo a primeira etapa é adquirir o \textit{dataset} que será utilizado.

\section{Coletando Base de Dados}
A primeira etapa do processo inclui coletar dados para que, posteriormente, seja possível mineirar atributos dentro deles. O coletor é algo simples, basicamente um \textit{script} que roda de tempos em tempos baseado na configuração apresentada. Foi desenvolvida uma função responsável por coletar em tempo real tweets em uma determinada área e que em seu corpo tivesse uma ou mais palavras chaves. Com isso é possível pesquisar um usuário e seus últimos 200 tweets.

De todos os modelos previamente explicados, os utilizados dentro do coletor são os de \textit{user} e \textit{tweet}. Eles podem ser encontrados dentro de \textit{dumont/collector/schemas}. Antes do dado ser salvo é possível notar a criação de um objeto partindo do atributo de texto referente ao tweet. Um dos problemas durante a mineração de dados é o uso de \textit{emojis} em textos. Sabendo que \textit{emojis} podem expressar sentimentos, armazenar e tratar esses dados poderia ser relevante na hora de confirmar o sentimento em frases, durante esse processo a lógica para localizar e extrair esses elementos foi abstraida para uma biblioteca chamada \textit{Emojinator}\footnote{https://github.com/getdumont/emojinator}. Além do texto tratado, também será obtida informações do \textit{emojis} utilizados no meio do texto.

Para ativar o coletor foi criado um CLI\footnote{CLI é abreviatura para \textit{Command Line Interface}, em outras palavras, uma interface que permite executar códigos direto do terminal}. Antes de rodar o comando é necessário configurar o projeto conforme o \autoref{app:configuracoes}. Após terminar a configuração deve ser executado o comando \textit{docker-compose up collector}, logo que finalizar a coleta vai preencher o MongoDB com os dados necessários para as próximas etapas. Durante essa pesquisa o comando foi rodado diversas vezes em um periodo de tempo, para criar uma base inicial. Vale lembrar que a idéia final se consiste em um mapeando periódico de perfis, porém nesse momento o coletor foi feito unicamente para juntar um aglomerado de dados indiferente de seus perfis.

No primeiro momento que o coletor foi rodado na pesquisa, foram coletados um total de 68583 tweets distribuídos entre 419 perfis, gerando uma média de aproximadamente 163 tweets por usuário. A massa de dados é bem ampla e achar uma amostra que suprisse as necessidades poderia ser algo complexo. Para isso o enfoque inicial é localizar uma amostra com os perfis que contem a maior massa de tweets negativos, entretanto, isso não é possível sem idealizar uma segunda parte do processo, no caso o pré-processamento e a mineração.

\subsection{Tarefas de Pré-Processamento}
O segundo passo após coletar os dados é rodar scripts de pré-processamento e mineração. Uma das maiores dificuldades é como manipular os dados de maneira incremental. Desde que a pesquisa teve inicio, muitas ideias surgiram, novos pontos de vistas e novos dados a serem minerados. Foi adotado uma propriedade chamada \textit{processing\_version}, essa propriedade marca o documento com a versão do processamento dele.

Dentro da pasta \textit{dumont/tasks/processing} existe duas pastas, uma para processar usuários e outra para processar tweets, ambas exportam vários estágios de processamento, esse estágio é exportado e passado para uma classe chamada \textit{Processor} localizada no arquivo \textit{dumont/tasks/processing/\_\_init\_\_.py}, o nível de processamento base é o 0 (nível inserido na hora que o coletor salva do dado no banco), a partir disso é possível atualizar o processamento por um script.

As tarefas de processamento são responsáveis por:

\begin{itemize}
    \item Normalização de Palavras: Transformar girias e erros conhecidos em palavras corretas.
    \item Remoção de \textit{stop-words}: Existem palavras que prejudicam a analise textual por não serem essenciais ou estarem colocadas de maneira equivocada. Um dos processos retira esse tipo de ruído do texto.
    \item Arvore Léxica: Criar uma arvore léxica baseada na frase original do tweet e na frase que já foi tratada removendo as \textit{stop-words}.
    \item Analise de Sentimento: Utilizando a API do Google Language é retirado o sentimento da frase original e da frase tratada também.
\end{itemize}

Para obter esses dados basta rodar o comando \textit{docker-compose up tasks}. Com esses dados já é possível ter alguma noção de informações relevantes dos textos, porém ainda é necessário de embasamento técnico, ou seja, um dado especialista que possa orientar a máquina a utilizar as demais propriedades mapeadas para localizar um delta em comum, entretanto, exigir que todos os dados da massa sejam analisados é algo inviavel. Para isso é necessário gerar uma pré-amostra.
\section{Amostras e Análise de Dados}
Com os textos devidamente pré-processados e com os atributos iniciais minerados, é necessário a coleta de uma amostra para que seja possível em uma segunda parte do processo, adicionar os atributos referentes a futura inferência da EADS, para que essa amostra seja coletada é necessário que em primeiro instante seja feita uma análise dos dados. Para entender a próxima análise é necessário compreender que a análise de sentimento retorna dois fatores: \textit{score} e magnitude, o \textit{score} varia entre 1 e -1, sendo 1 representativo ao sentimento positivo e -1 o negativo, logo 0 seria neutro. Além disso existe a magnitude que é para expressar o quão o sentimento é presente no texto. No caso do Twitter onde postagens tendem a ser mais diretas e sucintas pelo seu tamanho, a magnitude tende a ser muito variável.

Para maior eficiencia os tweets foram pelos \textit{scores}, a cada espaçamento de 0.2. Ao observar o Figura \ref{fig:sentiment-relation}, notasse que as extremidades tendem a ter menos tweets, média de 1000 a 2500 tweets por agrupamento, entretanto grupos mais centralizados (emoções neutras) chegam a picos de 30000 tweets, já que o dado desejado é uma massa negativa e seus usuários será dado enfoque ao lado negativo do mapeamento.

\begin{figure}[!ht]
    \centering
    \includegraphics[width=.6\textwidth]{imagens/relacao-sentimento.png}
    \caption{Relação de sentimentos por tweet dentro da massa de dados}
    \label{fig:sentiment-relation}
\end{figure}


Partindo do principio que a partir de um \textit{score} de -0.2 já existem menos dados, foi tomada como premissa que qualquer dado com \textit{score} superior a esse é um dado não viável para amostra. Seguindo essa premissa é preciso mapear de cada usuário o percentual de publicações a abaixo de -0.2, com isso será possível obter um percentual da densidade de publicações negativas naquele perfil. Analisando os dados da Figura \ref{fig:negative-pop-relation}, é possível notar que a maior quantidade de percentual negativo esta entre 20\% e 40\%, entretanto, é impossível afirmar que os dados suprem as necessidades que a inteligência artificial ira necessitar para realizar a inferência na base. Refletindo sobre a EADS, a versão simplificada tem 21 perguntas, logo, necessita-se de no mínimo 21 tweets negativos na esperança que cada um responda pelo menos 1 das perguntas. Para validar a idéia é necessário descobrir a média de tweets dentro dos percentuais que mais contém massa negativa.

\begin{figure}[!ht]
    \centering
    \includegraphics[width=.6\textwidth]{imagens/relacao-massa-neg.png}
    \caption{Percentual de massa negativa por quantidade de perfis}
    \label{fig:negative-pop-relation}
\end{figure}


Entretanto, pode-se notar mais um fator no dado, existem perfis com 100\%, ou outros percentuais abrangentes, de sentimento negativo devido a conter poucos tweets, porém, todos ou a maioria negativo. Logo pode-se afirmar que a quantidade de tweets é algo relevante. Como já abordado a EADS é composta por 21 questões, logo, se fosse utilizar do fator sorte, seriam necessários 21 tweets negativos, onde cada uma respondesse uma pergunta diferentes da escala. Entretanto, o fato de a quantidade de tweets negativos entre 20\% a 40\% de massa negativa é equivalente a 9592, de um total de 413 usuários já aumenta essa limite minimo para 23 (média de tweets negativos da amostra analisada).

Visando o aumento da eficácia, contar com um limite baixo poderia prejudicar a análise, logo foi executado o cálculo \[ (T * C) + M / D \] onde \(T\) era o total de perguntas da EADS (21), \(P\) era o potencial de ampliação de chances (2), \(M\) é a média da massa retirada do cálculo anterior (23), \(D\) o total de dissolução da quantidade acumulada, no caso, foi utilizado o valor 2 contemplando a média basica entre o total de perguntas e média da massa. O resultado proveniente do cálculo foi aproximadamente 32 tweets. Dessa amostra foram minerados 21 perfis e 907 tweets negativos. Ao observar as Figura \ref{fig:sample-relation} é possível notar que a média de tweets negativos pegos para análise em relação a base é de 1.3\% enquanto em relação aos usuários 5\%.

\begin{figure}[!ht]
    \centering
    \includegraphics[width=.6\textwidth]{imagens/relacao-amostra.png}
    \caption{Percentual de dados utilizados na amostra em relação a base total de tweets e usuários}
    \label{fig:sample-relation}
\end{figure}

Com uma base menor as chances de achar tweets que respondam as questões necessárias aumentam, o próximo passo seria simular a entrada de dados especialistas na base.



\chapter{RESULTADOS, ANÁLISE E DISCUSSÃO}
Tratar as metodologias e abordagens por partes será a melhor abordagem para analisar os resultados e gerar discussões de forma mais clara e coesa.

A primeira parte a ser analisada é a mineração de dados. No final foram obtidos 21 perfis, de 907 tweets negativos, desses 310 continham uma ou mais questões da EADS ligadas a ele, contabilizando um total de 688 questões respondidas. Representando assim 34\% dos tweets negativos tendo em média 2.2 questões da EADS relacionadas a ele.

Sobre a mineração alguns pontos e conclusões podem ser retirados. A primeira é a falta de outras "rodadas" de coleta e mineração com outros conjuntos de palavras chaves poderiam ter feito a diferença, uma vez que o corpo de perguntas menos respondidas são conhecidos, uma mineração envolvendo um aumento desse tipo de tweet poderia nivelar a base de conhecimento. E por fim se a falta de profissionais da psicologia/psiquiatria dentro do projeto. Com conhecimento mais sólido e tempo seriá capaz mapear, em caso de não localizar, frases que contemplassem as questões com menos dados, podendo assim, preencher o banco de dados manualmente. Além disso, com um grupo maior de profissionais a chance das respostas serem mais apropriadas aumentam, sem contar o fato de uma quantidade maior de profissionais poderia dar conta de analisar "rodadas" de mineração cada vez maiores.

A segunda parte implica sobre o modelo de predicção de questões EADS. Por mais acertividade que algo com poucos dados tenha conquistado o \textit{score} do modelo aplicando Naive Bayes foi de 42\%, onde já foi evidenciado que a falha nos \textit{recalls} esta atrelado com a falta de dados. Ao observar a Figura \ref{fig:rpr} pode-se notar a relação inversa de precisão e recuo perante a quantidade de tweets classificados. Quanto maior a quantidade de tweets maior a precisão e menor a taxa de recuo.

\begin{figure}[!ht]
    \centering
    \includegraphics[width=.75\textwidth]{imagens/rpr.png}
    \caption{Relação de sentimentos por tweet dentro da massa de dados}
    \label{fig:rpr}
\end{figure}

Entretanto, na Figura \ref{fig:qpp} pode-se notar que a pontuação não segue esse padrão. Por determinado motivo a taxa de melhor conversão, relacionando os gráficos, seria entre 60 a 80 tweets. Porém, o questionamento fica explicitado, o melhor resultado viria de um banco mais heterogeneo, onde todas as perguntas tivessem média de 60 a 80 tweets classificados.

\begin{figure}[!ht]
    \centering
    \includegraphics[width=.8\textwidth]{imagens/qpp.png}
    \caption{Relação de sentimentos por tweet dentro da massa de dados}
    \label{fig:qpp}
\end{figure}

Logo, ter muitos dados seria algo relativo, o importante em fato, seria ter uma quantidade plausivel de dados em cada item, sem uma divergencia muito grande, como acontece principalmente com o a questão "Não tive nenhum sentimento positivo". Para evidenciar isso e partindo da premissa que primeira análise que gerou os gráficos a partir dos tweets minerados para o cenário, foi gerada uma segunda predicção adicionando na massa os dados coletados dos usuários na segunda parte da pesquisa. Aumentando assim a quantidade de itens de aproximadamente 690 para aproximadamente 1100 repostas. Se acompanhar a Figura \ref{fig:prediction2} é possível notar uma queda no \textit{score}, o que indica que uma base com dados pareados tende a ser mais efetiva do que uma base com muitos dados dispersos. Vale lembrar que a divergencia das questões pode vir a ser um problema de falta de dado especialista. Uma vez que os dados utilizados foram inseridos a partir do conhecimento teórico do autor com ajuda de algumas das biografia citadas préviamente na pesquisa.

\begin{figure}[!ht]
    \centering
    \includegraphics[width=.5\textwidth]{imagens/prediction2.png}
    \caption{Tabela de resultados do Naive Bayes para 1091 resultados}
    \label{fig:prediction2}
\end{figure}

Por fim, para inferir a EADS em si, é necessário recorrencia, em outras palavras, inferir um item é apenas uma parte do processo, a escala é feita semanalmente, logo uma pergunta isolada pode ter menos representatividade do que uma pergunta que aparece recorrentemente em um periodo de tempo. A falta de dados não permite que afirmações especificas relacionadas ao modelo sejam feitas, entretanto, é possível dizer que existe relação entre a expressão de um perfil no Twitter ao seu estado emocional, logo é possível, inferir a EADS utilizando recursos da inteligencia artificial. Se em um caso de poucos recursos e dados foi possível obter uma pontuação de 42\%, em um cenário melhor pesquisas futuras podem chegar a acertos de 70\% a 80\%, igualando hoje com pesquisas similares que unicamente classificão publicações como depressivas, o que de certo modo não contempla toda a complexidade da pesquisa apresentada até então.

\chapter*[Considerações Finais]{Considerações Finais}
O projeto iniciou com a premissa de se seria possível caracterizar dimensões afetivas negativas em perfis do twitter utilizando inteligência artificial. O avanço da psicologia nos últimos anos, devido a proporção que a saúde mental tem tomado, fez com que cada vez mais demandas por dados e pesquisas quantitavas aparececem. Automatizar essas pesquisas e habilitar os psicólogos, psiquiatras e até mesmo cientistas sociais a terem bases maiores com taxas de acerto alta seria de grande auxilo e magnitude para evolução das pesquisas na área. Com a ascenção das redes sociais esses dados e automatização se tornaram mais viaveis uma vez que o jeito com que as pessoas publicão e se expressão é mais direto, principalmente devido a facilidade de publicar qualquer tipo de coisa de um celular.

Entretanto, chegar a esse ponto requer que etapas sejam desbravadas. Até o presente momento, utilizando dos recursos da Escala de Ansiedade, Depressão e Stress e de conceitos de mineração de dados e inteligência artificial, foi proposto a criação de um \textit{dataset} capaz de inferir questões da EADS em um determinado tweet para que posteriormente fosse feito um ponderamento e inferido a EADS completa para um perfil, caracterizando assim as dimensões afetivas negativas para um determinado usuário.

De toda a base coletada, aproximadamente 1.3\% continha dados relevantes para a pesquisa, 683 perguntas foram coletadas dentre 310 tweets e foi possível criar um modelo inicial de 60\% de presição, 44\% de recuo somando uma pontuação de 42\%. Isso prova que com mais dados e avanços na mineração e no modelo é possível inferir questões da EADS em um tweet, além disso, com a recorrencia das questões nos tweets de um determinado periodo se faz possível predizer as dimensões afetivas negativas pela formula da própria EADS.

É possível chegar a conclusão que a caracterização de dimensões afetivas negativas em perfis do Twitter utilizando inteligência artificial é possível, entretanto, os recursos presentes até o momento nessa pesquisa apenas evidenciaram a possíbilidade e geraram resultados para provar o conceito. A pesquisa estipulada em 9 usuários comprovou que 3 continham conteudo para uma possível análise, entretanto, a base coletada deve ser pelo menos 20 vezes maior, em uma média de 60 perfis válidos seria necessário aproximadamente 200 perfis acompanhados. Mesmo com o fator de falta de dados, foi possível evidenciar que existe uma relação entre os items mais frequentes em um perfil com a taxa de identificação da escala, além disso, foi possível notar algumas relações de constructos entre os 3 perfis análisados, infelizmente, a ausencia de dados torna ilegitima um afirmação mais precisa das relações provando apenas que existe uma veracidade em inferir a frequencia de items com seu valor em uma escala.


Para futuras pesquisas, que darão continuidade ao projeto aqui batizado de Dumont, é necessário um grupo de psicólogos para gerar um base coesa para a máquina consumir, assim podendo reduzir algumas divergencias dentro dos dados analisados. Para isso será executada novas "rodadas" de mineração, validação com usuários e testes de modelos utilizando outros algoritmos e ferramentas. Também é necessário estudar a utilização de outros algoritmos de aprendizagem supervisionada além de testar novos conjuntos de atributos para melhorar a \textit{performance} do modelo inicial. A quantidade de dados também precisa ser revista, pesquisas com massas maiores precisam ser feitas a fim entender as similaridades entre as escalas e dados das respostas para poder inferir a EADS de maneira correta, além de rever a base de respostas para igualar os dados e segregar-los melhor tendo em vista a diminuição do \textit{recall} no primeiro modelo. Por fim, utilizar os dados para predizer as dimensões para um determinado grupo dividido por características como sexo, idade, localidade, frequencia de postagens, interação, assuntos chaves entre outros se torna relevante para gerar dados para futuras pesquisas além da área de T.I.

% \addcontentsline{toc}{section}{\protect\numberline{}Considerações Finais}%



% ----------------------------------------------------------
% ELEMENTOS PÓS-TEXTUAIS
% ----------------------------------------------------------
\postextual
% ----------------------------------------------------------

% ----------------------------------------------------------
% Referências bibliográficas
% ----------------------------------------------------------

\bibliography{references}


% Inicia os apêndices
\begin{apendicesenv}
\partapendices % Imprime uma página indicando o início dos apêndices
\chapter{Git e Github}
\label{app:git}
O Git é uma ferramenta de versionamento de código, ou seja, uma ferramenta utilizada para que toda a alteração que alguém fizer em um código tenha a possibilidade de ser marcada, assim criando uma árvore das modificações e dando ao usuário a possibilidade de navegar entre suas alterações. A ferramenta foi criada por Linus Torvalds durante a criação de Linux, para que fosse possível manter um histórico seguro do desenvolvimento do sistema operacional. Entretanto, o Git é unicamente uma ferramenta de terminal, com o desenvolvimento da internet e os recursos gráficos sendo cada vez mais utilizados para representações, juntamente com a necessidade de algo mais acessível, nasceram várias \textit{interfaces} para a ferramenta, uma delas e também a maior, o GitHub ganhou visibilidade e é hoje uma das maiores plataformas de código aberta do mundo e recentemente foi adquirida pela Microsoft\footnote{\url{https://g1.globo.com/economia/tecnologia/noticia/microsoft-compra-github-por-us-75-bilhoes.ghtml}}.

É importante saber sobre Git nos dias de hoje, maiores informações podem ser obtidas em um livro\footnote{\url{https://git-scm.com/book/en/v2}} fornecido pela própria mantenedora da ferramenta. No caso desse trabalho é possível simplesmente baixar o código pelo Github. Para isso, é necessário entender a \textit{interface} do GitHub.

Ao observar a Figura \ref{fig:git_init}, pode-se notar um cabeçalho com alguns menus, logo abaixo o nome da organização getdumont seguida pelo nome do repositório\footnote{Repositório é o nome dado ao local onde ficará armazenado todo o código de um determinado projeto}. Acessando o repositório principal do Dumont\footnote{\url{https://github.com/getdumont/dumont}} pode-se notar 5 abas, que nesse momento são irrelevantes, seguindo encontra-se alguns dados como descrição, algumas estatísticas e por fim o botão verde \textit{Clone or Download}, aqui pode-se baixar o código clicando no botão e em seguida clicando em \textit{Download ZIP}. Além disso, logo abaixo é possível ver uma tabela com a estrutura do projeto, essa estrutura é navegável pelo browser, assim qualquer arquivo citado durante essa documentação pode ser acessado sem necessidade do código na máquina.

\begin{figure}[!ht]
    \centering
    \includegraphics[width=.75\textwidth]{imagens/git_init.png}
    \caption{Imagem demonstrando a interface inicial do git}
    \label{fig:git_init}
\end{figure}

Já na Figura \ref{fig:git_file}, pode-se observar um exemplo do que foi dito anteriormente. Aqui uma demonstração de como fica um dos arquivos do projeto aberto no navegador. As 2 coisas mais importantes a serem notadas aqui é o caminho do arquivo \textit{dumont/collector/index.js} e as linhas que são mostradas no arquivo. Será utilizado destes recursos durante a documentação para exemplificar e apontar códigos.

\begin{figure}[!ht]
    \centering
    \includegraphics[width=.8\textwidth]{imagens/git_file.png}
    \caption{Imagem demonstrando a interface do git referente a um arquivo do projeto}
    \label{fig:git_file}
\end{figure}

\chapter{Docker}
\label{app:docker}

A modernidade e o avanço na programação levaram os desenvolvedores a publicarem suas aplicações cada vez mais rápido e com maior frequência. Com o tempo se criou o termo DevOps, a abreviação para o que em português seria: “Desenvolvimento e Operação”. Usualmente os times de DevOps eram compostos pelos indivíduos responsáveis por operacionalizar toda a parte de publicação e monitoramento das aplicações. As dificuldades encontradas por times era, e até hoje ainda é, a divergência entre tecnologia perante diferentes ambientes, e principalmente, o tempo que a aplicação demora em seu \textit{Deploy} e \textit{Rollback}\footnote{\textit{Deploy} é o termo utilizado para publicar algo na nuvem, enquanto \textit{rollback} é o termo para retroceder uma versão recém publicada}.

O Docker nasceu para suprir não só esse problema, padronizando ambientes, mas também para isolar as dependências e ferramentas instaladas através dele. Uma vez executado o Docker cria o que chamamos de imagem, é basicamente uma versão minimalista do Linux que tem todas as dependências e códigos da sua aplicação. Com essa imagem gerada é possível executa-la e dar origem a um \textit{container}. Podem existir vários containers rodando simultaneamente, e o mais importante, interligados. Com isso subir um banco de dados especifico sem instala-lo em sua maquina, ou testar sua aplicação com outras versões de uma linguagem, se tornou algo simples. O Docker fez tanto sucesso que as principais fornecedoras de aplicação em nuvem como a Amazon, Google e Azure tem serviços dedicados a rodarem a partir da ferramenta.

Para facilitar a execução, foram configuradas todas as imagens necessárias para que a execução seja rápida e direta. Obviamente coordenar uma grande leva de containers seria um problema, para isso, foi criado o Docker Compose, basicamente um arquivo que administra todas as imagens e interliga elas tornando assim mais fácil a conexão entre os containers. Dentro da raiz do projeto existe um \textit{dumont/docker-compose.yml}, com todas as partes da aplicação. Existe também um \textit{dumont/docker-compose.dev.yml} que é o que será utilizado para rodar a pesquisa na máquina local.

Para baixar a ferramenta basta acessar o site\footnote{\url{https://www.docker.com/products/docker-desktop}} e baixa-la para seu sistema operacional. Uma vez com o Docker e o código em mãos é necessário configurar alguns outros elementos. Obviamente antes mesmo de coletar dados é necessário um lugar para guarda-los, como já abordado utilizaremos o MongoDB.


\input{elementos-textuais/apendices-sections/configuracoes}
\chapter{Tabela de Normalização}
\label{app:tabelanorm}

\begin{table}[]
\begin{tabular}{c|c|c}
\textbf{Termos Encontrados} & \textbf{Valor Substituido} & \textbf{Total Encontado} \\ \hline
p, pra & para & 5566 \\
q & que & 3289 \\
n, nn, nao & não & 2453 \\
vc, vx, ce & você & 1646 \\
pq & porque & 1388 \\
eh & é & 893 \\
mt, mto & muito & 803 \\
hj & hoje & 487 \\
cmg & comigo & 318 \\
msm, mm & mesmo & 292 \\
agr & agora & 222 \\
pqp & caramba & 203 \\
d & de & 171 \\
ngm & ninguem & 133 \\
gajo & menino & 103 \\
mds & meu deus & 95 \\
vdd & verdade & 91 \\
aq & aqui & 77 \\
gaja & menina & 71 \\
nois & nós & 54 \\
dnv & denovo & 48 \\
bag, bang, bagulho & coisa & 43 \\
qq & o que que & 41 \\
tamo & estamos & 37 \\
bls, blz & beleza & 31 \\
qria & queria & 26 \\
amg & amigo & 20 \\
bglh & bagulho & 18 \\
nmrl & namoral & 17 \\
nnc & nunca & 16 \\
eae, iai & e ae & 14 \\
men & cara & 18 \\
mrm & gente & 12 \\
vey & cara & 10 \\
dx & deixa & 9 \\
grt & garoto & 7 \\
mrd & merda & 6 \\
bão & bom & 4 \\
daki & daqui & 2 \\
todxs & todos & 2
\end{tabular}
\end{table}



\end{apendicesenv}



% ----------------------------------------------------------
% Anexos
% ----------------------------------------------------------

% ---
% Inicia os anexos 
% ---
\begin{anexosenv}


\partanexos  % Imprime uma página indicando o início dos anexos

\chapter{Escala de Ansiedade, Depressão e Stress}
\label{app:eads}

\begin{table}[h]
\resizebox{15cm}{!} {
    \begin{tabular}{|c|l|c|}
    \hline
    \textbf{Id} & \textbf{Pergunta} & \textbf{Constructo} \\ \hline
    0 & Tive dificuldade de me acalmar & Estresse \\ \hline
    1 & Minha boca ficou seca & Ansiedade \\ \hline
    2 & Não tive nenhum sentimento positivo & Depressão \\ \hline
    3 & \specialcell{Em alguns momentos tive dificuldade de respirar (chiado\\e falta de ar sem esforço físico)} & Ansiedade \\ \hline
    4 & Não consegui ter iniciativa para fazer as coisas & Depressão \\ \hline
    5 & Exagerei intencionalmente ao reagir a situações & Estresse \\ \hline
    6 & Tive tremedeira (por exemplo, nas mãos) & Ansiedade \\ \hline
    7 & Senti que estava sempre nervoso(a) & Estresse \\ \hline
    8 & \specialcell{Me preocupei com situações em que poderia entrar em\\pânico e parecer ridículo(a)} & Ansiedade \\ \hline
    9 & Senti que não tinha vontade de nada & Depressão \\ \hline
    10 & Me senti inquieto(a) & Estresse \\ \hline
    11 & Tive dificuldade de relaxar & Estresse \\ \hline
    12 & Me senti deprimido e sem motivação & Depressão \\ \hline
    13 & \specialcell{Eu não conseguia tolerar as coisas que me impediam de\\continuar a fazer o que estava realizando} & Estresse \\ \hline
    14 & Eu senti que ia entrar em pânico & Ansiedade \\ \hline
    15 & Nada me deixou entusiasmado & Depressão \\ \hline
    16 & Eu senti que era uma pessoa sem valor & Depressão \\ \hline
    17 & Eu senti que estava sendo muito sensível/emotivo & Estresse \\ \hline
    18 & \specialcell{Eu percebi uma mudança nos meus batimentos cardíacos\\embora não estivesse praticando exercício rigoroso\\(ex. batimento cardíaco acelerado ou irregular)} & Ansiedade \\ \hline
    19 & Eu senti medo sem motivo & Ansiedade \\ \hline
    20 & Senti que a vida não tinha sentido & Depressão \\ \hline
    \end{tabular}
}
\end{table}


\end{anexosenv}


\end{document}
