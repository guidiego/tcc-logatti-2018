\documentclass[12pt,oneside,a4paper,chapter=TITLE, english, french,	spanish, brazil]{abntex2-logatti}


\titulo{Caracterização das dimensões afetivas negativas em perfis de redes sociais por meio de Inteligência Artificial.}
\curso{Sistemas de Informação}
\autor{Guilherme Diego Albino Francisco}
\RG{40.741.310-8}
\local{Araraquara - SP}
\data{\MONTH\space / \the\year}
\orientador[Orientadora:]{Cristina Cibeli Vidotti Ivo de Medeiros}
\preambulo{Trabalho de Conclusão de Curso, apresentado às Faculdades Integradas de Araraquara, como requisito parcial para obtenção do título de Bacharel em }

\begin{document}

% Retira espaço extra obsoleto entre as frases.
\frenchspacing 

% ----------------------------------------------------------
% ELEMENTOS PRÉ-TEXTUAIS
% ----------------------------------------------------------
 \pretextual


\imprimircapa

\folhaderosto



\newcommand{\listofgraficosname}{Lista de Gráficos}
\newcommand{\graficoname}{Gráfico}
\newfloat[chapter]{grafico}{logr}{\graficoname}
\newlistof{listofgraficos}{logr}{\listofgraficosname}
\newlistentry{grafico}{logr}{0}

\counterwithout{grafico}{chapter}
\renewcommand{\cftgraficoname}{\graficoname\space}
\renewcommand*{\cftgraficoaftersnum}{\hfill--\hfill}


\pagebreak
\begin{dedicatoria}
 Este trabalho é dedicado a todos que um dia sonharam em melhorar a vida de alguém com grandes ou pequenos atos.
\end{dedicatoria}


\begin{agradecimentos}
Primeiramente gostaria de agradecer aos meus pais, minha irmã e a toda minha família pelo apoio nesses últimos anos. Todas as vezes que tranquei a faculdade ou troquei de curso e mesmo assim sempre acreditarem no meu potencial e me incentivaram. 

Gostaria de agradecer a todos os professores da Logatti. Em especial ao Fabio Fornazzari Papini que além de um excelente coordenador sempre foi um grande amigo e me deu grandes conselhos e orientação nessa caminhada profissional. 

Meu agradecimento mais que especial a minha orientadora Cristina Cibeli Vidotti Ivo de Medeiros, que me introduziu a Inteligência Artificial e sempre acredita em tudo o que me proponho a fazer por mais maluco que seja. 

Por último, porém de longe menos importante a todos os meus amigos, que sempre estiveram do meu lado me levando para frente, gostaria de nomear alguns nesse agradecimento. 

Jéssica Temporal e Leticia Portella me levaram a ter interesse por análise de dados e me ensinaram muito sobre o mesmo. 

João Daher e Mateus Freira que sempre me sedem um pouco do seu tempo para me passar algum conhecimento valioso de inteligência artificial. 

Alex Cortes e Tiago Correa que me ajudaram com toda a parte visual do projeto.

Por fim Jaqueline Alves que me ajudou demais com a parte de psicologia cuja a mesma não tinha conhecimento nenhum, foi graças a suas referências que pude tocar essa ideia.



\end{agradecimentos}


\begin{epigrafe}
“You don't understand anything until you learn it more than one way” – MINSKY,
marvin. Managing an Information Security and Privacy Awareness and Training
Program (2005)
\end{epigrafe}



% resumo em português
\setlength{\absparsep}{18pt} % ajusta o espaçamento dos parágrafos do resumo, espaço entre o resumo e as palavras-chave
% \begin{resumo}
%  Segundo a ABNT, o resumo deve ressaltar o
%  objetivo, o método, os resultados e as conclusões do documento. A ordem e a extensão
%  destes itens dependem do tipo de resumo (informativo ou indicativo) e do
%  tratamento que cada item recebe no documento original. O resumo deve ser
%  precedido da referência do documento, com exceção do resumo inserido no
%  próprio documento. (\ldots) As palavras-chave devem figurar logo abaixo do
%  resumo, antecedidas da expressão Palavras-chave:, separadas entre si por
%  ponto e finalizadas também por ponto.

%  \textbf{Palavras-chave}: latex. abntex. editoração de texto.
% \end{resumo}


% resumo em inglês
% \begin{resumo}[Abstract]
%  \begin{otherlanguage*}{english}
%    This is the english abstract. 

%    \textbf{Key-words}: latex. abntex. text editoration.
%  \end{otherlanguage*}
% \end{resumo}



% ---
% inserir lista de ilustrações
% ---
\pdfbookmark[0]{\listfigurename}{lof}
\listoffigures*
\cleardoublepage
% ---


% Lista de Gráficos
% \pdfbookmark[0]{\listofgraficosname}{log}
% \listofgraficos*
% \cleardoublepage

% ---
% inserir lista de tabelas
% ---
% \pdfbookmark[0]{\listtablename}{lot}
% \listoftables*
% \cleardoublepage
% ---

% Lista de Quadros
% \pdfbookmark[0]{\listofquadrosname}{loq}
% \listofquadros*
% \cleardoublepage


\begin{siglas}
  \item[APPA] Application to Process and Produce Analytic Data
  \item[IA] Inteligência Artificial
  \item[Tweet] Nome utilizado para postagem de um usuário no Twitter.
\end{siglas}


% ---
% inserir lista de símbolos
% ---
% \begin{simbolos}
%   \item[$ \Gamma $] Letra grega Gama
%   \item[$ \Lambda $] Lambda
%   \item[$ \zeta $] Letra grega minúscula zeta
%   \item[$ \in $] Pertence
% \end{simbolos}
% ---


% inserir o sumario
% ----
\pdfbookmark[0]{\contentsname}{toc}
\tableofcontents*
\cleardoublepage
% ---



% ----------------------------------------------------------
% ELEMENTOS TEXTUAIS
% ----------------------------------------------------------
\textual

\chapter*[Introdução]{Introdução}
\chapter*[Introdução]{Introdução}
\addcontentsline{toc}{chapter}{Introdução}

Nos dias atuais a depressão é um assunto cada vez mais recorrente e relevante na sociedade, porém os estudos sobre tal transtorno já é algo antigo. Um dos marcos do estudo da depressão foi quando Freud publicou sua obra \textit{“Mourning and Melancholia”} em 1916. Desde as primeiras citações sobre melancolia e depressão os avanços nas áreas de psicologia e psiquiatria fizeram com que o assunto conquistasse cada vez mais espaço e visibilidade entre as pessoas. Hoje no Brasil temos uma média de 11 milhões de brasileiros afetados por esse transtorno \cite{paho2017-letstalk}.

Analisar um paciente com índices de depressão é uma tarefa cotidiana para um psicólogo ou psiquiatra. A forma com que as pessoas utilizam recursos como fala e escrita deixam traços que podem ser utilizados para analisar padrões emocionais e comportamentais. Na escrita a construção de uma frase já é o suficiente para dizer muito sobre a pessoa. Existem vários estudos que analisam discursos em prol de achar padrões emocionais. A escrita esta presente todos os dias em nosso cotidiano, principalmente depois da ascensão das redes sociais quando o jeito que nos expressamos publicamente se tornou mais simples.

Graças aos compartilhamentos em redes sociais, a quantidade de conteúdo disponível vem apenas aumentando, basta um rápido acesso a um perfil para mapearmos informações pessoais, sejam elas exatas como nome, idade, endereço e até familiares ou conteúdos mais abstratas como gostos musicais e literários. Todo esse ápice tecnológico que começou a aproximadamente quatro anos atrás e deu inicio a era dos dados. Ja que nas últimas duas décadas a facilidade fornecida pelo computador fez com que o ser humano procura-se cada vez mais praticidade e informação, o dado que é capaz de ser organizado a fim de se transformar em informação relevante se tornou algo ainda mais valioso. Empresas como Facebook, Google, Uber e Airbnb ganharam o mercado exatamente por conseguirem coletar os dados importantes e administrá-los em forma de soluções para seus usuários.

Obviamente outras pessoas se aproveitaram desse momento e começaram a utilizar de dados públicos para fazer ferramentas voltadas para analise de mercado e tomada de decisão. Isso se deve a facilidade em mapear visitações, curtidas, compartilhamentos e até publicações opinando sobre algum produto ou serviço. Aplicar a mesma lógica para analisar um perfil e tentar descobrir se a pessoa tem algum índice de depressão é um pouco mais complexo, porem factivel.

A resposta mais lógica para mapear perfis com depressão é chamar especialistas para analisar os perfis e gerar então os dados desejados. O maior problema, que se repete em quase todo processo que é automatizado, é o fator humano. Para analisarmos uma quantidade de dados satisfatória seria necessário um numero de profissionais muito grande, sem contar a logística para registrar esses dados e mante-los consistentes.

Por outro lado, a maquina é lógica. Nesse caso analise de dados abstratos como textos, só seriam possíveis com aprendizado de máquina. Ainda assim existiria outro problema que é o fator da saúde mental. Analisar dados quando o contexto tem pouca variação como em “gostei de x” ou “odiei y”, onde a maior barreira poderia ser o sarcasmo, é bem diferente de analisar uma frase como “Queria estar morta”, onde além do sarcasmo existe a chance da mesma representar um estado emocional ou simplesmente um meme\footnote{Uma imagem, video, ou frase, normalmente de humor, transmitida por usuários na internet. Um elemento cultural transferido por meios não genéticos}.

A proposta do trabalho é utilizar de dados públicos do Twitter para coletar perfis e suas respectivas postagens recentes. Em seguida, será necessário transformar essa base dados em uma base de conhecimento inserindo informações como indice de depressão e palavras chaves, para que isso seja possivel, será implementado um sistema capaz de recolher as analises de psicólogos e psiquiatras e através de funções auxiliares gerar o indicadores falados anteriormente. Para maior veracidade será solicitado ajuda de diversos profissionais, de diversos niveis, da area de psicologia e psiquiatria para que nossa base seja conscistente e capaz de gerar dados reais e relevantes. Por último, sera utilizado essa base de conhecimento para treinar uma rede neural afim de que ela seja capaz de replicar as analises dos profissionais, fazendo assim, com que a maquina seja capaz de identificar índices de depressão autonomamente.

Para a pesquisa, foi selecionado o paradigma funcional utilizada nas linguagens como Javascript, Python e Golang para construção do softwares e \textit{scripts} responsáveis pelo funcionamento do sistema. Além disso o banco utilizado será o MongoDB para armazenar nosso conjunto de dados pela alta performance em leitura e indexação de dados.

\cleardoublepage
\chapter{SÍNTESE BIBLIOGRÁFICA}
\chapter{Revisão Bibliográfica}
A pesquisa proposta, assim como os temas abordados nela, é multidisciplinar. As próximas sessões abordarão tópicos necessários para o entendimento da posterior conclusão do projeto e suas abordagens.

Primeiramente será introduzido a \textbf{linguística} devido aos temas de analise de discurso, essencial para que seja possível analisar como as amostras se expressão em seus textos, e a introdução ao processamento de linguagem natural.

Em seguida, será necessário entender um pouco sobre \textbf{Psicologia}, dentro dela os tópicos traumas emocionais e modelos de mensuração que serão brevemente abordados devido ao cunho dessa pesquisa. Além disso, será introduzido ao conceito de Psicologia Cognitiva.

Os tópicos já explicados serão utilizados para fundamentarmos a \textbf{inteligência artificial}, nessa sessão tambem será apresentado o conceito base de IA com ênfase em aprendizado de máquina.

Por fim, será tratado o tema \textit{\textbf{Data Science}}, os tópicos relacionados a manipulação de dados, desde a mineração e gestão até a sua representação e organização.

\section{Linguística}
A gramática é composta por: um conjunto finito de letras que formam o chamado alfabeto e um conjunto de regras e normas. Utilizando das regras e do finito número de palavras formadas a partir das letras, é possível se expressar através de uma sentença. O estado representado por essa sentença pode variar de acordo com a regra aplicada. É impossível cobrir todos os estados com uma única regra pelo motivo de existirem números infinitos de sentenças a serem formadas \cite[13-25]{chomsky2002syntactic}. Essa capacidade de obter descrições de forma simplificada através da linguagem é primeira área cognitiva do ser humano \cite[131]{putnam1975mind}.

Na inteligência artificial, o ato de juntar símbolos (padrões físicos) em expressões (estruturas) utilizando um conjunto de regras (processos), é considerado um sistema de símbolos físicos. Acredita-se que um sistema desse nesse formato possui os meios necessários e suficientes para realizar ações inteligentes de forma geral \cite[116]{newell1976ComputerSA}. Entretanto, o que foi escrito é diferente do que é compreendido, vide duplo sentidos, logo o contexto é necessário. No dia-a-dia, existem multiplos fatores que ajudam a entender o sentido de uma frase, porem, para em uma máquina os mesmos fatores muitas vezes não se aplicam. Nessa sessão nosso enfoque é em introduzir alguns estudos fomentados pela linguística.

% sections
\subsection{Analise de Discurso}
Ao decorrer de um texto (que é algo concreto), pode-se caracterizar diversos níveis de geração de sentido. O fundamental, responsável pela primeira formulação de sentido a partir do discernimento de termos dentro de um contexto. O narrativo, onde o autor utiliza dos valores fundamentais através de um sujeito tomando a direção da prosa. E por fim, o discursivo, relacionado as escolhas de tempo, espaço, pessoa e figura durante a narrativa dos fundamentos, dando a essa narrativa um ponto de vista. Logo, o termo discurso é dado como um suporte abstrato por trás do texto, afim da concretização da sua ideia central \cite[13-17]{gregolin1995ad}.

A analise de discurso é, de forma sucinta, uma analise do que foi dito, de como foi dito e qual o sentido do que foi dito. As primeiras manifestações do assunto foram no século XX com autores russos que, além de isolar e definir elementos de uma linguagem poética queriam definir determinantes por trás do perfil artístico do escritor. O tempo fez com que a analise de discurso se se desenvolve e se ramificasse em varias vertentes, uma delas a francesa que apoia a possibilidade de automatizar essa analise por meio da informática. A área continua sendo um campo complexo e de continuo estudo por trás das definições e metodologias para abordar e sustentar as novas unidades de analise. \cite[22]{souza2006ad}.

Os discursos se diferem de pessoa para pessoa devido ao nível discursivo, a necessidade de expressar um determinado sentido leva o autor a se colocar em um ponto de vista durante sua narrativa. Do contexto da pesquisa, entender o discurso do usuário para mapear o motivo do seu estado mental é um fator de total relevância para entender o estado dele. A pesquisa realizada por Modesto Leite \cite[134]{modesto2005adepre}, mostra em seus resultados que os discursos apresentados pelos pacientes fundamentavam o motivo psicológico do por que os mesmo teriam o transtorno. Partindo dos principio apresentados sobre um discurso, por mais que as palavras sejam localizadas, o como um computador seria capaz de inferir o sentido da frase se torna o ponto chave em discussão.

\subsection{Processamento de Linguagem Natural}
Partindo do principio de um dicurso, por mais que as palavras sejam localizadas, como inferir o sentido da frase? Mesmo parecendo óbvio, entender palavras não significa entender o contexto, é necessário se familiarizar com o ambiente e o momento afim de idealizar o que está sendo transmitido. Essa conexão entre elementos é tratada no estudo do \textit{connectivism}\footnote{Integração dos princípios de rede, caos, complexidade e teorias de auto-organização. Seu objetivo é entender decisões baseado nas mudanças de componentes fundamentais \cite{siemens2014connectivism}.}. De acordo com a linha de pensamento, estabelecida pelo estudo, os neurônios seriam os agentes cognitivos responsáveis por planejar, construir e representar essas informações que nosso cérebro recebe. Criar soluções para para problemas pontuais que envolvam a ligua que é utilizado no dia-a-dia se uma pessoa, essa é a definição por trás do \textbf{Processamento de Linguagem Natural} \cite{brandura1996, maria2015npl}.

\begin{quote}
Além disso, esse é o ponto principal da contribuição da Linguística para o PLN, qual seja, o de fornecer dados linguísticos que a máquina não é capaz de inferir, mas pode, em parte, processar, melhorando o seu desempenho.
\end{quote}


\section{Psicologia}

A psicologia é, descrita como, a ciência da vida mental, capaz de analisando os desejos, sentimentos, razões, sentimentos, decisões entre outras faculdades mentais entender o posicionamento e o estado emocional do ser. Entender o nosso estado e como isso impacta em a vida é o grande desafio da área \cite[4-8]{william1890principles}.

Nossa pesquisa utiliza da psicologia em dois pontos distintos, porem, interligados. A primeira delas é o envolvimento da psicologia com a depressão, em segundo a participação da área da psicologia cognitiva na evolução da Inteligencia Artificial. Ambos os pontos se interligam ao se questionar o motivos de alguem ter depressão, ou o quão plausivel é o mapeamento disso através de técnicas desenvolvidas dentro da área nos ultimos anos.

% sections
\subsection{Depressão}
Desanimo, perca de interesse, inibição e bloqueio de sentimentos são alguns sintomas exibidos por pessoas melancólicas \cite[276]{freud}. A \textbf{melancolia} seria uma condição maléfica de enfraquecimento da sáude mental de um ser. Partindo do principio de Fairbain onde o ser humano busca por gratificação, a não gratificação poderia ser o motivo de um estado melancólico.

A \textbf{depressão} é uma forma atenuada de melancolia \cite{roudinesco2000}. Classificada como \textbf{transtorno de humor}, diferente de outras variações mais regulares de humor, pode causar grandes danos a vida cotidiana uma vez que, por definição, altera a percepção de si mesmo maximizando o peso dos seus problemas diante de sua própria pespectiva. Por tais motivos, a melancolia e a depressão compartilham de sintomas similares, entretando, a dinamica de suas origens, relações e concepções podem criar diversas perspectivas o que leva ao ponto de como se pode medir algo tão abstrado. \cite{}

\input{elementos-textuais/r-bibliografica-sections/psicologia/ansiedade}
\subsection{Escala Depressão, Ansiedade e Stress}
Adotando um modelo dividido em 3 sub-escalas a \textbf{Escala de Depressão, Ansiedade e Stress} ou \textbf{EADS} foi proposta na premissa de uma maior assertividade na analise das dimensões afetivas negativas, uma vez que existiam outras pesquisas com proposta similar como o \textit{ Beck Anxiety Inventory} (BAI) e \textit{Beck Depression Inventory} (BDI). As diferenças entre essas escalas são: sua execução, o fator de correlação das sub-escalas propostas pelo EADS e o inventário de Stress introduzido durante o estudo (como mencionado na sessão anterior).

A EADS completa é composta de 42 itens, por fins de facilitar a geração de dados iremos usar o modelo de 21 itens, que são divididos igualmente entre as escalas. Mesmo que um dos itens pertença a uma escala ele pode ter correlação com alguma outra. Esses itens são afirmações que podem ser respondidas por números de 1 a 4 e representam desde "não se aplica a mim" até "se aplicou a mim na maior parte das vezes". Representam as dimensões mais negativas os maiores valores gerados pela soma dos itens de cada subcategoria \cite{lovibond1995structure, ribeiro2004contribuiccao}.

Diferente da proposta de auto avaliação ou avaliação mediada\footnote{Nesses casos os usuários são responsáveis por responder as perguntas por si mesmos ou por meio de um mediador que ira preencher o formulário.}, como é proposta pela EADS,  uma das premissas da pesquisa é inferir os resultados das escalas utilizando textos cotidianos de uma amostra em rede social. Isso nos leva a entender os conceitos cognitivos por trás da psicologia que levaram os autores a propor suas escalas e pesquisas.



\subsection{Psicologia Cognitiva}
A psicologia é, descrita como, a ciência da saúde mental, capaz de analisar os desejos, sentimentos, razões, decisões dentre outras faculdades mentais afim de entender o posicionamento e estado emocional do ser \cite[4-6]{william1890principles}. Os estudos que visavam entender dos animais e humanos a capacidade de pensar, memorizar, perceber e no caso humano utilizar um dialeto (linguagem), foi uma área que surgiu antes mesmo da aparição das pesquisas sobre Inteligencia Artifical e sua vontade de entender os mesmos préceitos. Essa área da psicologia ficou conhecida como \textbf{Psicologia Cognitiva}.

O poder humano de realizar mutiplas tarefas, o questionamento de um cadeia de eventos físicos cancelar ou não uma outra e a recente popularidade e divergência da língua, fez com que em 1958 surgisse o primeiro modelo psicológico \cite[4-7]{broadbent1958perception}.

Os estudos da psicologia cognitiva gerou grandes avanços para área interdiciplinar das \textbf{ciencias cognitivas}. Novas abordagens como a junção de modelos criados em computação para IA e técnicas experiementais da psicologia foram criadas a fim do entendimento da mente humana. Durante a criação do \textit{General Problem Solver}, por exemplo, Newell e Simon propuseram não só a implementação de algoritimos descritos por antigos estudiosos afim de resolver um leque de problemas, mas tambem, entender como a maquina estava realizando isso, assim, seria possivel uma comparação com a mente humana na premissa de aprimorar conceitos ja existentes dentro da ciencia \cite[3-5]{newell1961gps, russell2003artificial}.

Nessa pesquisa, a psicologia cognitiva tera uma grande impacto em entender as \textit{features} da IA, uma vez que, sera necessário entender quais parametros serão analisados em pró de demonstrar a compatibilidade entre dois textos. Porem, alem disso, entender o conceito da psicologia nos ajuda a entender um dos vários fundamentos da IA que serão apresentados na próxima sessão.
\section{Inteligencia Artificial}

Inteligência é, por definição, uma coleção sistemática de habilidades e funções com objetivo de processar diferentes tipos de informações de diversas maneiras \cite[49]{guilford1982cognitive}. Simular essa "coleção sistemática de habilidades e funções", reconhecer formatos, trajetórias, sinais, deduzir proposições entre outras, criando entidades inteligentes é o foco da inteligência artificial.

Do momento em que a IA passou a ser um campo de estudo até o momento atual, a quantidade de abordagens apresentadas por diversos pesquisadores afim de construir uma máquina inteligente apenas aumentou. Esses métodos se auto-contribuíram propondo novas visões e gerando novas abordagens a partir de descobertas dentro das suas linhas de pensamento. Isso levou a IA para seu estado atual. \cite[1-2]{russell2003artificial}.

Ao decorrer da sessão serão apresentados os fundamentos visando introduzir as demais áreas que colaborarão com a computação para a criação da inteligência artificial, seguido pelas definições de agentes racionais.

\section{Fundamentos da Inteligencia Artificial}

Psicologia, Linguística, Economia, Filosofia e Matemática, esses são alguns dos fundamentos abordados até então. É possível entender o conceito por trás da inteligencia, porem, para que a IA seja bem sucedida e necessário um artefato, e esse artefato tem sido o computador. A engenharia da computação é um outro fundamento, que até agora foi tratado implicitamente, seu foco é construir maquinas eficientes, sendo assim, além dos hardwares, novos sistemas operacionais, linguagens e ferramentas surgiram fortalecendo o campo de IA \cite[13-14]{russell2003artificial}.


Nas últimas sessões foi abordados os temas \textbf{linguistica} e \textbf{psicologia}, e claramente é possivel vislumbrar sua relevancia dentro da área de Inteligencia Artifical. O motivo dessa relevancia é que ambas são consideradas fundamentos da IA assim como outras áreas que serão abordadas aqui.

% Filosofia e intro a agentes
\textit{Law of thought} (lei do pensamento) é uma lei psicológica que acompanha um processo mental, logo, necessita estar de acordo com uma razão ou lógica \cite[289-291]{frege1956thought}. Para as pessoas a lógica parece ser algo extremamente banal, porem, a teoria por trás do que é logico e o que é racional é algo bem extenso. Desde a Grécia antiga filósofos e/ou matemáticos estudam o conceito de lógica. \cite[4-5]{russell2003artificial}. Durante todo esse período varias escolas nos proporcionaram diversas visões sobre o assunto. O racionalismo por exemplo, visa que podemos adquirir conhecimento independente da nossa experiencia sensorial \cite{rationalismvsempiricism}. Dentro dessa ideia nasceram o dualismo, onde afirmamos que nossa mente é algo natural e sem conhecimento do mundo externo \cite[7]{descartes2013rene}, e o materialismo, que contrariando o dualismo afirma que a mente é formada pelas operações do nosso cérebro \cite[6]{russell2003artificial}.
Em contradição com o racionalismo, existe o empirismo, onde a experiencia sensorial é a fonte final de conhecimento \cite{rationalismvsempiricism}.

% Economia
Para que se torne possível realizar ações racionais é necessário entender o ambiente aonde você esta e suas variáveis \cite[99]{simon1955behavioral}. Diferente do que muitos pensam o estudo da economia se trata de dinheiro porem se trata de como guiamos nossas decisões baseadas nos retornos esperados. Os estudos da economia dentro da IA ainda se aplicam a questões como como devemos agir quando o que esperamos esta em um futuro distante ou então se devemos continuar quando outros fatores não continuarem a nos favorecer \cite[9]{russell2003artificial}. A ação racional descrita é feita por um agente, que é descrito como autônomo e racional, uma vez que não necessita diretamente de um humano para agir e é construído com propósito de retirar a melhor performance com base em seu objetivo \cite[2]{ wooldridge1994agent}.

% Matematica
Esse agente esta dentro de um ambiente, é necessário que esteja claro o que pode ou não ser computado, quais são as regras que podemos aplicar e principalmente como conseguimos obter algo racional em caso informações incertas \cite[7]{russell2003artificial}. Os princípios matemáticos aplicados aqui partem das pequisas em cima das proposições, para que seja possível deduzir uma proposição não é necessário apenas inferência, em alguns casos, é necessário que esses dados sejam convertidos a fatores numéricos, afim de utilizar funções aritméticas para resolver os problemas \cite[2-4]{boole1854investigation}. Além disso, existe a probabilidade, em pouco tempo sua capacidade de resolver teoremas inacabados e ajudar a mensurar problemas a tornou indispensável para a IA \cite[9]{russell2003artificial}.

A cibernética e a teoria de controle é um outro fundamento da IA, o conceito por trás desse fundamento foi criado por em 1961 por Wiener \cite[15]{russell2003artificial}. Muitas vezes é necessário que a maquina tenha conhecimento de fatores externos, que usualmente não pode ser transmitido pela ausência de transmissores de informação humanos, como olhos, ouvidos e boca. Se esses estímulos são tão relevantes seria necessário que de forma elétrica fossem criados sensores capazes de controlar essas informações e integrar com os softwares de maneira que passassem dados numéricos e não exatamente seus sentimentos em relação a algo \cite[3-7]{wiener1961cybernetics}. Diferente dos demais, esse fundamento não impacta nossa pesquisa, logo, não será aprofundado como os demais.

A neurociência, é responsável por estudar o nosso cérebro, ou sendo mais exato, como nossas redes neurais funcionam, e é o ultimo fundamento abordado. \cite[10]{russell2003artificial}. A biologia em si era em grande parte descritiva, propor um modelo matemático para o funcionamento de um cérebro vislumbraria em entender o metabolismo como centro de transmissão e calcular a forças emitidas durante esse processo, para que então, fosse transformar a ação ocorrida em um cérebro humano em uma função aritmética \cite[1-3]{rashevsky1960mathematical}. Criar entidades mais inteligentes necessitária de avanços em campos externos ao de computação, como a biologia, para entender se é possível maximizar a inteligencia humana, outras definições como até onde entidades ficariam mais inteligentes, como implementar interfaces para suportar algo talvez superior a inteligencia humana ou então aumentar a capacidade de nossos computadores para que eles fossem capazes de suprir o potencial de um cérebro dependeriam em grande parte de avanços de hardware \cite[1-2]{vinge1993coming}. Mesmo com tantas discussões em torno do assunto, a proposta de singularidade de Vinge não tem uma comparação informativa, e mesmo que fossemos capaz de ter memória e processamento infinito ainda não é possível entender como armazenar e replicar os padrões encontrados na neurociência \cite[11-12]{russell2003artificial}. Os estudos cognitivos tornaram possível entender melhor o funcionamento da mente humana; o método racional e a proposta de pensar nos meios que nos levam a um fim fez com que fosse capaz de abstrair modelos inteligentes através de agentes; a neurociência, por sua vez, nos deu a magnitude de como transmitir conhecimento e aprimorar os modelos, o fator inteligencia e aprendizado passaram a ser mais vistos dentro do ramo de IA, e nos levaram para ascensão da aprendizagem de máquina.
\subsection{Agentes Racionais}
%Tranformar em intro
Mas seria possível uma maquina pensar como nós e entender as mesmas descrições? Esse foi o questionamento inicial de Turing, a partir disso ele propôs o \textit{Imitation Game},que anos após se tornaria o famoso Teste de Turing. Além disso, algumas diretrizes para construção da maquina capaz de passar em seu teste e objeções a suas próprias afirmações foram dadas. Duas dessas objeções seriam o tamanho finito do armazenamento das maquinas seguido da explicação que apenas reconhecer frases, buscar a informação necessária e levá-la ao usuário não seria o suficiente para passar no teste, as maquinas teriam que ser capazes de guardar instruções e se aprimorar do mesmo modo que um ser humano, assim, seria possível conseguir se adaptar a novas situações \cite[144-155]{turing1950}. Logo, questionar se uma maquina pode pensar é o mesmo que questionar se um submarino pode nadar, essa frase foi utilizada por Dijsktra \cite{dijkstra898} ao explicar que um submarino nunca realizaria o ato de nadar, porem, continuaria executando seu objetivo e propósito da melhor maneria possível.

Quando foi proposto à máquina entender um conjunto de texto e responder corretamente a isso \cite[146]{turing1950}, entramos indiretamente no campo da linguística.

Logo para passar no Teste de Turing, seria necessário criarmos uma maquina capaz de aprender e se adaptar a diversas situações usando Processamento de Linguagem Natural e Referencias de Conhecimento para entender o que foi requisitado além de um algorítimo capaz de automaticamente tomar decisões baseados nos dados fornecidos. \cite[2]{russell2003artificial}. Todas essas etapas podem ser/ou não solucionadas com agentes, que serão explicados a seguir. Durante essa mesma época, em 1956, foi proposto que a quantidade de informações a serem referenciadas seria uma das maiores dificuldades em entender as relações entre as variantes \cite[81-82]{miller1956magical}, de acordo com Russel \cite[13]{russell2003artificial} o inicio do campo da Ciência Cognitiva ocorreu no \textit{workshop} onde esse artigo foi publicado. Essa parceria entre as áreas de computação e psicologia trouxeram grandes avanços ao juntar modelos computacionais com teorias experimentais de psicologia, essa abordagem causou divergências entre estudiosos que discutiam se esse modelo ou um algorítimo performariam melhor na execução de uma tarefa. \cite[3]{russell2003artificial}. O campo da Ciência Cognitiva,  que tinha vasta abrangência como é possível vislumbrar em \textit{“The MIT encyclopedia of the cognitive science”} \cite{wilson2001encyclopedia} que reuniu seis diferentes áreas para abordar o tema, ganhou tração.

%-------[]
Em resumo o agente recebera um \textit{input}\footnote{Entrada de dados} e será responsável por gerar um \textit{output}\footnote{Dado de Saída}, ao longo do tempo o mesmo agente gerara múltiplas percepções e essas formarão uma sequencia de percepções\footnote{Não serão todos os modelos que seguirão a proposta sequencial, existem casos em que a linha temporal não afeta o desenvolvimento da decisões tornando-as episódicas}. \cite[34-35]{russell2003artificial} Existem definições dadas aos agentes racionais, iremos definir algumas delas a seguir \cite[42-45]{russell2003artificial}:

\begin{itemize}
 \item \textbf{Totalmente, parcialmente ou não observador:} essa definição é gerada pela quantidade de fatores do ambiente que seu agente recebe, um agente que tem todas as informações do ambiente é totalmente observador enquanto um que não recebe nada, precisando assim manter alguns estados, é não observador.
 \item \textbf{Estocástico ou Determinístico:} quando é impossível determinar o próximo estado através do anterior o agente é Estocástico, caso ao contrario ele é Determinístico.
 \item \textbf{Episódicos ou Sequenciais:} já foi dito que em diversas abordagens são gerados sequencias de percepção, quando essa sequencia é alterado a partir de alguma mudança de estado chamamos o agente de sequencial, caso ao contrario o agente é Episódico.
 \item \textbf{Estáticos ou Dinâmicos} essa definição é referente ao ambiente, quando nosso ambiente não infere alterações chamamos o agente de estático, caso ao contrario Dinâmico.
 \item \textbf{Continuo ou Distinto:} Quando existem finitas possibilidades de estado pode se afirmar que o agente é Distinto, quando as possibilidades são infinitas é dado o nome de Continuo.
 \item \textbf{Conhecido ou Desconhecido:} Quando o agente necessita aprender algo e não consegue realizar a ação por si só ele é um agente desconhecido, caso ao contrário ele é conhecido.
\end{itemize}

Para que seja possível definir se o agente está ou não gerando os dados esperados é necessário medir sua performance, então, é necessário que analisar o ambiente gerado a partir das percepções e conferir se os dados são os esperados ou não \cite[294-295]{frege1956thought}.

Racional é algo baseado ou acordado com uma razão ou lógica\footnote{Oxford Dictionarie:  https://en.oxforddictionaries.com/definition/rational}, existem dois pilares para a lógica: a Conversão responsável por expressar a mesma proposição em diferentes formas e o Silogismo responsável por localizar um termo em comum que conecte duas dessas proposições. \cite[175]{boole1854investigation}.

\section{Ciência por trás dos dados}
Como um ser humano, os dados têm seu ciclo de vida, esse ciclo pode alterar-se dependendo de sua finalidade dentro da aplicação. O tema e o objetivo desta pesquisa é coberto por uma matéria, já antiga, nomeada KDD ou \textit{Knowledge Discovery in Database}, como o próprio nome traduzido sugere, a Descoberta de Conhecimento em Base de Dados abrange varias subclasses como mineração de dados, analise de dados, probabilidade e até mesmo a própria IA. A KDD se justifica pela sua metodologia de manipulação de dados afim de extrair o conhecimento almejado de um base. Se observar o fluxo descrito na figura \ref{fig:kdd}, pode-se notar que uma cadeia de eventos como: seleção, pré-processamento, mutação, mineração e interpretação. Esses eventos geram novos estados para o dado, e os moldam focando o objetivo de retirar o conhecimento pela interpretação no final do processo.

\begin{figure}[H]
    \centering
    \includegraphics[width=.8\textwidth]{imagens/kdd.png}
    \caption{Representação da metodologia defendida pela KDD. Fonte: o autor.}
    \label{fig:kdd}
\end{figure}

Com o passar do tempo, várias abordagens surgiram para tratar dados e essa massa de pesquisas e definições deu-se por bagunçar em partes as nomenclaturas e definições que estão atrelados a ciência por trás dos dados. Para evitar confusões as definições apresentadas nessa sessão serão sucintas e tem como objetivo futuros tópicos e discussões apresentados durante os resultados dessa pesquisa. Devido a grande divergência, ocorrida pela disseminação rápida de algumas termologias, os termos abordados serão guiados pelas seguintes referencias \cite{laender2002brief, fayyad1996kdd, hand2007principles}.

\subsection{Seleção de Dados}
O ato de selecionar os dados com enfoque no conhecimento que se deseja obter deles afim de gerar um conjunto de dados especifico é também conhecido como \textit{data collection}, durante o processo do KDD é o primeiro passo responsável por gerar um amostra mais focada no problema.

\subsection{Mineração de Dados}
Nessa subseção propõe-se que seja dissertado sobre os processos do KDD a partir da seleção até a interpretação. Vale ressaltar que para obter conhecimento não é necessário uma IA, sistemas de tomada de decisão trabalham com probabilidade matemática sob dados exatos, o que em muitos casos, já seria o suficiente para obter a informação do dado. Entretanto, o foco da pesquisa se baseia na implementação de um sistema inteligente e isso inclina essa explicação para o fato de: o aprendizado de máquina é uma das possíveis formas de se minerar um dado.

A mineração de dados ou popularmente conhecida como \textit{data mining}, baseia-se em retirar os valores mais relevantes e valiosos para se inferir um conhecimento. Os demais passos como pré-processamento e mutação se assemelham a alguns processos citados anteriormente como a redução dimensional ou ainda a engenharia de atributos.


\subsection{Análise de Dados}
E finalmente, a análise. O passo descrito como interpretação no KDD se refere a entender os dados de saída vindos da mineração afim de afirmar ou descarta hipóteses, com isso é possível refinar o processo e explorar novas possibilidades. Uma das palavras que hoje se tornaram popular dentro dá área é o \textit{data storytelling}, ou, o ato do dado contar uma história. Abordar a hipótese de maneira empírica e demonstrar a veracidade dela discutindo abordagem e algoritmos gerados é um dos objetivos dessa área.




\cleardoublepage
\chapter{MATERIAIS E MÉTODOS}
\chapter{Documentação}
O objetivo da pesquisa é localizar pontos comuns entre usúarios portadores de afetividades negativas (stress, ansiedade e depressão) afim de caracterizar tais dimensões nesses perfis. Esse tipo de caracterização se asemelha muito com uma aprendizagem não supervisionada, logo, antes mesmo dessa caracterização é necessário mapearmos atributos que serão usados para isso fazendo-se necessário a identificação de usúarios com dimensões afetivas negativas primeiro.

Como observado, existem vários passos para conclusão desse projeto, nossa estrutura de processamento contára com um processo de mineração e dois processos de inteligencia artificial afim de gerar dois modelos lógicos. O primeiro modelo responsavel por inferir valores da EADS em um perfil, e o segundo, de predizer a partir de dados do perfil a chance de ter algum nivel de afetividade especifico.

O projeto por tras dessa pesquisa tem alguns outros pontos sociais envolvidos, que assim, tornam a área de atuação da pesquisa um pouco mais ampla, entretando nenhum desses pontos, exceto a utilização de profissionais capacidados para geração de alguns atributos em nossa base, impactam diretamentamente a \textit{performance} e por isso não serão detalhados. Logo essa documentação é voltada unicamente ao sistema que desenvolveremos.

Pode-se observar na figura 6, o sistema é divido em dois núcleos, o Dumont responsável por minerar e gerar toda nossa base de dados, e o 14BIS que será responsavel pelas Inteligencias Artificias. O aparente terceiro núcleo, na realidade, é simplesmente um banco de dados integrado gerado pelo APPA dentro do Dumont e que será consumido pelo 14BIS posteriormente.

\begin{figure}[H]
    \centering
    \includegraphics[width=.8\textwidth]{imagens/tecnologias.png}
    \caption{Desenho apresentando os núcleos do projeto}
    \label{fig:tecnologias}
\end{figure}

Existem basicamente 5 técnologias em evidencia:
\begin{itemize}
 \item Rust: Rust é uma linguagem altamente performatica, preza por custo zero em abstração, atua com excelencia em processamento paralelo e seu modelo de alocação de memória evita \textit{dataraces}\footnote{}
 \item Python: É a linguagem atual mais utilizada no mundo de Aprendizado de Máquina, sua simplicidade ja á torna simples de usar, porem, a quantidade de materiais, bibliotecas e artigos sobre o assunto á tornam nossa principal linguagem nesse projeto.
 \item Node/Javascript: Node é o interpretador que permite com que rodemos o Javascript (linguagem originalmente de browser no servidor). A linguagem tem um grande ganho com integrações e será utilizada para consumir recursos vindos de APIs.
 \item Go: Go se asemelha muito com Rust, porem, tem um ambiente menos burocratico devido a seus diferentes objetivos. Go será nossa escolha para qualquer processo que necessite muito recurso de processamento devido sua atual exploração dentro da area de IA.
 \item RocksDB: O RocksDB é um banco chave valor criado e distribuido gratuitamente pelo facebook. É um banco altamente performatico para leitura e escrita.
 \item Docker: Docker é uma ferramenta para infra-estrutura, será utilizado para rodar nossa aplicação em containers e facilitar nosso \textit{deploy}\footnote{}.
\end{itemize}

Existira uma sessão explicando o funcionamento do APPA, e como serão desenvolvidos os scripts que ele ira gerenciar, tanto quanto a parte explicativa sobre nossa Inteligencia Artificial, logo, nessa introdução os casos de uso serão tratados de forma sucinta. Se observar a figura 7, notara que o Dumont ira utilizar da API do twitter para coletar dados públicos, posteriormente esses dados serão processados e mutacionados a fim de gerar uma base de dados, por final essa base dados será salva em um banco embutido. Tanto o Dumont quanto o 14BIS irão consumir os dados salvos, porem é responsabilidade do 14BIS consumi-los a fim de treinar nossa IA para gerar nossos modelos lógicos.

\begin{figure}
    \centering
    \includegraphics[width=.8\textwidth]{imagens/tcc_caso_de_uso.png}
    \caption{Diagrama de caso de uso do sistema}
    \label{fig:tcc_caso_de_uso}
\end{figure}

Para melhor entendimento a documentação será dividida em núcleos teóricos e práticos. Primeiramente será dado toda a descrição teórica sobre a escolha de atributos, e funcionamento das ferramentas que serão utilizadas no processo. Em seguida entratemos em núcleos mais práticos como a coleta e mineração, onde alem de explicar como rodar o Dumont, tambem serão explicados como e por que foram construidas essas partes do sistema. Como ja dito, a primeira etapa se consiste em entender nossos atributos.

\section{Engenharia de Atributos}
A escolha dos atributos, também intitulada popularmente como \textit{feature engineering}, é o ato mais importante durante a mineração de dados, por que é através desses atributos que as maquinas irão aprender. É válido destacar que essa sessão serve apenas para introduzir a razão dos atributos, seu detalhamento será dado durante a sua implementação.

O primeiro atributo relevante aqui é o sentimento. Já que será abordado dimensões afetivas negativas, o sentimento expressado por uma frase tem um grande impacto como atributo. Entretanto, o sentido em uma frase pode ser mais fácil de ser extraído em textos concisos, ou seja, normalizar os textos é necessário.

Um dos atributos utilizados aqui será o texto normalizado, para isso será utilizado o \textit{spacy}, uma biblioteca Python para remover palavras que oferecem apenas ruídos ao resultado. Além disso, também será tirada a arvore sintática, para que seja possível estabelecer padrões de discurso na IA, ou então, reconhecer certas palavras presentes em demais analises.

Uma vez observado os nossos atributos, seria necessário erguer um sistema capaz de realizar tarefas e persistir esses dados em algum banco de dados. Porém, será utilizado nessa pesquisa uma ferramenta para gerenciar tais tarefas.

\section{\textit{Application to Process and Produce Analytic Data (APPA)}}

Definir uma boa base de dados é o ponto mais critico durante a criação de uma inteligência supervisionada. O \textit{Application to Process and Produce Analytic Data} (APPA), ou, Aplicação para Processamento e Produção de Dados Analíticos, é um ferramenta criada pelo autor dessa pesquisa capaz de cadastrar tarefas, em qualquer linguagem e utilizar delas para coletar e processar esses dados afim de gerar uma base de conhecimento.

Explicando com mais detalhes, a figura \ref{fig:appa_eng} mostra o funcionamento da ferramenta. Existe uma arquivo chamado \textit{config.yml} que tem mapeado todas as tarefas e entidades de processamento. Essas tarefas podem ser escritas em qualquer linguagem de programação e algumas delas podem ser responsáveis por coletar dados, os scripts que são escritos com intuito de retornar dados para processamento são chamados coletores.

\begin{figure}
    \centering
    \includegraphics[width=.8\textwidth]{imagens/appa_eng.png}
    \caption{Diagrama demonstrando o funcionamento da ferramenta APPA}
    \label{fig:appa_eng}
\end{figure}

Um dado emitido por um coletor é marcado ou não por uma \textit{tag}, ou marcação, essa responsável por creditar qual coletor será responsável por processar. Por padrão um dado enviado sem destino é marcado com o simbolo \textit{underline}. Toda tarefa dentro do APPA é considerada uma tarefa com múltiplas saídas, logo, é possível enviar dados em lotes e tempo real para que o APPA processe dentro das entidades. 

Entidades de processamento, por sua vez, são compostas por coletores e tarefas de processamento. As tarefas de processamento podem ser síncronas ou assíncronas, ambas executam individualmente por unidade de dado, algo similar a uma linha de dados de um banco, a diferença é seu formato de execução, como o próprio nome fala as tarefas assíncronas não respeitam ordem de execução e são executadas paralelamente pelo processador. Alem disso existem as redutoras e/ou mapeadoras, que consumirão todo o banco gerado após a execução das tarefas síncronas e assíncronas e retornara um novo estado para o banco total. Cada entidade gera seus dados em um banco incorporado isolado para os dados processados por elas.

Diferente de outras ferramentas, o processo de mineração nunca é interrompido. Qualquer erro que aconteça na aplicação é gerenciado por uma camada e registrado em um banco de erros, após isso o processamento segue para os demais dados.

\section{Coleta de Dados}
A coleta é feita utilizando a API publica do twitter, o link para a documentção é \url{https://developer.twitter.com/en/docs}. Será trabalhado no projeto duas entidades: Tweet e Usuário. O tweet é a entidade que representa a publicação do usuário, enquanto o usuário contém informações necessárias sobre o seu perfil.

Para que seja possível acessar a API é necessário criar uma conta de desenvolvimento e gerar o \textit{token} de acesso através do link \url{https://apps.twitter.com/app/new}. Com a chave em mãos é possível replicar o arquivo /collector/client/.env\_sample dentro do projeto Dumont para /collector/client/.env, e conforme demonstrado na figura {ref}, completar os campos necessários.

Para rodar basta executar o comando \textit{docker run -it --rm --name dumont -v \$"PWD":/usr/src/app -w /usr/src/app node:9 node collector/twitter.js}, e verá a saída de dados.

Uma vez configurado o coletor, ainda é necessário entender e configurar as outras tarefas para que o APPA funcione apropriadamente, logo, é necessário entender como essas entidades ficarão no final e quais as tarefas que realização essa manipulação.


\subsection{Mineração}

Existe um outros serviços contidos dentro do projeto o \textit{dumont/specialist\_api} e \textit{dumont/specialist\_app} que gera uma API e uma interface gráfica para injeção de dados especialistas. Para subir ambos os serviços basta utilizar o comando \textit{docker-compose -f docker-compose.dev.yml up specialist\_app}. Entretanto, é necessário de um usuário para injetar as analises, é possível criar o documento na mão dentro do mongo, porem, existe um binário dentro da pasta chamado \textit{dumont/create\_specialist}, basta executa-lo passando o e-mail e ele te devolvera uma senha aleatória. Então basta acessar \textit{\url{http://127.0.0.1:3000/}} e utilizar os dados para entrar no sistema. Logo que autenticado você encontrara uma tela igual a da figura \ref{fig:specialist}, nela existe o tweet, uma área para adicionar perguntas da EADS relacionadas, e um local onde pode-se identificar palavras chaves dentro daquele tweet.

\begin{figure}
    \centering
    \includegraphics[width=1\textwidth]{imagens/specialist.png}
    \caption{Interface de captação de respostas dos especialistas}
    \label{fig:specialist}
\end{figure}

Com o sistema rodando, e dados sendo coletados e analisados é necessária uma amostragem para melhor assertividade e desenvolvimento. Durante a pesquisa, em uma preliminar foram coletados mais de 160GB de dados. A amostra foi tirada antes mesmo da inserção de dados especialistas, logo é necessário conhecer os \textit{scripts} de seleção do dumont.


% \chapter{RESULTADOS, ANÁLISE E DISCUSSÃO}

% \chapter*[Considerações Finais]{Considerações Finais}





% ----------------------------------------------------------
% ELEMENTOS PÓS-TEXTUAIS
% ----------------------------------------------------------
\postextual
% ----------------------------------------------------------

% ----------------------------------------------------------
% Referências bibliográficas
% ----------------------------------------------------------

\bibliography{references}


% Inicia os apêndices
% \begin{apendicesenv}


% \partapendices % Imprime uma página indicando o início dos apêndices


% \chapter{Quisque libero justo}







% \end{apendicesenv}



% ----------------------------------------------------------
% Anexos
% ----------------------------------------------------------

% ---
% Inicia os anexos 
% ---
% \begin{anexosenv}


% \partanexos  % Imprime uma página indicando o início dos anexos

% ----
% \chapter{Primeiro}

% Este anexo é sobre como se pode ficar rico. Para ficar rico...





% \end{anexosenv}


\end{document}
